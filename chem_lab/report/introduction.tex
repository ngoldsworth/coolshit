\section{Introduciton}

This experiment centered on the relationship between reactants, products,
and the role that heat often plays in a chemical reaction.
Specifically, this experiment attempted to determine the heats of reaction for several processes:
\koh dissolving in water,
\koh reacting with aqueous \hcl
and aqueous \koh reacting with aqueous \hcl.
The particular concept of interest in this experiment is Hess's Law,
which states that the total change in enthalpy between the beginning and end of a particular reaction is
independent of the path the reaction takes through the parameter space \cite{hess2019}. 
The application of this concept in this lab is to show that
--for two different ways of completing the same reaction between \koh and \ch{HCl}--
the change in enthalpy is identical. 

The main tool used in this experiment was a digital thermistor,
used to record the temperature of a solution over time as the solution evolved in a styrofoam cup.
This setup proved to be a satisfactory calorimeter for the thermal analysis done. Three reactions were observed,
one being a solid \koh reacting with aqueous \hcl, and two more where the same basic reaction occurred, but split into two steps.
This served to be a basic check on Hess's law.


