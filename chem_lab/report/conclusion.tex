\section{Conclusion}

The goal of this experiment was to investigate the application of Hess's law to the 
reaction that a mixture of \koh and \hcl undergoes. In particular, the changes in enthalpy between the entire reaction from solid \koh and aqueous \hcl were expected to match the changes in enthalpy between two successive reactions: the dissolution of \koh in water, and then the mixture of this solution into an aqueous \hcl solution.
In this experiment, it was found that the reactions did mostly satisfy Hess's law, as the sequence of two reactions --at least across the six groups-- generally produces the same change in heat as the single-step reaction done in Part II. This experiment's data supports Hess's law. 

Some error was found. In particular, the $b$-intercept of the linear regression of Part II data
was found to be negative, which would imply that starting with no \koh should mean a loss of heat, which breaks the rule of energy conservation.
Improper data analysis may play a part in this, but more likely, this experimental error comes from a lower level of precision in lab and the fact that this experiment used a colorimeter that was by no means perfect or close to perfect. Another experimenter could use a better calorimeter to produce more precise and, more likely, more accurate results.
