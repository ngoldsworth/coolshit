\begin{figure}[H]
    \begin{center}
	%% Creator: Matplotlib, PGF backend
%%
%% To include the figure in your LaTeX document, write
%%   \input{<filename>.pgf}
%%
%% Make sure the required packages are loaded in your preamble
%%   \usepackage{pgf}
%%
%% Figures using additional raster images can only be included by \input if
%% they are in the same directory as the main LaTeX file. For loading figures
%% from other directories you can use the `import` package
%%   \usepackage{import}
%% and then include the figures with
%%   \import{<path to file>}{<filename>.pgf}
%%
%% Matplotlib used the following preamble
%%
\begingroup%
\makeatletter%
\begin{pgfpicture}%
\pgfpathrectangle{\pgfpointorigin}{\pgfqpoint{5.668056in}{3.133457in}}%
\pgfusepath{use as bounding box, clip}%
\begin{pgfscope}%
\pgfsetbuttcap%
\pgfsetmiterjoin%
\definecolor{currentfill}{rgb}{1.000000,1.000000,1.000000}%
\pgfsetfillcolor{currentfill}%
\pgfsetlinewidth{0.000000pt}%
\definecolor{currentstroke}{rgb}{1.000000,1.000000,1.000000}%
\pgfsetstrokecolor{currentstroke}%
\pgfsetdash{}{0pt}%
\pgfpathmoveto{\pgfqpoint{0.000000in}{0.000000in}}%
\pgfpathlineto{\pgfqpoint{5.668056in}{0.000000in}}%
\pgfpathlineto{\pgfqpoint{5.668056in}{3.133457in}}%
\pgfpathlineto{\pgfqpoint{0.000000in}{3.133457in}}%
\pgfpathclose%
\pgfusepath{fill}%
\end{pgfscope}%
\begin{pgfscope}%
\pgfsetbuttcap%
\pgfsetmiterjoin%
\definecolor{currentfill}{rgb}{1.000000,1.000000,1.000000}%
\pgfsetfillcolor{currentfill}%
\pgfsetlinewidth{0.000000pt}%
\definecolor{currentstroke}{rgb}{0.000000,0.000000,0.000000}%
\pgfsetstrokecolor{currentstroke}%
\pgfsetstrokeopacity{0.000000}%
\pgfsetdash{}{0pt}%
\pgfpathmoveto{\pgfqpoint{0.530556in}{0.515123in}}%
\pgfpathlineto{\pgfqpoint{5.568056in}{0.515123in}}%
\pgfpathlineto{\pgfqpoint{5.568056in}{2.825123in}}%
\pgfpathlineto{\pgfqpoint{0.530556in}{2.825123in}}%
\pgfpathclose%
\pgfusepath{fill}%
\end{pgfscope}%
\begin{pgfscope}%
\pgfsetbuttcap%
\pgfsetroundjoin%
\definecolor{currentfill}{rgb}{0.000000,0.000000,0.000000}%
\pgfsetfillcolor{currentfill}%
\pgfsetlinewidth{0.803000pt}%
\definecolor{currentstroke}{rgb}{0.000000,0.000000,0.000000}%
\pgfsetstrokecolor{currentstroke}%
\pgfsetdash{}{0pt}%
\pgfsys@defobject{currentmarker}{\pgfqpoint{0.000000in}{-0.048611in}}{\pgfqpoint{0.000000in}{0.000000in}}{%
\pgfpathmoveto{\pgfqpoint{0.000000in}{0.000000in}}%
\pgfpathlineto{\pgfqpoint{0.000000in}{-0.048611in}}%
\pgfusepath{stroke,fill}%
}%
\begin{pgfscope}%
\pgfsys@transformshift{0.759533in}{0.515123in}%
\pgfsys@useobject{currentmarker}{}%
\end{pgfscope}%
\end{pgfscope}%
\begin{pgfscope}%
\definecolor{textcolor}{rgb}{0.000000,0.000000,0.000000}%
\pgfsetstrokecolor{textcolor}%
\pgfsetfillcolor{textcolor}%
\pgftext[x=0.759533in,y=0.417901in,,top]{\color{textcolor}\rmfamily\fontsize{10.000000}{12.000000}\selectfont \(\displaystyle 0\)}%
\end{pgfscope}%
\begin{pgfscope}%
\pgfsetbuttcap%
\pgfsetroundjoin%
\definecolor{currentfill}{rgb}{0.000000,0.000000,0.000000}%
\pgfsetfillcolor{currentfill}%
\pgfsetlinewidth{0.803000pt}%
\definecolor{currentstroke}{rgb}{0.000000,0.000000,0.000000}%
\pgfsetstrokecolor{currentstroke}%
\pgfsetdash{}{0pt}%
\pgfsys@defobject{currentmarker}{\pgfqpoint{0.000000in}{-0.048611in}}{\pgfqpoint{0.000000in}{0.000000in}}{%
\pgfpathmoveto{\pgfqpoint{0.000000in}{0.000000in}}%
\pgfpathlineto{\pgfqpoint{0.000000in}{-0.048611in}}%
\pgfusepath{stroke,fill}%
}%
\begin{pgfscope}%
\pgfsys@transformshift{1.505349in}{0.515123in}%
\pgfsys@useobject{currentmarker}{}%
\end{pgfscope}%
\end{pgfscope}%
\begin{pgfscope}%
\definecolor{textcolor}{rgb}{0.000000,0.000000,0.000000}%
\pgfsetstrokecolor{textcolor}%
\pgfsetfillcolor{textcolor}%
\pgftext[x=1.505349in,y=0.417901in,,top]{\color{textcolor}\rmfamily\fontsize{10.000000}{12.000000}\selectfont \(\displaystyle 50\)}%
\end{pgfscope}%
\begin{pgfscope}%
\pgfsetbuttcap%
\pgfsetroundjoin%
\definecolor{currentfill}{rgb}{0.000000,0.000000,0.000000}%
\pgfsetfillcolor{currentfill}%
\pgfsetlinewidth{0.803000pt}%
\definecolor{currentstroke}{rgb}{0.000000,0.000000,0.000000}%
\pgfsetstrokecolor{currentstroke}%
\pgfsetdash{}{0pt}%
\pgfsys@defobject{currentmarker}{\pgfqpoint{0.000000in}{-0.048611in}}{\pgfqpoint{0.000000in}{0.000000in}}{%
\pgfpathmoveto{\pgfqpoint{0.000000in}{0.000000in}}%
\pgfpathlineto{\pgfqpoint{0.000000in}{-0.048611in}}%
\pgfusepath{stroke,fill}%
}%
\begin{pgfscope}%
\pgfsys@transformshift{2.251164in}{0.515123in}%
\pgfsys@useobject{currentmarker}{}%
\end{pgfscope}%
\end{pgfscope}%
\begin{pgfscope}%
\definecolor{textcolor}{rgb}{0.000000,0.000000,0.000000}%
\pgfsetstrokecolor{textcolor}%
\pgfsetfillcolor{textcolor}%
\pgftext[x=2.251164in,y=0.417901in,,top]{\color{textcolor}\rmfamily\fontsize{10.000000}{12.000000}\selectfont \(\displaystyle 100\)}%
\end{pgfscope}%
\begin{pgfscope}%
\pgfsetbuttcap%
\pgfsetroundjoin%
\definecolor{currentfill}{rgb}{0.000000,0.000000,0.000000}%
\pgfsetfillcolor{currentfill}%
\pgfsetlinewidth{0.803000pt}%
\definecolor{currentstroke}{rgb}{0.000000,0.000000,0.000000}%
\pgfsetstrokecolor{currentstroke}%
\pgfsetdash{}{0pt}%
\pgfsys@defobject{currentmarker}{\pgfqpoint{0.000000in}{-0.048611in}}{\pgfqpoint{0.000000in}{0.000000in}}{%
\pgfpathmoveto{\pgfqpoint{0.000000in}{0.000000in}}%
\pgfpathlineto{\pgfqpoint{0.000000in}{-0.048611in}}%
\pgfusepath{stroke,fill}%
}%
\begin{pgfscope}%
\pgfsys@transformshift{2.996980in}{0.515123in}%
\pgfsys@useobject{currentmarker}{}%
\end{pgfscope}%
\end{pgfscope}%
\begin{pgfscope}%
\definecolor{textcolor}{rgb}{0.000000,0.000000,0.000000}%
\pgfsetstrokecolor{textcolor}%
\pgfsetfillcolor{textcolor}%
\pgftext[x=2.996980in,y=0.417901in,,top]{\color{textcolor}\rmfamily\fontsize{10.000000}{12.000000}\selectfont \(\displaystyle 150\)}%
\end{pgfscope}%
\begin{pgfscope}%
\pgfsetbuttcap%
\pgfsetroundjoin%
\definecolor{currentfill}{rgb}{0.000000,0.000000,0.000000}%
\pgfsetfillcolor{currentfill}%
\pgfsetlinewidth{0.803000pt}%
\definecolor{currentstroke}{rgb}{0.000000,0.000000,0.000000}%
\pgfsetstrokecolor{currentstroke}%
\pgfsetdash{}{0pt}%
\pgfsys@defobject{currentmarker}{\pgfqpoint{0.000000in}{-0.048611in}}{\pgfqpoint{0.000000in}{0.000000in}}{%
\pgfpathmoveto{\pgfqpoint{0.000000in}{0.000000in}}%
\pgfpathlineto{\pgfqpoint{0.000000in}{-0.048611in}}%
\pgfusepath{stroke,fill}%
}%
\begin{pgfscope}%
\pgfsys@transformshift{3.742795in}{0.515123in}%
\pgfsys@useobject{currentmarker}{}%
\end{pgfscope}%
\end{pgfscope}%
\begin{pgfscope}%
\definecolor{textcolor}{rgb}{0.000000,0.000000,0.000000}%
\pgfsetstrokecolor{textcolor}%
\pgfsetfillcolor{textcolor}%
\pgftext[x=3.742795in,y=0.417901in,,top]{\color{textcolor}\rmfamily\fontsize{10.000000}{12.000000}\selectfont \(\displaystyle 200\)}%
\end{pgfscope}%
\begin{pgfscope}%
\pgfsetbuttcap%
\pgfsetroundjoin%
\definecolor{currentfill}{rgb}{0.000000,0.000000,0.000000}%
\pgfsetfillcolor{currentfill}%
\pgfsetlinewidth{0.803000pt}%
\definecolor{currentstroke}{rgb}{0.000000,0.000000,0.000000}%
\pgfsetstrokecolor{currentstroke}%
\pgfsetdash{}{0pt}%
\pgfsys@defobject{currentmarker}{\pgfqpoint{0.000000in}{-0.048611in}}{\pgfqpoint{0.000000in}{0.000000in}}{%
\pgfpathmoveto{\pgfqpoint{0.000000in}{0.000000in}}%
\pgfpathlineto{\pgfqpoint{0.000000in}{-0.048611in}}%
\pgfusepath{stroke,fill}%
}%
\begin{pgfscope}%
\pgfsys@transformshift{4.488610in}{0.515123in}%
\pgfsys@useobject{currentmarker}{}%
\end{pgfscope}%
\end{pgfscope}%
\begin{pgfscope}%
\definecolor{textcolor}{rgb}{0.000000,0.000000,0.000000}%
\pgfsetstrokecolor{textcolor}%
\pgfsetfillcolor{textcolor}%
\pgftext[x=4.488610in,y=0.417901in,,top]{\color{textcolor}\rmfamily\fontsize{10.000000}{12.000000}\selectfont \(\displaystyle 250\)}%
\end{pgfscope}%
\begin{pgfscope}%
\pgfsetbuttcap%
\pgfsetroundjoin%
\definecolor{currentfill}{rgb}{0.000000,0.000000,0.000000}%
\pgfsetfillcolor{currentfill}%
\pgfsetlinewidth{0.803000pt}%
\definecolor{currentstroke}{rgb}{0.000000,0.000000,0.000000}%
\pgfsetstrokecolor{currentstroke}%
\pgfsetdash{}{0pt}%
\pgfsys@defobject{currentmarker}{\pgfqpoint{0.000000in}{-0.048611in}}{\pgfqpoint{0.000000in}{0.000000in}}{%
\pgfpathmoveto{\pgfqpoint{0.000000in}{0.000000in}}%
\pgfpathlineto{\pgfqpoint{0.000000in}{-0.048611in}}%
\pgfusepath{stroke,fill}%
}%
\begin{pgfscope}%
\pgfsys@transformshift{5.234426in}{0.515123in}%
\pgfsys@useobject{currentmarker}{}%
\end{pgfscope}%
\end{pgfscope}%
\begin{pgfscope}%
\definecolor{textcolor}{rgb}{0.000000,0.000000,0.000000}%
\pgfsetstrokecolor{textcolor}%
\pgfsetfillcolor{textcolor}%
\pgftext[x=5.234426in,y=0.417901in,,top]{\color{textcolor}\rmfamily\fontsize{10.000000}{12.000000}\selectfont \(\displaystyle 300\)}%
\end{pgfscope}%
\begin{pgfscope}%
\definecolor{textcolor}{rgb}{0.000000,0.000000,0.000000}%
\pgfsetstrokecolor{textcolor}%
\pgfsetfillcolor{textcolor}%
\pgftext[x=3.049306in,y=0.238889in,,top]{\color{textcolor}\rmfamily\fontsize{10.000000}{12.000000}\selectfont Time (seconds)}%
\end{pgfscope}%
\begin{pgfscope}%
\pgfsetbuttcap%
\pgfsetroundjoin%
\definecolor{currentfill}{rgb}{0.000000,0.000000,0.000000}%
\pgfsetfillcolor{currentfill}%
\pgfsetlinewidth{0.803000pt}%
\definecolor{currentstroke}{rgb}{0.000000,0.000000,0.000000}%
\pgfsetstrokecolor{currentstroke}%
\pgfsetdash{}{0pt}%
\pgfsys@defobject{currentmarker}{\pgfqpoint{-0.048611in}{0.000000in}}{\pgfqpoint{0.000000in}{0.000000in}}{%
\pgfpathmoveto{\pgfqpoint{0.000000in}{0.000000in}}%
\pgfpathlineto{\pgfqpoint{-0.048611in}{0.000000in}}%
\pgfusepath{stroke,fill}%
}%
\begin{pgfscope}%
\pgfsys@transformshift{0.530556in}{0.788520in}%
\pgfsys@useobject{currentmarker}{}%
\end{pgfscope}%
\end{pgfscope}%
\begin{pgfscope}%
\definecolor{textcolor}{rgb}{0.000000,0.000000,0.000000}%
\pgfsetstrokecolor{textcolor}%
\pgfsetfillcolor{textcolor}%
\pgftext[x=0.294444in,y=0.740295in,left,base]{\color{textcolor}\rmfamily\fontsize{10.000000}{12.000000}\selectfont \(\displaystyle 21\)}%
\end{pgfscope}%
\begin{pgfscope}%
\pgfsetbuttcap%
\pgfsetroundjoin%
\definecolor{currentfill}{rgb}{0.000000,0.000000,0.000000}%
\pgfsetfillcolor{currentfill}%
\pgfsetlinewidth{0.803000pt}%
\definecolor{currentstroke}{rgb}{0.000000,0.000000,0.000000}%
\pgfsetstrokecolor{currentstroke}%
\pgfsetdash{}{0pt}%
\pgfsys@defobject{currentmarker}{\pgfqpoint{-0.048611in}{0.000000in}}{\pgfqpoint{0.000000in}{0.000000in}}{%
\pgfpathmoveto{\pgfqpoint{0.000000in}{0.000000in}}%
\pgfpathlineto{\pgfqpoint{-0.048611in}{0.000000in}}%
\pgfusepath{stroke,fill}%
}%
\begin{pgfscope}%
\pgfsys@transformshift{0.530556in}{1.129295in}%
\pgfsys@useobject{currentmarker}{}%
\end{pgfscope}%
\end{pgfscope}%
\begin{pgfscope}%
\definecolor{textcolor}{rgb}{0.000000,0.000000,0.000000}%
\pgfsetstrokecolor{textcolor}%
\pgfsetfillcolor{textcolor}%
\pgftext[x=0.294444in,y=1.081070in,left,base]{\color{textcolor}\rmfamily\fontsize{10.000000}{12.000000}\selectfont \(\displaystyle 22\)}%
\end{pgfscope}%
\begin{pgfscope}%
\pgfsetbuttcap%
\pgfsetroundjoin%
\definecolor{currentfill}{rgb}{0.000000,0.000000,0.000000}%
\pgfsetfillcolor{currentfill}%
\pgfsetlinewidth{0.803000pt}%
\definecolor{currentstroke}{rgb}{0.000000,0.000000,0.000000}%
\pgfsetstrokecolor{currentstroke}%
\pgfsetdash{}{0pt}%
\pgfsys@defobject{currentmarker}{\pgfqpoint{-0.048611in}{0.000000in}}{\pgfqpoint{0.000000in}{0.000000in}}{%
\pgfpathmoveto{\pgfqpoint{0.000000in}{0.000000in}}%
\pgfpathlineto{\pgfqpoint{-0.048611in}{0.000000in}}%
\pgfusepath{stroke,fill}%
}%
\begin{pgfscope}%
\pgfsys@transformshift{0.530556in}{1.470070in}%
\pgfsys@useobject{currentmarker}{}%
\end{pgfscope}%
\end{pgfscope}%
\begin{pgfscope}%
\definecolor{textcolor}{rgb}{0.000000,0.000000,0.000000}%
\pgfsetstrokecolor{textcolor}%
\pgfsetfillcolor{textcolor}%
\pgftext[x=0.294444in,y=1.421845in,left,base]{\color{textcolor}\rmfamily\fontsize{10.000000}{12.000000}\selectfont \(\displaystyle 23\)}%
\end{pgfscope}%
\begin{pgfscope}%
\pgfsetbuttcap%
\pgfsetroundjoin%
\definecolor{currentfill}{rgb}{0.000000,0.000000,0.000000}%
\pgfsetfillcolor{currentfill}%
\pgfsetlinewidth{0.803000pt}%
\definecolor{currentstroke}{rgb}{0.000000,0.000000,0.000000}%
\pgfsetstrokecolor{currentstroke}%
\pgfsetdash{}{0pt}%
\pgfsys@defobject{currentmarker}{\pgfqpoint{-0.048611in}{0.000000in}}{\pgfqpoint{0.000000in}{0.000000in}}{%
\pgfpathmoveto{\pgfqpoint{0.000000in}{0.000000in}}%
\pgfpathlineto{\pgfqpoint{-0.048611in}{0.000000in}}%
\pgfusepath{stroke,fill}%
}%
\begin{pgfscope}%
\pgfsys@transformshift{0.530556in}{1.810845in}%
\pgfsys@useobject{currentmarker}{}%
\end{pgfscope}%
\end{pgfscope}%
\begin{pgfscope}%
\definecolor{textcolor}{rgb}{0.000000,0.000000,0.000000}%
\pgfsetstrokecolor{textcolor}%
\pgfsetfillcolor{textcolor}%
\pgftext[x=0.294444in,y=1.762620in,left,base]{\color{textcolor}\rmfamily\fontsize{10.000000}{12.000000}\selectfont \(\displaystyle 24\)}%
\end{pgfscope}%
\begin{pgfscope}%
\pgfsetbuttcap%
\pgfsetroundjoin%
\definecolor{currentfill}{rgb}{0.000000,0.000000,0.000000}%
\pgfsetfillcolor{currentfill}%
\pgfsetlinewidth{0.803000pt}%
\definecolor{currentstroke}{rgb}{0.000000,0.000000,0.000000}%
\pgfsetstrokecolor{currentstroke}%
\pgfsetdash{}{0pt}%
\pgfsys@defobject{currentmarker}{\pgfqpoint{-0.048611in}{0.000000in}}{\pgfqpoint{0.000000in}{0.000000in}}{%
\pgfpathmoveto{\pgfqpoint{0.000000in}{0.000000in}}%
\pgfpathlineto{\pgfqpoint{-0.048611in}{0.000000in}}%
\pgfusepath{stroke,fill}%
}%
\begin{pgfscope}%
\pgfsys@transformshift{0.530556in}{2.151620in}%
\pgfsys@useobject{currentmarker}{}%
\end{pgfscope}%
\end{pgfscope}%
\begin{pgfscope}%
\definecolor{textcolor}{rgb}{0.000000,0.000000,0.000000}%
\pgfsetstrokecolor{textcolor}%
\pgfsetfillcolor{textcolor}%
\pgftext[x=0.294444in,y=2.103395in,left,base]{\color{textcolor}\rmfamily\fontsize{10.000000}{12.000000}\selectfont \(\displaystyle 25\)}%
\end{pgfscope}%
\begin{pgfscope}%
\pgfsetbuttcap%
\pgfsetroundjoin%
\definecolor{currentfill}{rgb}{0.000000,0.000000,0.000000}%
\pgfsetfillcolor{currentfill}%
\pgfsetlinewidth{0.803000pt}%
\definecolor{currentstroke}{rgb}{0.000000,0.000000,0.000000}%
\pgfsetstrokecolor{currentstroke}%
\pgfsetdash{}{0pt}%
\pgfsys@defobject{currentmarker}{\pgfqpoint{-0.048611in}{0.000000in}}{\pgfqpoint{0.000000in}{0.000000in}}{%
\pgfpathmoveto{\pgfqpoint{0.000000in}{0.000000in}}%
\pgfpathlineto{\pgfqpoint{-0.048611in}{0.000000in}}%
\pgfusepath{stroke,fill}%
}%
\begin{pgfscope}%
\pgfsys@transformshift{0.530556in}{2.492395in}%
\pgfsys@useobject{currentmarker}{}%
\end{pgfscope}%
\end{pgfscope}%
\begin{pgfscope}%
\definecolor{textcolor}{rgb}{0.000000,0.000000,0.000000}%
\pgfsetstrokecolor{textcolor}%
\pgfsetfillcolor{textcolor}%
\pgftext[x=0.294444in,y=2.444170in,left,base]{\color{textcolor}\rmfamily\fontsize{10.000000}{12.000000}\selectfont \(\displaystyle 26\)}%
\end{pgfscope}%
\begin{pgfscope}%
\definecolor{textcolor}{rgb}{0.000000,0.000000,0.000000}%
\pgfsetstrokecolor{textcolor}%
\pgfsetfillcolor{textcolor}%
\pgftext[x=0.238889in,y=1.670123in,,bottom,rotate=90.000000]{\color{textcolor}\rmfamily\fontsize{10.000000}{12.000000}\selectfont Temperature (C)}%
\end{pgfscope}%
\begin{pgfscope}%
\pgfpathrectangle{\pgfqpoint{0.530556in}{0.515123in}}{\pgfqpoint{5.037500in}{2.310000in}}%
\pgfusepath{clip}%
\pgfsetrectcap%
\pgfsetroundjoin%
\pgfsetlinewidth{1.505625pt}%
\definecolor{currentstroke}{rgb}{1.000000,0.000000,0.000000}%
\pgfsetstrokecolor{currentstroke}%
\pgfsetdash{}{0pt}%
\pgfpathmoveto{\pgfqpoint{0.759533in}{1.025681in}}%
\pgfpathlineto{\pgfqpoint{0.774226in}{1.025681in}}%
\pgfpathlineto{\pgfqpoint{0.781684in}{1.033153in}}%
\pgfpathlineto{\pgfqpoint{0.789366in}{1.033153in}}%
\pgfpathlineto{\pgfqpoint{0.796824in}{1.025681in}}%
\pgfpathlineto{\pgfqpoint{0.849031in}{1.025681in}}%
\pgfpathlineto{\pgfqpoint{0.863947in}{0.995727in}}%
\pgfpathlineto{\pgfqpoint{0.878864in}{1.010717in}}%
\pgfpathlineto{\pgfqpoint{0.893780in}{1.010717in}}%
\pgfpathlineto{\pgfqpoint{0.901238in}{1.003225in}}%
\pgfpathlineto{\pgfqpoint{0.908696in}{0.988222in}}%
\pgfpathlineto{\pgfqpoint{0.916154in}{0.980711in}}%
\pgfpathlineto{\pgfqpoint{0.923613in}{0.980711in}}%
\pgfpathlineto{\pgfqpoint{0.931071in}{0.988222in}}%
\pgfpathlineto{\pgfqpoint{0.938768in}{1.003225in}}%
\pgfpathlineto{\pgfqpoint{0.960903in}{1.003225in}}%
\pgfpathlineto{\pgfqpoint{0.968600in}{1.010717in}}%
\pgfpathlineto{\pgfqpoint{0.975820in}{1.010717in}}%
\pgfpathlineto{\pgfqpoint{0.983278in}{1.018202in}}%
\pgfpathlineto{\pgfqpoint{1.005652in}{1.018202in}}%
\pgfpathlineto{\pgfqpoint{1.013110in}{1.025681in}}%
\pgfpathlineto{\pgfqpoint{1.020807in}{1.025681in}}%
\pgfpathlineto{\pgfqpoint{1.028265in}{1.018202in}}%
\pgfpathlineto{\pgfqpoint{1.065556in}{1.018202in}}%
\pgfpathlineto{\pgfqpoint{1.080473in}{1.003225in}}%
\pgfpathlineto{\pgfqpoint{1.087931in}{1.010717in}}%
\pgfpathlineto{\pgfqpoint{1.095389in}{1.010717in}}%
\pgfpathlineto{\pgfqpoint{1.102847in}{1.018202in}}%
\pgfpathlineto{\pgfqpoint{1.110305in}{1.010717in}}%
\pgfpathlineto{\pgfqpoint{1.162512in}{1.010717in}}%
\pgfpathlineto{\pgfqpoint{1.170194in}{1.018202in}}%
\pgfpathlineto{\pgfqpoint{1.192345in}{1.018202in}}%
\pgfpathlineto{\pgfqpoint{1.200027in}{1.010717in}}%
\pgfpathlineto{\pgfqpoint{1.214719in}{1.010717in}}%
\pgfpathlineto{\pgfqpoint{1.222401in}{1.018202in}}%
\pgfpathlineto{\pgfqpoint{1.229859in}{1.018202in}}%
\pgfpathlineto{\pgfqpoint{1.237094in}{1.025681in}}%
\pgfpathlineto{\pgfqpoint{1.244776in}{1.025681in}}%
\pgfpathlineto{\pgfqpoint{1.252234in}{1.018202in}}%
\pgfpathlineto{\pgfqpoint{1.259692in}{1.018202in}}%
\pgfpathlineto{\pgfqpoint{1.266926in}{1.010717in}}%
\pgfpathlineto{\pgfqpoint{1.274608in}{0.995727in}}%
\pgfpathlineto{\pgfqpoint{1.282066in}{1.003225in}}%
\pgfpathlineto{\pgfqpoint{1.289525in}{1.003225in}}%
\pgfpathlineto{\pgfqpoint{1.304441in}{1.018202in}}%
\pgfpathlineto{\pgfqpoint{1.408855in}{1.018202in}}%
\pgfpathlineto{\pgfqpoint{1.416313in}{1.010717in}}%
\pgfpathlineto{\pgfqpoint{1.431468in}{1.010717in}}%
\pgfpathlineto{\pgfqpoint{1.438688in}{1.003225in}}%
\pgfpathlineto{\pgfqpoint{1.535882in}{1.003225in}}%
\pgfpathlineto{\pgfqpoint{1.543341in}{1.010717in}}%
\pgfpathlineto{\pgfqpoint{1.551037in}{1.003225in}}%
\pgfpathlineto{\pgfqpoint{1.558257in}{1.010717in}}%
\pgfpathlineto{\pgfqpoint{1.565715in}{1.003225in}}%
\pgfpathlineto{\pgfqpoint{1.573173in}{1.010717in}}%
\pgfpathlineto{\pgfqpoint{1.617922in}{1.010717in}}%
\pgfpathlineto{\pgfqpoint{1.625380in}{1.003225in}}%
\pgfpathlineto{\pgfqpoint{1.633077in}{1.010717in}}%
\pgfpathlineto{\pgfqpoint{1.640535in}{1.010717in}}%
\pgfpathlineto{\pgfqpoint{1.647755in}{1.003225in}}%
\pgfpathlineto{\pgfqpoint{1.655451in}{1.010717in}}%
\pgfpathlineto{\pgfqpoint{1.685284in}{1.010717in}}%
\pgfpathlineto{\pgfqpoint{1.692742in}{0.995727in}}%
\pgfpathlineto{\pgfqpoint{1.699962in}{0.965669in}}%
\pgfpathlineto{\pgfqpoint{1.707659in}{0.973194in}}%
\pgfpathlineto{\pgfqpoint{1.722575in}{1.003225in}}%
\pgfpathlineto{\pgfqpoint{1.730033in}{1.010717in}}%
\pgfpathlineto{\pgfqpoint{1.737491in}{1.003225in}}%
\pgfpathlineto{\pgfqpoint{1.752408in}{1.003225in}}%
\pgfpathlineto{\pgfqpoint{1.759866in}{0.995727in}}%
\pgfpathlineto{\pgfqpoint{1.774782in}{0.950601in}}%
\pgfpathlineto{\pgfqpoint{1.780599in}{0.943058in}}%
\pgfpathlineto{\pgfqpoint{1.789698in}{0.950601in}}%
\pgfpathlineto{\pgfqpoint{1.797156in}{0.965669in}}%
\pgfpathlineto{\pgfqpoint{1.804615in}{0.988222in}}%
\pgfpathlineto{\pgfqpoint{1.810671in}{0.995727in}}%
\pgfpathlineto{\pgfqpoint{1.819531in}{1.010717in}}%
\pgfpathlineto{\pgfqpoint{1.826989in}{1.018202in}}%
\pgfpathlineto{\pgfqpoint{1.833045in}{1.018202in}}%
\pgfpathlineto{\pgfqpoint{1.840503in}{1.025681in}}%
\pgfpathlineto{\pgfqpoint{1.915323in}{1.025681in}}%
\pgfpathlineto{\pgfqpoint{1.922782in}{1.018202in}}%
\pgfpathlineto{\pgfqpoint{1.931403in}{1.025681in}}%
\pgfpathlineto{\pgfqpoint{2.027196in}{1.025681in}}%
\pgfpathlineto{\pgfqpoint{2.034654in}{1.018202in}}%
\pgfpathlineto{\pgfqpoint{2.042112in}{1.018202in}}%
\pgfpathlineto{\pgfqpoint{2.049570in}{1.003225in}}%
\pgfpathlineto{\pgfqpoint{2.057028in}{0.980711in}}%
\pgfpathlineto{\pgfqpoint{2.064487in}{0.995727in}}%
\pgfpathlineto{\pgfqpoint{2.079403in}{1.010717in}}%
\pgfpathlineto{\pgfqpoint{2.086861in}{1.010717in}}%
\pgfpathlineto{\pgfqpoint{2.094543in}{1.003225in}}%
\pgfpathlineto{\pgfqpoint{2.116917in}{1.003225in}}%
\pgfpathlineto{\pgfqpoint{2.124375in}{0.988222in}}%
\pgfpathlineto{\pgfqpoint{2.131610in}{0.988222in}}%
\pgfpathlineto{\pgfqpoint{2.139068in}{0.973194in}}%
\pgfpathlineto{\pgfqpoint{2.146750in}{0.980711in}}%
\pgfpathlineto{\pgfqpoint{2.154208in}{0.980711in}}%
\pgfpathlineto{\pgfqpoint{2.161443in}{0.995727in}}%
\pgfpathlineto{\pgfqpoint{2.169124in}{0.995727in}}%
\pgfpathlineto{\pgfqpoint{2.176583in}{1.003225in}}%
\pgfpathlineto{\pgfqpoint{2.206415in}{1.003225in}}%
\pgfpathlineto{\pgfqpoint{2.213873in}{1.010717in}}%
\pgfpathlineto{\pgfqpoint{2.221332in}{1.010717in}}%
\pgfpathlineto{\pgfqpoint{2.228790in}{1.003225in}}%
\pgfpathlineto{\pgfqpoint{2.236248in}{1.010717in}}%
\pgfpathlineto{\pgfqpoint{2.251164in}{1.010717in}}%
\pgfpathlineto{\pgfqpoint{2.258622in}{1.018202in}}%
\pgfpathlineto{\pgfqpoint{2.266080in}{1.010717in}}%
\pgfpathlineto{\pgfqpoint{2.280997in}{1.010717in}}%
\pgfpathlineto{\pgfqpoint{2.288455in}{1.003225in}}%
\pgfpathlineto{\pgfqpoint{2.303610in}{1.003225in}}%
\pgfpathlineto{\pgfqpoint{2.310829in}{1.010717in}}%
\pgfpathlineto{\pgfqpoint{2.318288in}{1.010717in}}%
\pgfpathlineto{\pgfqpoint{2.333204in}{0.995727in}}%
\pgfpathlineto{\pgfqpoint{2.340662in}{1.003225in}}%
\pgfpathlineto{\pgfqpoint{2.348359in}{1.003225in}}%
\pgfpathlineto{\pgfqpoint{2.355817in}{1.010717in}}%
\pgfpathlineto{\pgfqpoint{2.363036in}{1.010717in}}%
\pgfpathlineto{\pgfqpoint{2.370495in}{1.003225in}}%
\pgfpathlineto{\pgfqpoint{2.392869in}{1.003225in}}%
\pgfpathlineto{\pgfqpoint{2.400566in}{1.010717in}}%
\pgfpathlineto{\pgfqpoint{2.422940in}{1.010717in}}%
\pgfpathlineto{\pgfqpoint{2.430398in}{1.003225in}}%
\pgfpathlineto{\pgfqpoint{2.445315in}{1.003225in}}%
\pgfpathlineto{\pgfqpoint{2.460231in}{1.018202in}}%
\pgfpathlineto{\pgfqpoint{2.467689in}{1.010717in}}%
\pgfpathlineto{\pgfqpoint{2.475371in}{1.010717in}}%
\pgfpathlineto{\pgfqpoint{2.482606in}{1.018202in}}%
\pgfpathlineto{\pgfqpoint{2.490064in}{1.010717in}}%
\pgfpathlineto{\pgfqpoint{2.497522in}{1.018202in}}%
\pgfpathlineto{\pgfqpoint{2.542271in}{1.018202in}}%
\pgfpathlineto{\pgfqpoint{2.549729in}{1.025681in}}%
\pgfpathlineto{\pgfqpoint{2.609618in}{1.025681in}}%
\pgfpathlineto{\pgfqpoint{2.617076in}{1.033153in}}%
\pgfpathlineto{\pgfqpoint{2.624311in}{1.025681in}}%
\pgfpathlineto{\pgfqpoint{2.631992in}{1.033153in}}%
\pgfpathlineto{\pgfqpoint{2.639451in}{1.033153in}}%
\pgfpathlineto{\pgfqpoint{2.646909in}{1.025681in}}%
\pgfpathlineto{\pgfqpoint{2.654367in}{1.025681in}}%
\pgfpathlineto{\pgfqpoint{2.661825in}{1.033153in}}%
\pgfpathlineto{\pgfqpoint{2.691658in}{1.033153in}}%
\pgfpathlineto{\pgfqpoint{2.706813in}{1.062978in}}%
\pgfpathlineto{\pgfqpoint{2.721490in}{1.195940in}}%
\pgfpathlineto{\pgfqpoint{2.729187in}{1.210589in}}%
\pgfpathlineto{\pgfqpoint{2.736645in}{1.217904in}}%
\pgfpathlineto{\pgfqpoint{2.743865in}{1.203267in}}%
\pgfpathlineto{\pgfqpoint{2.751323in}{1.210589in}}%
\pgfpathlineto{\pgfqpoint{2.759020in}{1.181266in}}%
\pgfpathlineto{\pgfqpoint{2.773697in}{1.268939in}}%
\pgfpathlineto{\pgfqpoint{2.781394in}{1.362935in}}%
\pgfpathlineto{\pgfqpoint{2.788852in}{1.341333in}}%
\pgfpathlineto{\pgfqpoint{2.796310in}{1.434558in}}%
\pgfpathlineto{\pgfqpoint{2.803530in}{1.427422in}}%
\pgfpathlineto{\pgfqpoint{2.811227in}{1.441688in}}%
\pgfpathlineto{\pgfqpoint{2.818685in}{1.463044in}}%
\pgfpathlineto{\pgfqpoint{2.825904in}{1.533855in}}%
\pgfpathlineto{\pgfqpoint{2.833601in}{1.576070in}}%
\pgfpathlineto{\pgfqpoint{2.841059in}{1.632042in}}%
\pgfpathlineto{\pgfqpoint{2.848518in}{1.818361in}}%
\pgfpathlineto{\pgfqpoint{2.855737in}{1.893155in}}%
\pgfpathlineto{\pgfqpoint{2.863434in}{1.852437in}}%
\pgfpathlineto{\pgfqpoint{2.870892in}{1.832007in}}%
\pgfpathlineto{\pgfqpoint{2.878350in}{1.859236in}}%
\pgfpathlineto{\pgfqpoint{2.885808in}{1.879603in}}%
\pgfpathlineto{\pgfqpoint{2.893266in}{1.777296in}}%
\pgfpathlineto{\pgfqpoint{2.900725in}{1.715337in}}%
\pgfpathlineto{\pgfqpoint{2.908421in}{1.673788in}}%
\pgfpathlineto{\pgfqpoint{2.923099in}{1.632042in}}%
\pgfpathlineto{\pgfqpoint{2.930557in}{1.632042in}}%
\pgfpathlineto{\pgfqpoint{2.938254in}{1.770433in}}%
\pgfpathlineto{\pgfqpoint{2.945474in}{1.749813in}}%
\pgfpathlineto{\pgfqpoint{2.960629in}{1.625065in}}%
\pgfpathlineto{\pgfqpoint{2.968087in}{1.590096in}}%
\pgfpathlineto{\pgfqpoint{2.975306in}{1.562021in}}%
\pgfpathlineto{\pgfqpoint{2.982764in}{1.519739in}}%
\pgfpathlineto{\pgfqpoint{2.990461in}{1.533855in}}%
\pgfpathlineto{\pgfqpoint{2.997919in}{1.708426in}}%
\pgfpathlineto{\pgfqpoint{3.005139in}{1.838822in}}%
\pgfpathlineto{\pgfqpoint{3.012836in}{1.947155in}}%
\pgfpathlineto{\pgfqpoint{3.027752in}{2.060824in}}%
\pgfpathlineto{\pgfqpoint{3.035210in}{2.120417in}}%
\pgfpathlineto{\pgfqpoint{3.042668in}{2.159925in}}%
\pgfpathlineto{\pgfqpoint{3.050126in}{2.186166in}}%
\pgfpathlineto{\pgfqpoint{3.065043in}{2.303303in}}%
\pgfpathlineto{\pgfqpoint{3.072501in}{2.335569in}}%
\pgfpathlineto{\pgfqpoint{3.079959in}{2.374136in}}%
\pgfpathlineto{\pgfqpoint{3.087417in}{2.386954in}}%
\pgfpathlineto{\pgfqpoint{3.094875in}{2.393357in}}%
\pgfpathlineto{\pgfqpoint{3.102333in}{2.393357in}}%
\pgfpathlineto{\pgfqpoint{3.109792in}{2.399754in}}%
\pgfpathlineto{\pgfqpoint{3.117250in}{2.386954in}}%
\pgfpathlineto{\pgfqpoint{3.132166in}{2.374136in}}%
\pgfpathlineto{\pgfqpoint{3.138222in}{2.380547in}}%
\pgfpathlineto{\pgfqpoint{3.154541in}{2.393357in}}%
\pgfpathlineto{\pgfqpoint{3.161999in}{2.425300in}}%
\pgfpathlineto{\pgfqpoint{3.184373in}{2.469829in}}%
\pgfpathlineto{\pgfqpoint{3.197887in}{2.482511in}}%
\pgfpathlineto{\pgfqpoint{3.206748in}{2.488846in}}%
\pgfpathlineto{\pgfqpoint{3.212804in}{2.501501in}}%
\pgfpathlineto{\pgfqpoint{3.227720in}{2.526758in}}%
\pgfpathlineto{\pgfqpoint{3.236580in}{2.533061in}}%
\pgfpathlineto{\pgfqpoint{3.242636in}{2.533061in}}%
\pgfpathlineto{\pgfqpoint{3.257553in}{2.545654in}}%
\pgfpathlineto{\pgfqpoint{3.265011in}{2.558229in}}%
\pgfpathlineto{\pgfqpoint{3.272708in}{2.558229in}}%
\pgfpathlineto{\pgfqpoint{3.288787in}{2.583327in}}%
\pgfpathlineto{\pgfqpoint{3.296469in}{2.583327in}}%
\pgfpathlineto{\pgfqpoint{3.302540in}{2.595849in}}%
\pgfpathlineto{\pgfqpoint{3.317218in}{2.608355in}}%
\pgfpathlineto{\pgfqpoint{3.332373in}{2.595849in}}%
\pgfpathlineto{\pgfqpoint{3.339831in}{2.602104in}}%
\pgfpathlineto{\pgfqpoint{3.354747in}{2.602104in}}%
\pgfpathlineto{\pgfqpoint{3.369664in}{2.589590in}}%
\pgfpathlineto{\pgfqpoint{3.377122in}{2.558229in}}%
\pgfpathlineto{\pgfqpoint{3.385967in}{2.533061in}}%
\pgfpathlineto{\pgfqpoint{3.392038in}{2.558229in}}%
\pgfpathlineto{\pgfqpoint{3.399496in}{2.595849in}}%
\pgfpathlineto{\pgfqpoint{3.406954in}{2.620842in}}%
\pgfpathlineto{\pgfqpoint{3.414412in}{2.627080in}}%
\pgfpathlineto{\pgfqpoint{3.421871in}{2.639541in}}%
\pgfpathlineto{\pgfqpoint{3.436787in}{2.651986in}}%
\pgfpathlineto{\pgfqpoint{3.444245in}{2.664413in}}%
\pgfpathlineto{\pgfqpoint{3.451927in}{2.664413in}}%
\pgfpathlineto{\pgfqpoint{3.459161in}{2.670620in}}%
\pgfpathlineto{\pgfqpoint{3.466620in}{2.670620in}}%
\pgfpathlineto{\pgfqpoint{3.474301in}{2.683021in}}%
\pgfpathlineto{\pgfqpoint{3.481760in}{2.689216in}}%
\pgfpathlineto{\pgfqpoint{3.488994in}{2.689216in}}%
\pgfpathlineto{\pgfqpoint{3.504134in}{2.701591in}}%
\pgfpathlineto{\pgfqpoint{3.548659in}{2.701591in}}%
\pgfpathlineto{\pgfqpoint{3.556341in}{2.707773in}}%
\pgfpathlineto{\pgfqpoint{3.563799in}{2.707773in}}%
\pgfpathlineto{\pgfqpoint{3.571257in}{2.701591in}}%
\pgfpathlineto{\pgfqpoint{3.586174in}{2.701591in}}%
\pgfpathlineto{\pgfqpoint{3.593632in}{2.707773in}}%
\pgfpathlineto{\pgfqpoint{3.608548in}{2.707773in}}%
\pgfpathlineto{\pgfqpoint{3.616006in}{2.713950in}}%
\pgfpathlineto{\pgfqpoint{3.668213in}{2.713950in}}%
\pgfpathlineto{\pgfqpoint{3.675672in}{2.720123in}}%
\pgfpathlineto{\pgfqpoint{3.683368in}{2.720123in}}%
\pgfpathlineto{\pgfqpoint{3.690827in}{2.713950in}}%
\pgfpathlineto{\pgfqpoint{3.705743in}{2.713950in}}%
\pgfpathlineto{\pgfqpoint{3.713201in}{2.720123in}}%
\pgfpathlineto{\pgfqpoint{3.720421in}{2.713950in}}%
\pgfpathlineto{\pgfqpoint{3.787783in}{2.713950in}}%
\pgfpathlineto{\pgfqpoint{3.795241in}{2.707773in}}%
\pgfpathlineto{\pgfqpoint{3.802699in}{2.707773in}}%
\pgfpathlineto{\pgfqpoint{3.810157in}{2.713950in}}%
\pgfpathlineto{\pgfqpoint{3.817615in}{2.713950in}}%
\pgfpathlineto{\pgfqpoint{3.825073in}{2.707773in}}%
\pgfpathlineto{\pgfqpoint{3.854906in}{2.707773in}}%
\pgfpathlineto{\pgfqpoint{3.862588in}{2.701591in}}%
\pgfpathlineto{\pgfqpoint{3.869822in}{2.707773in}}%
\pgfpathlineto{\pgfqpoint{3.899655in}{2.707773in}}%
\pgfpathlineto{\pgfqpoint{3.907337in}{2.701591in}}%
\pgfpathlineto{\pgfqpoint{3.937169in}{2.701591in}}%
\pgfpathlineto{\pgfqpoint{3.944628in}{2.695406in}}%
\pgfpathlineto{\pgfqpoint{3.974460in}{2.695406in}}%
\pgfpathlineto{\pgfqpoint{3.989377in}{2.670620in}}%
\pgfpathlineto{\pgfqpoint{3.996835in}{2.639541in}}%
\pgfpathlineto{\pgfqpoint{4.004293in}{2.627080in}}%
\pgfpathlineto{\pgfqpoint{4.011751in}{2.620842in}}%
\pgfpathlineto{\pgfqpoint{4.019209in}{2.602104in}}%
\pgfpathlineto{\pgfqpoint{4.026667in}{2.539360in}}%
\pgfpathlineto{\pgfqpoint{4.041584in}{2.577059in}}%
\pgfpathlineto{\pgfqpoint{4.049042in}{2.608355in}}%
\pgfpathlineto{\pgfqpoint{4.056500in}{2.620842in}}%
\pgfpathlineto{\pgfqpoint{4.064197in}{2.627080in}}%
\pgfpathlineto{\pgfqpoint{4.078874in}{2.627080in}}%
\pgfpathlineto{\pgfqpoint{4.086571in}{2.614601in}}%
\pgfpathlineto{\pgfqpoint{4.094029in}{2.608355in}}%
\pgfpathlineto{\pgfqpoint{4.123862in}{2.558229in}}%
\pgfpathlineto{\pgfqpoint{4.131081in}{2.558229in}}%
\pgfpathlineto{\pgfqpoint{4.138778in}{2.545654in}}%
\pgfpathlineto{\pgfqpoint{4.146236in}{2.539360in}}%
\pgfpathlineto{\pgfqpoint{4.153695in}{2.545654in}}%
\pgfpathlineto{\pgfqpoint{4.160914in}{2.558229in}}%
\pgfpathlineto{\pgfqpoint{4.168611in}{2.564510in}}%
\pgfpathlineto{\pgfqpoint{4.176069in}{2.558229in}}%
\pgfpathlineto{\pgfqpoint{4.183527in}{2.570787in}}%
\pgfpathlineto{\pgfqpoint{4.190985in}{2.595849in}}%
\pgfpathlineto{\pgfqpoint{4.198443in}{2.614601in}}%
\pgfpathlineto{\pgfqpoint{4.205902in}{2.620842in}}%
\pgfpathlineto{\pgfqpoint{4.213598in}{2.620842in}}%
\pgfpathlineto{\pgfqpoint{4.235734in}{2.583327in}}%
\pgfpathlineto{\pgfqpoint{4.243192in}{2.577059in}}%
\pgfpathlineto{\pgfqpoint{4.250651in}{2.577059in}}%
\pgfpathlineto{\pgfqpoint{4.258109in}{2.558229in}}%
\pgfpathlineto{\pgfqpoint{4.265806in}{2.551944in}}%
\pgfpathlineto{\pgfqpoint{4.273025in}{2.551944in}}%
\pgfpathlineto{\pgfqpoint{4.280483in}{2.539360in}}%
\pgfpathlineto{\pgfqpoint{4.287941in}{2.533061in}}%
\pgfpathlineto{\pgfqpoint{4.295638in}{2.545654in}}%
\pgfpathlineto{\pgfqpoint{4.310316in}{2.620842in}}%
\pgfpathlineto{\pgfqpoint{4.318013in}{2.651986in}}%
\pgfpathlineto{\pgfqpoint{4.332690in}{2.676823in}}%
\pgfpathlineto{\pgfqpoint{4.340148in}{2.683021in}}%
\pgfpathlineto{\pgfqpoint{4.347845in}{2.676823in}}%
\pgfpathlineto{\pgfqpoint{4.355303in}{2.676823in}}%
\pgfpathlineto{\pgfqpoint{4.362523in}{2.670620in}}%
\pgfpathlineto{\pgfqpoint{4.370220in}{2.670620in}}%
\pgfpathlineto{\pgfqpoint{4.385136in}{2.658201in}}%
\pgfpathlineto{\pgfqpoint{4.392594in}{2.658201in}}%
\pgfpathlineto{\pgfqpoint{4.400052in}{2.645766in}}%
\pgfpathlineto{\pgfqpoint{4.407510in}{2.645766in}}%
\pgfpathlineto{\pgfqpoint{4.422427in}{2.670620in}}%
\pgfpathlineto{\pgfqpoint{4.428483in}{2.683021in}}%
\pgfpathlineto{\pgfqpoint{4.452259in}{2.683021in}}%
\pgfpathlineto{\pgfqpoint{4.459718in}{2.676823in}}%
\pgfpathlineto{\pgfqpoint{4.472993in}{2.676823in}}%
\pgfpathlineto{\pgfqpoint{4.482092in}{2.670620in}}%
\pgfpathlineto{\pgfqpoint{4.519144in}{2.670620in}}%
\pgfpathlineto{\pgfqpoint{4.526841in}{2.664413in}}%
\pgfpathlineto{\pgfqpoint{4.534299in}{2.664413in}}%
\pgfpathlineto{\pgfqpoint{4.541757in}{2.670620in}}%
\pgfpathlineto{\pgfqpoint{4.549215in}{2.670620in}}%
\pgfpathlineto{\pgfqpoint{4.556674in}{2.676823in}}%
\pgfpathlineto{\pgfqpoint{4.579048in}{2.676823in}}%
\pgfpathlineto{\pgfqpoint{4.593964in}{2.664413in}}%
\pgfpathlineto{\pgfqpoint{4.599782in}{2.664413in}}%
\pgfpathlineto{\pgfqpoint{4.616339in}{2.651986in}}%
\pgfpathlineto{\pgfqpoint{4.622395in}{2.645766in}}%
\pgfpathlineto{\pgfqpoint{4.629853in}{2.645766in}}%
\pgfpathlineto{\pgfqpoint{4.646171in}{2.633313in}}%
\pgfpathlineto{\pgfqpoint{4.652466in}{2.627080in}}%
\pgfpathlineto{\pgfqpoint{4.667144in}{2.595849in}}%
\pgfpathlineto{\pgfqpoint{4.682299in}{2.583327in}}%
\pgfpathlineto{\pgfqpoint{4.690920in}{2.602104in}}%
\pgfpathlineto{\pgfqpoint{4.696976in}{2.602104in}}%
\pgfpathlineto{\pgfqpoint{4.704673in}{2.608355in}}%
\pgfpathlineto{\pgfqpoint{4.712131in}{2.608355in}}%
\pgfpathlineto{\pgfqpoint{4.720753in}{2.620842in}}%
\pgfpathlineto{\pgfqpoint{4.726809in}{2.620842in}}%
\pgfpathlineto{\pgfqpoint{4.741964in}{2.633313in}}%
\pgfpathlineto{\pgfqpoint{4.749183in}{2.633313in}}%
\pgfpathlineto{\pgfqpoint{4.756880in}{2.620842in}}%
\pgfpathlineto{\pgfqpoint{4.764338in}{2.602104in}}%
\pgfpathlineto{\pgfqpoint{4.771558in}{2.589590in}}%
\pgfpathlineto{\pgfqpoint{4.779016in}{2.570787in}}%
\pgfpathlineto{\pgfqpoint{4.786713in}{2.545654in}}%
\pgfpathlineto{\pgfqpoint{4.794171in}{2.514138in}}%
\pgfpathlineto{\pgfqpoint{4.801391in}{2.495176in}}%
\pgfpathlineto{\pgfqpoint{4.809087in}{2.450772in}}%
\pgfpathlineto{\pgfqpoint{4.816545in}{2.444411in}}%
\pgfpathlineto{\pgfqpoint{4.824004in}{2.507822in}}%
\pgfpathlineto{\pgfqpoint{4.831462in}{2.558229in}}%
\pgfpathlineto{\pgfqpoint{4.838920in}{2.570787in}}%
\pgfpathlineto{\pgfqpoint{4.846378in}{2.577059in}}%
\pgfpathlineto{\pgfqpoint{4.861294in}{2.564510in}}%
\pgfpathlineto{\pgfqpoint{4.868753in}{2.564510in}}%
\pgfpathlineto{\pgfqpoint{4.876211in}{2.558229in}}%
\pgfpathlineto{\pgfqpoint{4.883893in}{2.570787in}}%
\pgfpathlineto{\pgfqpoint{4.891127in}{2.589590in}}%
\pgfpathlineto{\pgfqpoint{4.898585in}{2.595849in}}%
\pgfpathlineto{\pgfqpoint{4.906043in}{2.595849in}}%
\pgfpathlineto{\pgfqpoint{4.920960in}{2.583327in}}%
\pgfpathlineto{\pgfqpoint{4.928418in}{2.595849in}}%
\pgfpathlineto{\pgfqpoint{4.936100in}{2.595849in}}%
\pgfpathlineto{\pgfqpoint{4.943558in}{2.614601in}}%
\pgfpathlineto{\pgfqpoint{4.958474in}{2.639541in}}%
\pgfpathlineto{\pgfqpoint{4.973390in}{2.639541in}}%
\pgfpathlineto{\pgfqpoint{4.980625in}{2.645766in}}%
\pgfpathlineto{\pgfqpoint{4.988307in}{2.639541in}}%
\pgfpathlineto{\pgfqpoint{4.995765in}{2.645766in}}%
\pgfpathlineto{\pgfqpoint{5.003223in}{2.639541in}}%
\pgfpathlineto{\pgfqpoint{5.018139in}{2.651986in}}%
\pgfpathlineto{\pgfqpoint{5.032832in}{2.651986in}}%
\pgfpathlineto{\pgfqpoint{5.040514in}{2.645766in}}%
\pgfpathlineto{\pgfqpoint{5.063127in}{2.645766in}}%
\pgfpathlineto{\pgfqpoint{5.070346in}{2.633313in}}%
\pgfpathlineto{\pgfqpoint{5.077805in}{2.627080in}}%
\pgfpathlineto{\pgfqpoint{5.085263in}{2.633313in}}%
\pgfpathlineto{\pgfqpoint{5.092721in}{2.645766in}}%
\pgfpathlineto{\pgfqpoint{5.107637in}{2.658201in}}%
\pgfpathlineto{\pgfqpoint{5.122554in}{2.658201in}}%
\pgfpathlineto{\pgfqpoint{5.152386in}{2.633313in}}%
\pgfpathlineto{\pgfqpoint{5.159844in}{2.633313in}}%
\pgfpathlineto{\pgfqpoint{5.167541in}{2.608355in}}%
\pgfpathlineto{\pgfqpoint{5.174999in}{2.602104in}}%
\pgfpathlineto{\pgfqpoint{5.182219in}{2.608355in}}%
\pgfpathlineto{\pgfqpoint{5.189916in}{2.608355in}}%
\pgfpathlineto{\pgfqpoint{5.197374in}{2.589590in}}%
\pgfpathlineto{\pgfqpoint{5.212051in}{2.526758in}}%
\pgfpathlineto{\pgfqpoint{5.219748in}{2.444411in}}%
\pgfpathlineto{\pgfqpoint{5.227206in}{2.342009in}}%
\pgfpathlineto{\pgfqpoint{5.234665in}{2.270919in}}%
\pgfpathlineto{\pgfqpoint{5.242123in}{2.257932in}}%
\pgfpathlineto{\pgfqpoint{5.249581in}{2.367719in}}%
\pgfpathlineto{\pgfqpoint{5.257039in}{2.361299in}}%
\pgfpathlineto{\pgfqpoint{5.264721in}{2.277405in}}%
\pgfpathlineto{\pgfqpoint{5.271955in}{2.166492in}}%
\pgfpathlineto{\pgfqpoint{5.286872in}{1.770433in}}%
\pgfpathlineto{\pgfqpoint{5.301788in}{1.355740in}}%
\pgfpathlineto{\pgfqpoint{5.309246in}{1.181266in}}%
\pgfpathlineto{\pgfqpoint{5.339079in}{0.620123in}}%
\pgfpathlineto{\pgfqpoint{5.339079in}{0.620123in}}%
\pgfusepath{stroke}%
\end{pgfscope}%
\begin{pgfscope}%
\pgfsetrectcap%
\pgfsetmiterjoin%
\pgfsetlinewidth{0.803000pt}%
\definecolor{currentstroke}{rgb}{0.000000,0.000000,0.000000}%
\pgfsetstrokecolor{currentstroke}%
\pgfsetdash{}{0pt}%
\pgfpathmoveto{\pgfqpoint{0.530556in}{0.515123in}}%
\pgfpathlineto{\pgfqpoint{0.530556in}{2.825123in}}%
\pgfusepath{stroke}%
\end{pgfscope}%
\begin{pgfscope}%
\pgfsetrectcap%
\pgfsetmiterjoin%
\pgfsetlinewidth{0.803000pt}%
\definecolor{currentstroke}{rgb}{0.000000,0.000000,0.000000}%
\pgfsetstrokecolor{currentstroke}%
\pgfsetdash{}{0pt}%
\pgfpathmoveto{\pgfqpoint{5.568056in}{0.515123in}}%
\pgfpathlineto{\pgfqpoint{5.568056in}{2.825123in}}%
\pgfusepath{stroke}%
\end{pgfscope}%
\begin{pgfscope}%
\pgfsetrectcap%
\pgfsetmiterjoin%
\pgfsetlinewidth{0.803000pt}%
\definecolor{currentstroke}{rgb}{0.000000,0.000000,0.000000}%
\pgfsetstrokecolor{currentstroke}%
\pgfsetdash{}{0pt}%
\pgfpathmoveto{\pgfqpoint{0.530556in}{0.515123in}}%
\pgfpathlineto{\pgfqpoint{5.568056in}{0.515123in}}%
\pgfusepath{stroke}%
\end{pgfscope}%
\begin{pgfscope}%
\pgfsetrectcap%
\pgfsetmiterjoin%
\pgfsetlinewidth{0.803000pt}%
\definecolor{currentstroke}{rgb}{0.000000,0.000000,0.000000}%
\pgfsetstrokecolor{currentstroke}%
\pgfsetdash{}{0pt}%
\pgfpathmoveto{\pgfqpoint{0.530556in}{2.825123in}}%
\pgfpathlineto{\pgfqpoint{5.568056in}{2.825123in}}%
\pgfusepath{stroke}%
\end{pgfscope}%
\begin{pgfscope}%
\definecolor{textcolor}{rgb}{0.000000,0.000000,0.000000}%
\pgfsetstrokecolor{textcolor}%
\pgfsetfillcolor{textcolor}%
\pgftext[x=3.049306in,y=2.908457in,,base]{\color{textcolor}\rmfamily\fontsize{12.000000}{14.400000}\selectfont Part I, KOH(s) \(\displaystyle \to\) KOH(aq)}%
\end{pgfscope}%
\end{pgfpicture}%
\makeatother%
\endgroup%

    \end{center}
    \caption{Temperature profile of the dissolution of solid \koh in water. }\label{fig:partI}
\end{figure}

\begin{figure}[H]
    \begin{center}
	%% Creator: Matplotlib, PGF backend
%%
%% To include the figure in your LaTeX document, write
%%   \input{<filename>.pgf}
%%
%% Make sure the required packages are loaded in your preamble
%%   \usepackage{pgf}
%%
%% Figures using additional raster images can only be included by \input if
%% they are in the same directory as the main LaTeX file. For loading figures
%% from other directories you can use the `import` package
%%   \usepackage{import}
%% and then include the figures with
%%   \import{<path to file>}{<filename>.pgf}
%%
%% Matplotlib used the following preamble
%%
\begingroup%
\makeatletter%
\begin{pgfpicture}%
\pgfpathrectangle{\pgfpointorigin}{\pgfqpoint{5.668056in}{3.133457in}}%
\pgfusepath{use as bounding box, clip}%
\begin{pgfscope}%
\pgfsetbuttcap%
\pgfsetmiterjoin%
\definecolor{currentfill}{rgb}{1.000000,1.000000,1.000000}%
\pgfsetfillcolor{currentfill}%
\pgfsetlinewidth{0.000000pt}%
\definecolor{currentstroke}{rgb}{1.000000,1.000000,1.000000}%
\pgfsetstrokecolor{currentstroke}%
\pgfsetdash{}{0pt}%
\pgfpathmoveto{\pgfqpoint{0.000000in}{0.000000in}}%
\pgfpathlineto{\pgfqpoint{5.668056in}{0.000000in}}%
\pgfpathlineto{\pgfqpoint{5.668056in}{3.133457in}}%
\pgfpathlineto{\pgfqpoint{0.000000in}{3.133457in}}%
\pgfpathclose%
\pgfusepath{fill}%
\end{pgfscope}%
\begin{pgfscope}%
\pgfsetbuttcap%
\pgfsetmiterjoin%
\definecolor{currentfill}{rgb}{1.000000,1.000000,1.000000}%
\pgfsetfillcolor{currentfill}%
\pgfsetlinewidth{0.000000pt}%
\definecolor{currentstroke}{rgb}{0.000000,0.000000,0.000000}%
\pgfsetstrokecolor{currentstroke}%
\pgfsetstrokeopacity{0.000000}%
\pgfsetdash{}{0pt}%
\pgfpathmoveto{\pgfqpoint{0.530556in}{0.515123in}}%
\pgfpathlineto{\pgfqpoint{5.568056in}{0.515123in}}%
\pgfpathlineto{\pgfqpoint{5.568056in}{2.825123in}}%
\pgfpathlineto{\pgfqpoint{0.530556in}{2.825123in}}%
\pgfpathclose%
\pgfusepath{fill}%
\end{pgfscope}%
\begin{pgfscope}%
\pgfsetbuttcap%
\pgfsetroundjoin%
\definecolor{currentfill}{rgb}{0.000000,0.000000,0.000000}%
\pgfsetfillcolor{currentfill}%
\pgfsetlinewidth{0.803000pt}%
\definecolor{currentstroke}{rgb}{0.000000,0.000000,0.000000}%
\pgfsetstrokecolor{currentstroke}%
\pgfsetdash{}{0pt}%
\pgfsys@defobject{currentmarker}{\pgfqpoint{0.000000in}{-0.048611in}}{\pgfqpoint{0.000000in}{0.000000in}}{%
\pgfpathmoveto{\pgfqpoint{0.000000in}{0.000000in}}%
\pgfpathlineto{\pgfqpoint{0.000000in}{-0.048611in}}%
\pgfusepath{stroke,fill}%
}%
\begin{pgfscope}%
\pgfsys@transformshift{0.759533in}{0.515123in}%
\pgfsys@useobject{currentmarker}{}%
\end{pgfscope}%
\end{pgfscope}%
\begin{pgfscope}%
\definecolor{textcolor}{rgb}{0.000000,0.000000,0.000000}%
\pgfsetstrokecolor{textcolor}%
\pgfsetfillcolor{textcolor}%
\pgftext[x=0.759533in,y=0.417901in,,top]{\color{textcolor}\rmfamily\fontsize{10.000000}{12.000000}\selectfont \(\displaystyle 0\)}%
\end{pgfscope}%
\begin{pgfscope}%
\pgfsetbuttcap%
\pgfsetroundjoin%
\definecolor{currentfill}{rgb}{0.000000,0.000000,0.000000}%
\pgfsetfillcolor{currentfill}%
\pgfsetlinewidth{0.803000pt}%
\definecolor{currentstroke}{rgb}{0.000000,0.000000,0.000000}%
\pgfsetstrokecolor{currentstroke}%
\pgfsetdash{}{0pt}%
\pgfsys@defobject{currentmarker}{\pgfqpoint{0.000000in}{-0.048611in}}{\pgfqpoint{0.000000in}{0.000000in}}{%
\pgfpathmoveto{\pgfqpoint{0.000000in}{0.000000in}}%
\pgfpathlineto{\pgfqpoint{0.000000in}{-0.048611in}}%
\pgfusepath{stroke,fill}%
}%
\begin{pgfscope}%
\pgfsys@transformshift{1.473831in}{0.515123in}%
\pgfsys@useobject{currentmarker}{}%
\end{pgfscope}%
\end{pgfscope}%
\begin{pgfscope}%
\definecolor{textcolor}{rgb}{0.000000,0.000000,0.000000}%
\pgfsetstrokecolor{textcolor}%
\pgfsetfillcolor{textcolor}%
\pgftext[x=1.473831in,y=0.417901in,,top]{\color{textcolor}\rmfamily\fontsize{10.000000}{12.000000}\selectfont \(\displaystyle 50\)}%
\end{pgfscope}%
\begin{pgfscope}%
\pgfsetbuttcap%
\pgfsetroundjoin%
\definecolor{currentfill}{rgb}{0.000000,0.000000,0.000000}%
\pgfsetfillcolor{currentfill}%
\pgfsetlinewidth{0.803000pt}%
\definecolor{currentstroke}{rgb}{0.000000,0.000000,0.000000}%
\pgfsetstrokecolor{currentstroke}%
\pgfsetdash{}{0pt}%
\pgfsys@defobject{currentmarker}{\pgfqpoint{0.000000in}{-0.048611in}}{\pgfqpoint{0.000000in}{0.000000in}}{%
\pgfpathmoveto{\pgfqpoint{0.000000in}{0.000000in}}%
\pgfpathlineto{\pgfqpoint{0.000000in}{-0.048611in}}%
\pgfusepath{stroke,fill}%
}%
\begin{pgfscope}%
\pgfsys@transformshift{2.188128in}{0.515123in}%
\pgfsys@useobject{currentmarker}{}%
\end{pgfscope}%
\end{pgfscope}%
\begin{pgfscope}%
\definecolor{textcolor}{rgb}{0.000000,0.000000,0.000000}%
\pgfsetstrokecolor{textcolor}%
\pgfsetfillcolor{textcolor}%
\pgftext[x=2.188128in,y=0.417901in,,top]{\color{textcolor}\rmfamily\fontsize{10.000000}{12.000000}\selectfont \(\displaystyle 100\)}%
\end{pgfscope}%
\begin{pgfscope}%
\pgfsetbuttcap%
\pgfsetroundjoin%
\definecolor{currentfill}{rgb}{0.000000,0.000000,0.000000}%
\pgfsetfillcolor{currentfill}%
\pgfsetlinewidth{0.803000pt}%
\definecolor{currentstroke}{rgb}{0.000000,0.000000,0.000000}%
\pgfsetstrokecolor{currentstroke}%
\pgfsetdash{}{0pt}%
\pgfsys@defobject{currentmarker}{\pgfqpoint{0.000000in}{-0.048611in}}{\pgfqpoint{0.000000in}{0.000000in}}{%
\pgfpathmoveto{\pgfqpoint{0.000000in}{0.000000in}}%
\pgfpathlineto{\pgfqpoint{0.000000in}{-0.048611in}}%
\pgfusepath{stroke,fill}%
}%
\begin{pgfscope}%
\pgfsys@transformshift{2.902425in}{0.515123in}%
\pgfsys@useobject{currentmarker}{}%
\end{pgfscope}%
\end{pgfscope}%
\begin{pgfscope}%
\definecolor{textcolor}{rgb}{0.000000,0.000000,0.000000}%
\pgfsetstrokecolor{textcolor}%
\pgfsetfillcolor{textcolor}%
\pgftext[x=2.902425in,y=0.417901in,,top]{\color{textcolor}\rmfamily\fontsize{10.000000}{12.000000}\selectfont \(\displaystyle 150\)}%
\end{pgfscope}%
\begin{pgfscope}%
\pgfsetbuttcap%
\pgfsetroundjoin%
\definecolor{currentfill}{rgb}{0.000000,0.000000,0.000000}%
\pgfsetfillcolor{currentfill}%
\pgfsetlinewidth{0.803000pt}%
\definecolor{currentstroke}{rgb}{0.000000,0.000000,0.000000}%
\pgfsetstrokecolor{currentstroke}%
\pgfsetdash{}{0pt}%
\pgfsys@defobject{currentmarker}{\pgfqpoint{0.000000in}{-0.048611in}}{\pgfqpoint{0.000000in}{0.000000in}}{%
\pgfpathmoveto{\pgfqpoint{0.000000in}{0.000000in}}%
\pgfpathlineto{\pgfqpoint{0.000000in}{-0.048611in}}%
\pgfusepath{stroke,fill}%
}%
\begin{pgfscope}%
\pgfsys@transformshift{3.616722in}{0.515123in}%
\pgfsys@useobject{currentmarker}{}%
\end{pgfscope}%
\end{pgfscope}%
\begin{pgfscope}%
\definecolor{textcolor}{rgb}{0.000000,0.000000,0.000000}%
\pgfsetstrokecolor{textcolor}%
\pgfsetfillcolor{textcolor}%
\pgftext[x=3.616722in,y=0.417901in,,top]{\color{textcolor}\rmfamily\fontsize{10.000000}{12.000000}\selectfont \(\displaystyle 200\)}%
\end{pgfscope}%
\begin{pgfscope}%
\pgfsetbuttcap%
\pgfsetroundjoin%
\definecolor{currentfill}{rgb}{0.000000,0.000000,0.000000}%
\pgfsetfillcolor{currentfill}%
\pgfsetlinewidth{0.803000pt}%
\definecolor{currentstroke}{rgb}{0.000000,0.000000,0.000000}%
\pgfsetstrokecolor{currentstroke}%
\pgfsetdash{}{0pt}%
\pgfsys@defobject{currentmarker}{\pgfqpoint{0.000000in}{-0.048611in}}{\pgfqpoint{0.000000in}{0.000000in}}{%
\pgfpathmoveto{\pgfqpoint{0.000000in}{0.000000in}}%
\pgfpathlineto{\pgfqpoint{0.000000in}{-0.048611in}}%
\pgfusepath{stroke,fill}%
}%
\begin{pgfscope}%
\pgfsys@transformshift{4.331020in}{0.515123in}%
\pgfsys@useobject{currentmarker}{}%
\end{pgfscope}%
\end{pgfscope}%
\begin{pgfscope}%
\definecolor{textcolor}{rgb}{0.000000,0.000000,0.000000}%
\pgfsetstrokecolor{textcolor}%
\pgfsetfillcolor{textcolor}%
\pgftext[x=4.331020in,y=0.417901in,,top]{\color{textcolor}\rmfamily\fontsize{10.000000}{12.000000}\selectfont \(\displaystyle 250\)}%
\end{pgfscope}%
\begin{pgfscope}%
\pgfsetbuttcap%
\pgfsetroundjoin%
\definecolor{currentfill}{rgb}{0.000000,0.000000,0.000000}%
\pgfsetfillcolor{currentfill}%
\pgfsetlinewidth{0.803000pt}%
\definecolor{currentstroke}{rgb}{0.000000,0.000000,0.000000}%
\pgfsetstrokecolor{currentstroke}%
\pgfsetdash{}{0pt}%
\pgfsys@defobject{currentmarker}{\pgfqpoint{0.000000in}{-0.048611in}}{\pgfqpoint{0.000000in}{0.000000in}}{%
\pgfpathmoveto{\pgfqpoint{0.000000in}{0.000000in}}%
\pgfpathlineto{\pgfqpoint{0.000000in}{-0.048611in}}%
\pgfusepath{stroke,fill}%
}%
\begin{pgfscope}%
\pgfsys@transformshift{5.045317in}{0.515123in}%
\pgfsys@useobject{currentmarker}{}%
\end{pgfscope}%
\end{pgfscope}%
\begin{pgfscope}%
\definecolor{textcolor}{rgb}{0.000000,0.000000,0.000000}%
\pgfsetstrokecolor{textcolor}%
\pgfsetfillcolor{textcolor}%
\pgftext[x=5.045317in,y=0.417901in,,top]{\color{textcolor}\rmfamily\fontsize{10.000000}{12.000000}\selectfont \(\displaystyle 300\)}%
\end{pgfscope}%
\begin{pgfscope}%
\definecolor{textcolor}{rgb}{0.000000,0.000000,0.000000}%
\pgfsetstrokecolor{textcolor}%
\pgfsetfillcolor{textcolor}%
\pgftext[x=3.049306in,y=0.238889in,,top]{\color{textcolor}\rmfamily\fontsize{10.000000}{12.000000}\selectfont Time (seconds)}%
\end{pgfscope}%
\begin{pgfscope}%
\pgfsetbuttcap%
\pgfsetroundjoin%
\definecolor{currentfill}{rgb}{0.000000,0.000000,0.000000}%
\pgfsetfillcolor{currentfill}%
\pgfsetlinewidth{0.803000pt}%
\definecolor{currentstroke}{rgb}{0.000000,0.000000,0.000000}%
\pgfsetstrokecolor{currentstroke}%
\pgfsetdash{}{0pt}%
\pgfsys@defobject{currentmarker}{\pgfqpoint{-0.048611in}{0.000000in}}{\pgfqpoint{0.000000in}{0.000000in}}{%
\pgfpathmoveto{\pgfqpoint{0.000000in}{0.000000in}}%
\pgfpathlineto{\pgfqpoint{-0.048611in}{0.000000in}}%
\pgfusepath{stroke,fill}%
}%
\begin{pgfscope}%
\pgfsys@transformshift{0.530556in}{0.691728in}%
\pgfsys@useobject{currentmarker}{}%
\end{pgfscope}%
\end{pgfscope}%
\begin{pgfscope}%
\definecolor{textcolor}{rgb}{0.000000,0.000000,0.000000}%
\pgfsetstrokecolor{textcolor}%
\pgfsetfillcolor{textcolor}%
\pgftext[x=0.294444in,y=0.643503in,left,base]{\color{textcolor}\rmfamily\fontsize{10.000000}{12.000000}\selectfont \(\displaystyle 24\)}%
\end{pgfscope}%
\begin{pgfscope}%
\pgfsetbuttcap%
\pgfsetroundjoin%
\definecolor{currentfill}{rgb}{0.000000,0.000000,0.000000}%
\pgfsetfillcolor{currentfill}%
\pgfsetlinewidth{0.803000pt}%
\definecolor{currentstroke}{rgb}{0.000000,0.000000,0.000000}%
\pgfsetstrokecolor{currentstroke}%
\pgfsetdash{}{0pt}%
\pgfsys@defobject{currentmarker}{\pgfqpoint{-0.048611in}{0.000000in}}{\pgfqpoint{0.000000in}{0.000000in}}{%
\pgfpathmoveto{\pgfqpoint{0.000000in}{0.000000in}}%
\pgfpathlineto{\pgfqpoint{-0.048611in}{0.000000in}}%
\pgfusepath{stroke,fill}%
}%
\begin{pgfscope}%
\pgfsys@transformshift{0.530556in}{1.087899in}%
\pgfsys@useobject{currentmarker}{}%
\end{pgfscope}%
\end{pgfscope}%
\begin{pgfscope}%
\definecolor{textcolor}{rgb}{0.000000,0.000000,0.000000}%
\pgfsetstrokecolor{textcolor}%
\pgfsetfillcolor{textcolor}%
\pgftext[x=0.294444in,y=1.039674in,left,base]{\color{textcolor}\rmfamily\fontsize{10.000000}{12.000000}\selectfont \(\displaystyle 26\)}%
\end{pgfscope}%
\begin{pgfscope}%
\pgfsetbuttcap%
\pgfsetroundjoin%
\definecolor{currentfill}{rgb}{0.000000,0.000000,0.000000}%
\pgfsetfillcolor{currentfill}%
\pgfsetlinewidth{0.803000pt}%
\definecolor{currentstroke}{rgb}{0.000000,0.000000,0.000000}%
\pgfsetstrokecolor{currentstroke}%
\pgfsetdash{}{0pt}%
\pgfsys@defobject{currentmarker}{\pgfqpoint{-0.048611in}{0.000000in}}{\pgfqpoint{0.000000in}{0.000000in}}{%
\pgfpathmoveto{\pgfqpoint{0.000000in}{0.000000in}}%
\pgfpathlineto{\pgfqpoint{-0.048611in}{0.000000in}}%
\pgfusepath{stroke,fill}%
}%
\begin{pgfscope}%
\pgfsys@transformshift{0.530556in}{1.484069in}%
\pgfsys@useobject{currentmarker}{}%
\end{pgfscope}%
\end{pgfscope}%
\begin{pgfscope}%
\definecolor{textcolor}{rgb}{0.000000,0.000000,0.000000}%
\pgfsetstrokecolor{textcolor}%
\pgfsetfillcolor{textcolor}%
\pgftext[x=0.294444in,y=1.435844in,left,base]{\color{textcolor}\rmfamily\fontsize{10.000000}{12.000000}\selectfont \(\displaystyle 28\)}%
\end{pgfscope}%
\begin{pgfscope}%
\pgfsetbuttcap%
\pgfsetroundjoin%
\definecolor{currentfill}{rgb}{0.000000,0.000000,0.000000}%
\pgfsetfillcolor{currentfill}%
\pgfsetlinewidth{0.803000pt}%
\definecolor{currentstroke}{rgb}{0.000000,0.000000,0.000000}%
\pgfsetstrokecolor{currentstroke}%
\pgfsetdash{}{0pt}%
\pgfsys@defobject{currentmarker}{\pgfqpoint{-0.048611in}{0.000000in}}{\pgfqpoint{0.000000in}{0.000000in}}{%
\pgfpathmoveto{\pgfqpoint{0.000000in}{0.000000in}}%
\pgfpathlineto{\pgfqpoint{-0.048611in}{0.000000in}}%
\pgfusepath{stroke,fill}%
}%
\begin{pgfscope}%
\pgfsys@transformshift{0.530556in}{1.880240in}%
\pgfsys@useobject{currentmarker}{}%
\end{pgfscope}%
\end{pgfscope}%
\begin{pgfscope}%
\definecolor{textcolor}{rgb}{0.000000,0.000000,0.000000}%
\pgfsetstrokecolor{textcolor}%
\pgfsetfillcolor{textcolor}%
\pgftext[x=0.294444in,y=1.832015in,left,base]{\color{textcolor}\rmfamily\fontsize{10.000000}{12.000000}\selectfont \(\displaystyle 30\)}%
\end{pgfscope}%
\begin{pgfscope}%
\pgfsetbuttcap%
\pgfsetroundjoin%
\definecolor{currentfill}{rgb}{0.000000,0.000000,0.000000}%
\pgfsetfillcolor{currentfill}%
\pgfsetlinewidth{0.803000pt}%
\definecolor{currentstroke}{rgb}{0.000000,0.000000,0.000000}%
\pgfsetstrokecolor{currentstroke}%
\pgfsetdash{}{0pt}%
\pgfsys@defobject{currentmarker}{\pgfqpoint{-0.048611in}{0.000000in}}{\pgfqpoint{0.000000in}{0.000000in}}{%
\pgfpathmoveto{\pgfqpoint{0.000000in}{0.000000in}}%
\pgfpathlineto{\pgfqpoint{-0.048611in}{0.000000in}}%
\pgfusepath{stroke,fill}%
}%
\begin{pgfscope}%
\pgfsys@transformshift{0.530556in}{2.276411in}%
\pgfsys@useobject{currentmarker}{}%
\end{pgfscope}%
\end{pgfscope}%
\begin{pgfscope}%
\definecolor{textcolor}{rgb}{0.000000,0.000000,0.000000}%
\pgfsetstrokecolor{textcolor}%
\pgfsetfillcolor{textcolor}%
\pgftext[x=0.294444in,y=2.228185in,left,base]{\color{textcolor}\rmfamily\fontsize{10.000000}{12.000000}\selectfont \(\displaystyle 32\)}%
\end{pgfscope}%
\begin{pgfscope}%
\pgfsetbuttcap%
\pgfsetroundjoin%
\definecolor{currentfill}{rgb}{0.000000,0.000000,0.000000}%
\pgfsetfillcolor{currentfill}%
\pgfsetlinewidth{0.803000pt}%
\definecolor{currentstroke}{rgb}{0.000000,0.000000,0.000000}%
\pgfsetstrokecolor{currentstroke}%
\pgfsetdash{}{0pt}%
\pgfsys@defobject{currentmarker}{\pgfqpoint{-0.048611in}{0.000000in}}{\pgfqpoint{0.000000in}{0.000000in}}{%
\pgfpathmoveto{\pgfqpoint{0.000000in}{0.000000in}}%
\pgfpathlineto{\pgfqpoint{-0.048611in}{0.000000in}}%
\pgfusepath{stroke,fill}%
}%
\begin{pgfscope}%
\pgfsys@transformshift{0.530556in}{2.672581in}%
\pgfsys@useobject{currentmarker}{}%
\end{pgfscope}%
\end{pgfscope}%
\begin{pgfscope}%
\definecolor{textcolor}{rgb}{0.000000,0.000000,0.000000}%
\pgfsetstrokecolor{textcolor}%
\pgfsetfillcolor{textcolor}%
\pgftext[x=0.294444in,y=2.624356in,left,base]{\color{textcolor}\rmfamily\fontsize{10.000000}{12.000000}\selectfont \(\displaystyle 34\)}%
\end{pgfscope}%
\begin{pgfscope}%
\definecolor{textcolor}{rgb}{0.000000,0.000000,0.000000}%
\pgfsetstrokecolor{textcolor}%
\pgfsetfillcolor{textcolor}%
\pgftext[x=0.238889in,y=1.670123in,,bottom,rotate=90.000000]{\color{textcolor}\rmfamily\fontsize{10.000000}{12.000000}\selectfont Temperature (C)}%
\end{pgfscope}%
\begin{pgfscope}%
\pgfpathrectangle{\pgfqpoint{0.530556in}{0.515123in}}{\pgfqpoint{5.037500in}{2.310000in}}%
\pgfusepath{clip}%
\pgfsetrectcap%
\pgfsetroundjoin%
\pgfsetlinewidth{1.505625pt}%
\definecolor{currentstroke}{rgb}{1.000000,0.000000,0.000000}%
\pgfsetstrokecolor{currentstroke}%
\pgfsetdash{}{0pt}%
\pgfpathmoveto{\pgfqpoint{0.759533in}{0.743507in}}%
\pgfpathlineto{\pgfqpoint{0.774048in}{0.755292in}}%
\pgfpathlineto{\pgfqpoint{0.782534in}{0.767049in}}%
\pgfpathlineto{\pgfqpoint{0.789448in}{0.778780in}}%
\pgfpathlineto{\pgfqpoint{0.796820in}{0.782684in}}%
\pgfpathlineto{\pgfqpoint{0.803963in}{0.782684in}}%
\pgfpathlineto{\pgfqpoint{0.811106in}{0.786586in}}%
\pgfpathlineto{\pgfqpoint{0.818248in}{0.786586in}}%
\pgfpathlineto{\pgfqpoint{0.825391in}{0.782684in}}%
\pgfpathlineto{\pgfqpoint{0.846820in}{0.782684in}}%
\pgfpathlineto{\pgfqpoint{0.853963in}{0.774873in}}%
\pgfpathlineto{\pgfqpoint{0.875392in}{0.774873in}}%
\pgfpathlineto{\pgfqpoint{0.882535in}{0.778780in}}%
\pgfpathlineto{\pgfqpoint{0.888335in}{0.778780in}}%
\pgfpathlineto{\pgfqpoint{0.895478in}{0.782684in}}%
\pgfpathlineto{\pgfqpoint{0.924050in}{0.782684in}}%
\pgfpathlineto{\pgfqpoint{0.932536in}{0.786586in}}%
\pgfpathlineto{\pgfqpoint{0.945693in}{0.786586in}}%
\pgfpathlineto{\pgfqpoint{0.953965in}{0.782684in}}%
\pgfpathlineto{\pgfqpoint{0.961108in}{0.786586in}}%
\pgfpathlineto{\pgfqpoint{0.967122in}{0.786586in}}%
\pgfpathlineto{\pgfqpoint{0.974265in}{0.782684in}}%
\pgfpathlineto{\pgfqpoint{1.067124in}{0.782684in}}%
\pgfpathlineto{\pgfqpoint{1.074267in}{0.778780in}}%
\pgfpathlineto{\pgfqpoint{1.088553in}{0.778780in}}%
\pgfpathlineto{\pgfqpoint{1.095696in}{0.770963in}}%
\pgfpathlineto{\pgfqpoint{1.102839in}{0.767049in}}%
\pgfpathlineto{\pgfqpoint{1.109982in}{0.755292in}}%
\pgfpathlineto{\pgfqpoint{1.124268in}{0.755292in}}%
\pgfpathlineto{\pgfqpoint{1.131411in}{0.763133in}}%
\pgfpathlineto{\pgfqpoint{1.145697in}{0.770963in}}%
\pgfpathlineto{\pgfqpoint{1.167354in}{0.770963in}}%
\pgfpathlineto{\pgfqpoint{1.174269in}{0.774873in}}%
\pgfpathlineto{\pgfqpoint{1.181411in}{0.774873in}}%
\pgfpathlineto{\pgfqpoint{1.188783in}{0.770963in}}%
\pgfpathlineto{\pgfqpoint{1.202840in}{0.770963in}}%
\pgfpathlineto{\pgfqpoint{1.217355in}{0.763133in}}%
\pgfpathlineto{\pgfqpoint{1.224498in}{0.763133in}}%
\pgfpathlineto{\pgfqpoint{1.245927in}{0.751367in}}%
\pgfpathlineto{\pgfqpoint{1.253070in}{0.751367in}}%
\pgfpathlineto{\pgfqpoint{1.267356in}{0.759214in}}%
\pgfpathlineto{\pgfqpoint{1.288785in}{0.759214in}}%
\pgfpathlineto{\pgfqpoint{1.295928in}{0.751367in}}%
\pgfpathlineto{\pgfqpoint{1.310442in}{0.751367in}}%
\pgfpathlineto{\pgfqpoint{1.317357in}{0.747439in}}%
\pgfpathlineto{\pgfqpoint{1.324500in}{0.735636in}}%
\pgfpathlineto{\pgfqpoint{1.331642in}{0.727753in}}%
\pgfpathlineto{\pgfqpoint{1.338785in}{0.715905in}}%
\pgfpathlineto{\pgfqpoint{1.345928in}{0.707991in}}%
\pgfpathlineto{\pgfqpoint{1.360443in}{0.715905in}}%
\pgfpathlineto{\pgfqpoint{1.367357in}{0.723806in}}%
\pgfpathlineto{\pgfqpoint{1.374500in}{0.715905in}}%
\pgfpathlineto{\pgfqpoint{1.381872in}{0.715905in}}%
\pgfpathlineto{\pgfqpoint{1.389015in}{0.719857in}}%
\pgfpathlineto{\pgfqpoint{1.395929in}{0.727753in}}%
\pgfpathlineto{\pgfqpoint{1.403072in}{0.727753in}}%
\pgfpathlineto{\pgfqpoint{1.453073in}{0.755292in}}%
\pgfpathlineto{\pgfqpoint{1.474731in}{0.755292in}}%
\pgfpathlineto{\pgfqpoint{1.481873in}{0.759214in}}%
\pgfpathlineto{\pgfqpoint{1.517588in}{0.759214in}}%
\pgfpathlineto{\pgfqpoint{1.524731in}{0.763133in}}%
\pgfpathlineto{\pgfqpoint{1.532089in}{0.763133in}}%
\pgfpathlineto{\pgfqpoint{1.539017in}{0.767049in}}%
\pgfpathlineto{\pgfqpoint{1.546160in}{0.763133in}}%
\pgfpathlineto{\pgfqpoint{1.553518in}{0.751367in}}%
\pgfpathlineto{\pgfqpoint{1.560446in}{0.743507in}}%
\pgfpathlineto{\pgfqpoint{1.567589in}{0.739573in}}%
\pgfpathlineto{\pgfqpoint{1.582089in}{0.723806in}}%
\pgfpathlineto{\pgfqpoint{1.589018in}{0.727753in}}%
\pgfpathlineto{\pgfqpoint{1.596161in}{0.723806in}}%
\pgfpathlineto{\pgfqpoint{1.603518in}{0.688153in}}%
\pgfpathlineto{\pgfqpoint{1.610661in}{0.688153in}}%
\pgfpathlineto{\pgfqpoint{1.617590in}{0.700065in}}%
\pgfpathlineto{\pgfqpoint{1.646162in}{0.715905in}}%
\pgfpathlineto{\pgfqpoint{1.653519in}{0.715905in}}%
\pgfpathlineto{\pgfqpoint{1.667805in}{0.692127in}}%
\pgfpathlineto{\pgfqpoint{1.674948in}{0.684176in}}%
\pgfpathlineto{\pgfqpoint{1.689234in}{0.676213in}}%
\pgfpathlineto{\pgfqpoint{1.696377in}{0.660250in}}%
\pgfpathlineto{\pgfqpoint{1.703520in}{0.652250in}}%
\pgfpathlineto{\pgfqpoint{1.717806in}{0.620123in}}%
\pgfpathlineto{\pgfqpoint{1.725177in}{0.648245in}}%
\pgfpathlineto{\pgfqpoint{1.739235in}{0.715905in}}%
\pgfpathlineto{\pgfqpoint{1.746378in}{0.735636in}}%
\pgfpathlineto{\pgfqpoint{1.753749in}{0.743507in}}%
\pgfpathlineto{\pgfqpoint{1.767807in}{0.751367in}}%
\pgfpathlineto{\pgfqpoint{1.775178in}{0.751367in}}%
\pgfpathlineto{\pgfqpoint{1.782321in}{0.755292in}}%
\pgfpathlineto{\pgfqpoint{1.789236in}{0.755292in}}%
\pgfpathlineto{\pgfqpoint{1.796379in}{0.759214in}}%
\pgfpathlineto{\pgfqpoint{1.832322in}{0.759214in}}%
\pgfpathlineto{\pgfqpoint{1.839465in}{0.755292in}}%
\pgfpathlineto{\pgfqpoint{1.853751in}{0.739573in}}%
\pgfpathlineto{\pgfqpoint{1.882323in}{0.723806in}}%
\pgfpathlineto{\pgfqpoint{1.903752in}{0.723806in}}%
\pgfpathlineto{\pgfqpoint{1.939467in}{0.704030in}}%
\pgfpathlineto{\pgfqpoint{1.946838in}{0.711949in}}%
\pgfpathlineto{\pgfqpoint{1.960896in}{0.719857in}}%
\pgfpathlineto{\pgfqpoint{1.982324in}{0.719857in}}%
\pgfpathlineto{\pgfqpoint{1.989467in}{0.723806in}}%
\pgfpathlineto{\pgfqpoint{1.996839in}{0.719857in}}%
\pgfpathlineto{\pgfqpoint{2.010896in}{0.727753in}}%
\pgfpathlineto{\pgfqpoint{2.046840in}{0.727753in}}%
\pgfpathlineto{\pgfqpoint{2.053983in}{0.723806in}}%
\pgfpathlineto{\pgfqpoint{2.061126in}{0.723806in}}%
\pgfpathlineto{\pgfqpoint{2.068269in}{0.727753in}}%
\pgfpathlineto{\pgfqpoint{2.075412in}{0.739573in}}%
\pgfpathlineto{\pgfqpoint{2.096841in}{0.751367in}}%
\pgfpathlineto{\pgfqpoint{2.125412in}{0.751367in}}%
\pgfpathlineto{\pgfqpoint{2.132555in}{0.747439in}}%
\pgfpathlineto{\pgfqpoint{2.138356in}{0.747439in}}%
\pgfpathlineto{\pgfqpoint{2.153984in}{0.739573in}}%
\pgfpathlineto{\pgfqpoint{2.161127in}{0.747439in}}%
\pgfpathlineto{\pgfqpoint{2.166927in}{0.751367in}}%
\pgfpathlineto{\pgfqpoint{2.175413in}{0.755292in}}%
\pgfpathlineto{\pgfqpoint{2.182556in}{0.755292in}}%
\pgfpathlineto{\pgfqpoint{2.189914in}{0.759214in}}%
\pgfpathlineto{\pgfqpoint{2.195499in}{0.759214in}}%
\pgfpathlineto{\pgfqpoint{2.211128in}{0.751367in}}%
\pgfpathlineto{\pgfqpoint{2.216928in}{0.751367in}}%
\pgfpathlineto{\pgfqpoint{2.224071in}{0.755292in}}%
\pgfpathlineto{\pgfqpoint{2.232557in}{0.755292in}}%
\pgfpathlineto{\pgfqpoint{2.238357in}{0.759214in}}%
\pgfpathlineto{\pgfqpoint{2.245714in}{0.759214in}}%
\pgfpathlineto{\pgfqpoint{2.252643in}{0.763133in}}%
\pgfpathlineto{\pgfqpoint{2.268486in}{0.763133in}}%
\pgfpathlineto{\pgfqpoint{2.274286in}{0.767049in}}%
\pgfpathlineto{\pgfqpoint{2.281215in}{0.767049in}}%
\pgfpathlineto{\pgfqpoint{2.288358in}{0.763133in}}%
\pgfpathlineto{\pgfqpoint{2.331430in}{0.763133in}}%
\pgfpathlineto{\pgfqpoint{2.338359in}{0.759214in}}%
\pgfpathlineto{\pgfqpoint{2.345716in}{0.763133in}}%
\pgfpathlineto{\pgfqpoint{2.360002in}{0.763133in}}%
\pgfpathlineto{\pgfqpoint{2.367145in}{0.759214in}}%
\pgfpathlineto{\pgfqpoint{2.381431in}{0.759214in}}%
\pgfpathlineto{\pgfqpoint{2.388802in}{0.755292in}}%
\pgfpathlineto{\pgfqpoint{2.395717in}{0.755292in}}%
\pgfpathlineto{\pgfqpoint{2.402860in}{0.751367in}}%
\pgfpathlineto{\pgfqpoint{2.438803in}{0.751367in}}%
\pgfpathlineto{\pgfqpoint{2.460004in}{0.739573in}}%
\pgfpathlineto{\pgfqpoint{2.467375in}{0.731696in}}%
\pgfpathlineto{\pgfqpoint{2.481433in}{0.723806in}}%
\pgfpathlineto{\pgfqpoint{2.510004in}{0.723806in}}%
\pgfpathlineto{\pgfqpoint{2.517376in}{0.715905in}}%
\pgfpathlineto{\pgfqpoint{2.524519in}{0.711949in}}%
\pgfpathlineto{\pgfqpoint{2.531433in}{0.711949in}}%
\pgfpathlineto{\pgfqpoint{2.545948in}{0.727753in}}%
\pgfpathlineto{\pgfqpoint{2.553091in}{0.727753in}}%
\pgfpathlineto{\pgfqpoint{2.560234in}{0.719857in}}%
\pgfpathlineto{\pgfqpoint{2.574520in}{0.727753in}}%
\pgfpathlineto{\pgfqpoint{2.581663in}{0.727753in}}%
\pgfpathlineto{\pgfqpoint{2.588806in}{0.719857in}}%
\pgfpathlineto{\pgfqpoint{2.595949in}{0.719857in}}%
\pgfpathlineto{\pgfqpoint{2.610463in}{0.727753in}}%
\pgfpathlineto{\pgfqpoint{2.617378in}{0.723806in}}%
\pgfpathlineto{\pgfqpoint{2.624521in}{0.723806in}}%
\pgfpathlineto{\pgfqpoint{2.631664in}{0.751367in}}%
\pgfpathlineto{\pgfqpoint{2.639035in}{0.763133in}}%
\pgfpathlineto{\pgfqpoint{2.645949in}{0.770963in}}%
\pgfpathlineto{\pgfqpoint{2.653092in}{0.774873in}}%
\pgfpathlineto{\pgfqpoint{2.667607in}{0.798272in}}%
\pgfpathlineto{\pgfqpoint{2.674521in}{0.798272in}}%
\pgfpathlineto{\pgfqpoint{2.681893in}{0.802162in}}%
\pgfpathlineto{\pgfqpoint{2.689036in}{0.809932in}}%
\pgfpathlineto{\pgfqpoint{2.696179in}{0.829306in}}%
\pgfpathlineto{\pgfqpoint{2.703093in}{0.894641in}}%
\pgfpathlineto{\pgfqpoint{2.710465in}{0.932690in}}%
\pgfpathlineto{\pgfqpoint{2.717608in}{0.906085in}}%
\pgfpathlineto{\pgfqpoint{2.724522in}{0.867839in}}%
\pgfpathlineto{\pgfqpoint{2.731894in}{0.867839in}}%
\pgfpathlineto{\pgfqpoint{2.739037in}{0.902273in}}%
\pgfpathlineto{\pgfqpoint{2.746180in}{0.985494in}}%
\pgfpathlineto{\pgfqpoint{2.753094in}{1.045190in}}%
\pgfpathlineto{\pgfqpoint{2.760466in}{1.126167in}}%
\pgfpathlineto{\pgfqpoint{2.767609in}{1.155304in}}%
\pgfpathlineto{\pgfqpoint{2.774752in}{1.277352in}}%
\pgfpathlineto{\pgfqpoint{2.781895in}{1.379251in}}%
\pgfpathlineto{\pgfqpoint{2.789037in}{1.407016in}}%
\pgfpathlineto{\pgfqpoint{2.796180in}{1.486036in}}%
\pgfpathlineto{\pgfqpoint{2.803323in}{1.590702in}}%
\pgfpathlineto{\pgfqpoint{2.824752in}{1.803678in}}%
\pgfpathlineto{\pgfqpoint{2.832110in}{1.845270in}}%
\pgfpathlineto{\pgfqpoint{2.839038in}{1.858003in}}%
\pgfpathlineto{\pgfqpoint{2.846181in}{1.886541in}}%
\pgfpathlineto{\pgfqpoint{2.853539in}{1.968140in}}%
\pgfpathlineto{\pgfqpoint{2.860681in}{2.033153in}}%
\pgfpathlineto{\pgfqpoint{2.882110in}{2.169878in}}%
\pgfpathlineto{\pgfqpoint{2.889253in}{2.187849in}}%
\pgfpathlineto{\pgfqpoint{2.896182in}{2.223613in}}%
\pgfpathlineto{\pgfqpoint{2.903539in}{2.238445in}}%
\pgfpathlineto{\pgfqpoint{2.910682in}{2.273878in}}%
\pgfpathlineto{\pgfqpoint{2.917825in}{2.291508in}}%
\pgfpathlineto{\pgfqpoint{2.924754in}{2.300301in}}%
\pgfpathlineto{\pgfqpoint{2.932111in}{2.306156in}}%
\pgfpathlineto{\pgfqpoint{2.939254in}{2.306156in}}%
\pgfpathlineto{\pgfqpoint{2.946183in}{2.329511in}}%
\pgfpathlineto{\pgfqpoint{2.953540in}{2.309081in}}%
\pgfpathlineto{\pgfqpoint{2.960683in}{2.279761in}}%
\pgfpathlineto{\pgfqpoint{2.967826in}{2.270934in}}%
\pgfpathlineto{\pgfqpoint{2.975198in}{2.279761in}}%
\pgfpathlineto{\pgfqpoint{2.982112in}{2.294440in}}%
\pgfpathlineto{\pgfqpoint{2.989255in}{2.294440in}}%
\pgfpathlineto{\pgfqpoint{3.003541in}{2.346961in}}%
\pgfpathlineto{\pgfqpoint{3.010684in}{2.355666in}}%
\pgfpathlineto{\pgfqpoint{3.017827in}{2.358564in}}%
\pgfpathlineto{\pgfqpoint{3.025198in}{2.373033in}}%
\pgfpathlineto{\pgfqpoint{3.032113in}{2.378809in}}%
\pgfpathlineto{\pgfqpoint{3.039256in}{2.381695in}}%
\pgfpathlineto{\pgfqpoint{3.053770in}{2.375922in}}%
\pgfpathlineto{\pgfqpoint{3.060685in}{2.370142in}}%
\pgfpathlineto{\pgfqpoint{3.067828in}{2.367250in}}%
\pgfpathlineto{\pgfqpoint{3.075199in}{2.373033in}}%
\pgfpathlineto{\pgfqpoint{3.089257in}{2.367250in}}%
\pgfpathlineto{\pgfqpoint{3.096400in}{2.358564in}}%
\pgfpathlineto{\pgfqpoint{3.103771in}{2.364356in}}%
\pgfpathlineto{\pgfqpoint{3.110914in}{2.375922in}}%
\pgfpathlineto{\pgfqpoint{3.117829in}{2.370142in}}%
\pgfpathlineto{\pgfqpoint{3.125200in}{2.358564in}}%
\pgfpathlineto{\pgfqpoint{3.132343in}{2.326597in}}%
\pgfpathlineto{\pgfqpoint{3.139486in}{2.285637in}}%
\pgfpathlineto{\pgfqpoint{3.146629in}{2.253237in}}%
\pgfpathlineto{\pgfqpoint{3.153772in}{2.253237in}}%
\pgfpathlineto{\pgfqpoint{3.160915in}{2.247325in}}%
\pgfpathlineto{\pgfqpoint{3.168058in}{2.235482in}}%
\pgfpathlineto{\pgfqpoint{3.175201in}{2.199796in}}%
\pgfpathlineto{\pgfqpoint{3.182344in}{2.124687in}}%
\pgfpathlineto{\pgfqpoint{3.196858in}{1.829312in}}%
\pgfpathlineto{\pgfqpoint{3.203773in}{1.712979in}}%
\pgfpathlineto{\pgfqpoint{3.218059in}{1.540316in}}%
\pgfpathlineto{\pgfqpoint{3.225202in}{1.587358in}}%
\pgfpathlineto{\pgfqpoint{3.246859in}{2.187849in}}%
\pgfpathlineto{\pgfqpoint{3.253774in}{2.288573in}}%
\pgfpathlineto{\pgfqpoint{3.260917in}{2.346961in}}%
\pgfpathlineto{\pgfqpoint{3.282345in}{2.467574in}}%
\pgfpathlineto{\pgfqpoint{3.289488in}{2.481757in}}%
\pgfpathlineto{\pgfqpoint{3.296860in}{2.490249in}}%
\pgfpathlineto{\pgfqpoint{3.304003in}{2.490249in}}%
\pgfpathlineto{\pgfqpoint{3.310917in}{2.487420in}}%
\pgfpathlineto{\pgfqpoint{3.318289in}{2.493077in}}%
\pgfpathlineto{\pgfqpoint{3.325432in}{2.501551in}}%
\pgfpathlineto{\pgfqpoint{3.332575in}{2.512830in}}%
\pgfpathlineto{\pgfqpoint{3.339489in}{2.526896in}}%
\pgfpathlineto{\pgfqpoint{3.346861in}{2.535318in}}%
\pgfpathlineto{\pgfqpoint{3.354004in}{2.552124in}}%
\pgfpathlineto{\pgfqpoint{3.361147in}{2.563299in}}%
\pgfpathlineto{\pgfqpoint{3.368290in}{2.588361in}}%
\pgfpathlineto{\pgfqpoint{3.375433in}{2.596689in}}%
\pgfpathlineto{\pgfqpoint{3.382576in}{2.582801in}}%
\pgfpathlineto{\pgfqpoint{3.389719in}{2.574452in}}%
\pgfpathlineto{\pgfqpoint{3.396862in}{2.568878in}}%
\pgfpathlineto{\pgfqpoint{3.404005in}{2.574452in}}%
\pgfpathlineto{\pgfqpoint{3.411148in}{2.585582in}}%
\pgfpathlineto{\pgfqpoint{3.416719in}{2.585582in}}%
\pgfpathlineto{\pgfqpoint{3.425433in}{2.582801in}}%
\pgfpathlineto{\pgfqpoint{3.431234in}{2.577236in}}%
\pgfpathlineto{\pgfqpoint{3.438377in}{2.582801in}}%
\pgfpathlineto{\pgfqpoint{3.445520in}{2.593914in}}%
\pgfpathlineto{\pgfqpoint{3.454005in}{2.602234in}}%
\pgfpathlineto{\pgfqpoint{3.461148in}{2.602234in}}%
\pgfpathlineto{\pgfqpoint{3.474091in}{2.607774in}}%
\pgfpathlineto{\pgfqpoint{3.482577in}{2.605005in}}%
\pgfpathlineto{\pgfqpoint{3.488377in}{2.599462in}}%
\pgfpathlineto{\pgfqpoint{3.495735in}{2.599462in}}%
\pgfpathlineto{\pgfqpoint{3.504006in}{2.616073in}}%
\pgfpathlineto{\pgfqpoint{3.511149in}{2.638142in}}%
\pgfpathlineto{\pgfqpoint{3.524307in}{2.668344in}}%
\pgfpathlineto{\pgfqpoint{3.532578in}{2.673818in}}%
\pgfpathlineto{\pgfqpoint{3.538378in}{2.679286in}}%
\pgfpathlineto{\pgfqpoint{3.552878in}{2.684748in}}%
\pgfpathlineto{\pgfqpoint{3.559807in}{2.684748in}}%
\pgfpathlineto{\pgfqpoint{3.567164in}{2.687477in}}%
\pgfpathlineto{\pgfqpoint{3.581236in}{2.687477in}}%
\pgfpathlineto{\pgfqpoint{3.588379in}{2.684748in}}%
\pgfpathlineto{\pgfqpoint{3.595736in}{2.687477in}}%
\pgfpathlineto{\pgfqpoint{3.602879in}{2.687477in}}%
\pgfpathlineto{\pgfqpoint{3.617165in}{2.676552in}}%
\pgfpathlineto{\pgfqpoint{3.624308in}{2.676552in}}%
\pgfpathlineto{\pgfqpoint{3.631451in}{2.684748in}}%
\pgfpathlineto{\pgfqpoint{3.652880in}{2.717410in}}%
\pgfpathlineto{\pgfqpoint{3.660023in}{2.720123in}}%
\pgfpathlineto{\pgfqpoint{3.667166in}{2.720123in}}%
\pgfpathlineto{\pgfqpoint{3.681452in}{2.709263in}}%
\pgfpathlineto{\pgfqpoint{3.688823in}{2.692932in}}%
\pgfpathlineto{\pgfqpoint{3.695738in}{2.660124in}}%
\pgfpathlineto{\pgfqpoint{3.717395in}{2.591138in}}%
\pgfpathlineto{\pgfqpoint{3.724310in}{2.566089in}}%
\pgfpathlineto{\pgfqpoint{3.731453in}{2.554920in}}%
\pgfpathlineto{\pgfqpoint{3.738824in}{2.532512in}}%
\pgfpathlineto{\pgfqpoint{3.745967in}{2.473252in}}%
\pgfpathlineto{\pgfqpoint{3.752882in}{2.481757in}}%
\pgfpathlineto{\pgfqpoint{3.767396in}{2.540926in}}%
\pgfpathlineto{\pgfqpoint{3.774539in}{2.554920in}}%
\pgfpathlineto{\pgfqpoint{3.781454in}{2.585582in}}%
\pgfpathlineto{\pgfqpoint{3.795968in}{2.662865in}}%
\pgfpathlineto{\pgfqpoint{3.802882in}{2.684748in}}%
\pgfpathlineto{\pgfqpoint{3.810025in}{2.701103in}}%
\pgfpathlineto{\pgfqpoint{3.817397in}{2.709263in}}%
\pgfpathlineto{\pgfqpoint{3.824540in}{2.714696in}}%
\pgfpathlineto{\pgfqpoint{3.831454in}{2.717410in}}%
\pgfpathlineto{\pgfqpoint{3.838826in}{2.717410in}}%
\pgfpathlineto{\pgfqpoint{3.845969in}{2.714696in}}%
\pgfpathlineto{\pgfqpoint{3.860255in}{2.714696in}}%
\pgfpathlineto{\pgfqpoint{3.867398in}{2.711980in}}%
\pgfpathlineto{\pgfqpoint{3.874541in}{2.714696in}}%
\pgfpathlineto{\pgfqpoint{3.881684in}{2.714696in}}%
\pgfpathlineto{\pgfqpoint{3.895970in}{2.709263in}}%
\pgfpathlineto{\pgfqpoint{3.917399in}{2.709263in}}%
\pgfpathlineto{\pgfqpoint{3.924542in}{2.706544in}}%
\pgfpathlineto{\pgfqpoint{3.945970in}{2.706544in}}%
\pgfpathlineto{\pgfqpoint{3.953113in}{2.701103in}}%
\pgfpathlineto{\pgfqpoint{3.967628in}{2.695657in}}%
\pgfpathlineto{\pgfqpoint{3.974542in}{2.695657in}}%
\pgfpathlineto{\pgfqpoint{3.981914in}{2.690205in}}%
\pgfpathlineto{\pgfqpoint{3.989057in}{2.687477in}}%
\pgfpathlineto{\pgfqpoint{3.996200in}{2.682017in}}%
\pgfpathlineto{\pgfqpoint{4.003114in}{2.682017in}}%
\pgfpathlineto{\pgfqpoint{4.017629in}{2.676552in}}%
\pgfpathlineto{\pgfqpoint{4.024543in}{2.679286in}}%
\pgfpathlineto{\pgfqpoint{4.031915in}{2.679286in}}%
\pgfpathlineto{\pgfqpoint{4.039058in}{2.682017in}}%
\pgfpathlineto{\pgfqpoint{4.046201in}{2.679286in}}%
\pgfpathlineto{\pgfqpoint{4.053115in}{2.684748in}}%
\pgfpathlineto{\pgfqpoint{4.060487in}{2.687477in}}%
\pgfpathlineto{\pgfqpoint{4.067630in}{2.687477in}}%
\pgfpathlineto{\pgfqpoint{4.074773in}{2.684748in}}%
\pgfpathlineto{\pgfqpoint{4.081916in}{2.684748in}}%
\pgfpathlineto{\pgfqpoint{4.103344in}{2.692932in}}%
\pgfpathlineto{\pgfqpoint{4.110487in}{2.692932in}}%
\pgfpathlineto{\pgfqpoint{4.124773in}{2.687477in}}%
\pgfpathlineto{\pgfqpoint{4.167631in}{2.687477in}}%
\pgfpathlineto{\pgfqpoint{4.174774in}{2.690205in}}%
\pgfpathlineto{\pgfqpoint{4.217846in}{2.690205in}}%
\pgfpathlineto{\pgfqpoint{4.224775in}{2.687477in}}%
\pgfpathlineto{\pgfqpoint{4.232132in}{2.687477in}}%
\pgfpathlineto{\pgfqpoint{4.239275in}{2.684748in}}%
\pgfpathlineto{\pgfqpoint{4.246418in}{2.687477in}}%
\pgfpathlineto{\pgfqpoint{4.282133in}{2.687477in}}%
\pgfpathlineto{\pgfqpoint{4.289276in}{2.684748in}}%
\pgfpathlineto{\pgfqpoint{4.296419in}{2.687477in}}%
\pgfpathlineto{\pgfqpoint{4.303562in}{2.684748in}}%
\pgfpathlineto{\pgfqpoint{4.317848in}{2.684748in}}%
\pgfpathlineto{\pgfqpoint{4.325219in}{2.682017in}}%
\pgfpathlineto{\pgfqpoint{4.339277in}{2.682017in}}%
\pgfpathlineto{\pgfqpoint{4.346420in}{2.679286in}}%
\pgfpathlineto{\pgfqpoint{4.367849in}{2.679286in}}%
\pgfpathlineto{\pgfqpoint{4.375220in}{2.676552in}}%
\pgfpathlineto{\pgfqpoint{4.382363in}{2.676552in}}%
\pgfpathlineto{\pgfqpoint{4.389278in}{2.673818in}}%
\pgfpathlineto{\pgfqpoint{4.403792in}{2.673818in}}%
\pgfpathlineto{\pgfqpoint{4.410935in}{2.671081in}}%
\pgfpathlineto{\pgfqpoint{4.417850in}{2.671081in}}%
\pgfpathlineto{\pgfqpoint{4.425221in}{2.668344in}}%
\pgfpathlineto{\pgfqpoint{4.439507in}{2.668344in}}%
\pgfpathlineto{\pgfqpoint{4.460936in}{2.660124in}}%
\pgfpathlineto{\pgfqpoint{4.468079in}{2.654637in}}%
\pgfpathlineto{\pgfqpoint{4.482365in}{2.649144in}}%
\pgfpathlineto{\pgfqpoint{4.489508in}{2.649144in}}%
\pgfpathlineto{\pgfqpoint{4.496879in}{2.651891in}}%
\pgfpathlineto{\pgfqpoint{4.503794in}{2.649144in}}%
\pgfpathlineto{\pgfqpoint{4.510937in}{2.649144in}}%
\pgfpathlineto{\pgfqpoint{4.518080in}{2.640895in}}%
\pgfpathlineto{\pgfqpoint{4.525223in}{2.635389in}}%
\pgfpathlineto{\pgfqpoint{4.546880in}{2.635389in}}%
\pgfpathlineto{\pgfqpoint{4.553795in}{2.638142in}}%
\pgfpathlineto{\pgfqpoint{4.560938in}{2.638142in}}%
\pgfpathlineto{\pgfqpoint{4.568309in}{2.640895in}}%
\pgfpathlineto{\pgfqpoint{4.575452in}{2.640895in}}%
\pgfpathlineto{\pgfqpoint{4.582366in}{2.638142in}}%
\pgfpathlineto{\pgfqpoint{4.589509in}{2.629877in}}%
\pgfpathlineto{\pgfqpoint{4.596881in}{2.624359in}}%
\pgfpathlineto{\pgfqpoint{4.604024in}{2.621599in}}%
\pgfpathlineto{\pgfqpoint{4.610938in}{2.624359in}}%
\pgfpathlineto{\pgfqpoint{4.625453in}{2.640895in}}%
\pgfpathlineto{\pgfqpoint{4.654025in}{2.651891in}}%
\pgfpathlineto{\pgfqpoint{4.702683in}{2.651891in}}%
\pgfpathlineto{\pgfqpoint{4.711169in}{2.649144in}}%
\pgfpathlineto{\pgfqpoint{4.716740in}{2.649144in}}%
\pgfpathlineto{\pgfqpoint{4.725454in}{2.646396in}}%
\pgfpathlineto{\pgfqpoint{4.754026in}{2.613308in}}%
\pgfpathlineto{\pgfqpoint{4.761169in}{2.613308in}}%
\pgfpathlineto{\pgfqpoint{4.788398in}{2.627119in}}%
\pgfpathlineto{\pgfqpoint{4.795756in}{2.627119in}}%
\pgfpathlineto{\pgfqpoint{4.804027in}{2.629877in}}%
\pgfpathlineto{\pgfqpoint{4.832599in}{2.618836in}}%
\pgfpathlineto{\pgfqpoint{4.838399in}{2.618836in}}%
\pgfpathlineto{\pgfqpoint{4.845756in}{2.621599in}}%
\pgfpathlineto{\pgfqpoint{4.852899in}{2.621599in}}%
\pgfpathlineto{\pgfqpoint{4.859828in}{2.624359in}}%
\pgfpathlineto{\pgfqpoint{4.874328in}{2.624359in}}%
\pgfpathlineto{\pgfqpoint{4.881471in}{2.621599in}}%
\pgfpathlineto{\pgfqpoint{4.981473in}{2.621599in}}%
\pgfpathlineto{\pgfqpoint{4.988844in}{2.624359in}}%
\pgfpathlineto{\pgfqpoint{5.002902in}{2.618836in}}%
\pgfpathlineto{\pgfqpoint{5.010045in}{2.613308in}}%
\pgfpathlineto{\pgfqpoint{5.017416in}{2.605005in}}%
\pgfpathlineto{\pgfqpoint{5.024331in}{2.588361in}}%
\pgfpathlineto{\pgfqpoint{5.052903in}{2.532512in}}%
\pgfpathlineto{\pgfqpoint{5.060046in}{2.529705in}}%
\pgfpathlineto{\pgfqpoint{5.067417in}{2.535318in}}%
\pgfpathlineto{\pgfqpoint{5.074560in}{2.546528in}}%
\pgfpathlineto{\pgfqpoint{5.081475in}{2.549327in}}%
\pgfpathlineto{\pgfqpoint{5.088846in}{2.549327in}}%
\pgfpathlineto{\pgfqpoint{5.095989in}{2.554920in}}%
\pgfpathlineto{\pgfqpoint{5.103132in}{2.557714in}}%
\pgfpathlineto{\pgfqpoint{5.117418in}{2.568878in}}%
\pgfpathlineto{\pgfqpoint{5.124561in}{2.568878in}}%
\pgfpathlineto{\pgfqpoint{5.145990in}{2.577236in}}%
\pgfpathlineto{\pgfqpoint{5.153133in}{2.571666in}}%
\pgfpathlineto{\pgfqpoint{5.160276in}{2.563299in}}%
\pgfpathlineto{\pgfqpoint{5.167419in}{2.552124in}}%
\pgfpathlineto{\pgfqpoint{5.174562in}{2.554920in}}%
\pgfpathlineto{\pgfqpoint{5.210505in}{2.582801in}}%
\pgfpathlineto{\pgfqpoint{5.217420in}{2.582801in}}%
\pgfpathlineto{\pgfqpoint{5.239077in}{2.574452in}}%
\pgfpathlineto{\pgfqpoint{5.253134in}{2.563299in}}%
\pgfpathlineto{\pgfqpoint{5.260506in}{2.554920in}}%
\pgfpathlineto{\pgfqpoint{5.267649in}{2.549327in}}%
\pgfpathlineto{\pgfqpoint{5.274563in}{2.538123in}}%
\pgfpathlineto{\pgfqpoint{5.281935in}{2.507193in}}%
\pgfpathlineto{\pgfqpoint{5.289078in}{2.447656in}}%
\pgfpathlineto{\pgfqpoint{5.296221in}{2.461891in}}%
\pgfpathlineto{\pgfqpoint{5.310507in}{2.521274in}}%
\pgfpathlineto{\pgfqpoint{5.317650in}{2.540926in}}%
\pgfpathlineto{\pgfqpoint{5.324793in}{2.554920in}}%
\pgfpathlineto{\pgfqpoint{5.331936in}{2.566089in}}%
\pgfpathlineto{\pgfqpoint{5.339079in}{2.568878in}}%
\pgfpathlineto{\pgfqpoint{5.339079in}{2.568878in}}%
\pgfusepath{stroke}%
\end{pgfscope}%
\begin{pgfscope}%
\pgfsetrectcap%
\pgfsetmiterjoin%
\pgfsetlinewidth{0.803000pt}%
\definecolor{currentstroke}{rgb}{0.000000,0.000000,0.000000}%
\pgfsetstrokecolor{currentstroke}%
\pgfsetdash{}{0pt}%
\pgfpathmoveto{\pgfqpoint{0.530556in}{0.515123in}}%
\pgfpathlineto{\pgfqpoint{0.530556in}{2.825123in}}%
\pgfusepath{stroke}%
\end{pgfscope}%
\begin{pgfscope}%
\pgfsetrectcap%
\pgfsetmiterjoin%
\pgfsetlinewidth{0.803000pt}%
\definecolor{currentstroke}{rgb}{0.000000,0.000000,0.000000}%
\pgfsetstrokecolor{currentstroke}%
\pgfsetdash{}{0pt}%
\pgfpathmoveto{\pgfqpoint{5.568056in}{0.515123in}}%
\pgfpathlineto{\pgfqpoint{5.568056in}{2.825123in}}%
\pgfusepath{stroke}%
\end{pgfscope}%
\begin{pgfscope}%
\pgfsetrectcap%
\pgfsetmiterjoin%
\pgfsetlinewidth{0.803000pt}%
\definecolor{currentstroke}{rgb}{0.000000,0.000000,0.000000}%
\pgfsetstrokecolor{currentstroke}%
\pgfsetdash{}{0pt}%
\pgfpathmoveto{\pgfqpoint{0.530556in}{0.515123in}}%
\pgfpathlineto{\pgfqpoint{5.568056in}{0.515123in}}%
\pgfusepath{stroke}%
\end{pgfscope}%
\begin{pgfscope}%
\pgfsetrectcap%
\pgfsetmiterjoin%
\pgfsetlinewidth{0.803000pt}%
\definecolor{currentstroke}{rgb}{0.000000,0.000000,0.000000}%
\pgfsetstrokecolor{currentstroke}%
\pgfsetdash{}{0pt}%
\pgfpathmoveto{\pgfqpoint{0.530556in}{2.825123in}}%
\pgfpathlineto{\pgfqpoint{5.568056in}{2.825123in}}%
\pgfusepath{stroke}%
\end{pgfscope}%
\begin{pgfscope}%
\definecolor{textcolor}{rgb}{0.000000,0.000000,0.000000}%
\pgfsetstrokecolor{textcolor}%
\pgfsetfillcolor{textcolor}%
\pgftext[x=3.049306in,y=2.908457in,,base]{\color{textcolor}\rmfamily\fontsize{12.000000}{14.400000}\selectfont Part II, KOH(s) + HCl \(\displaystyle \to\) KCl(aq) + H\(\displaystyle _2\)O(l)}%
\end{pgfscope}%
\end{pgfpicture}%
\makeatother%
\endgroup%

    \end{center}
    \caption{Temperature profile of the dissolution of solid \koh in a 1.5 M \hcl solution and the associated reaction.
   		Larger dips are result of experimental operator error:
		the thermistor was partially removed from the solution momentarily }\label{fig:partII}
\end{figure}

\begin{figure}[H]
    \begin{center}
	%% Creator: Matplotlib, PGF backend
%%
%% To include the figure in your LaTeX document, write
%%   \input{<filename>.pgf}
%%
%% Make sure the required packages are loaded in your preamble
%%   \usepackage{pgf}
%%
%% Figures using additional raster images can only be included by \input if
%% they are in the same directory as the main LaTeX file. For loading figures
%% from other directories you can use the `import` package
%%   \usepackage{import}
%% and then include the figures with
%%   \import{<path to file>}{<filename>.pgf}
%%
%% Matplotlib used the following preamble
%%
\begingroup%
\makeatletter%
\begin{pgfpicture}%
\pgfpathrectangle{\pgfpointorigin}{\pgfqpoint{5.845526in}{3.133457in}}%
\pgfusepath{use as bounding box, clip}%
\begin{pgfscope}%
\pgfsetbuttcap%
\pgfsetmiterjoin%
\definecolor{currentfill}{rgb}{1.000000,1.000000,1.000000}%
\pgfsetfillcolor{currentfill}%
\pgfsetlinewidth{0.000000pt}%
\definecolor{currentstroke}{rgb}{1.000000,1.000000,1.000000}%
\pgfsetstrokecolor{currentstroke}%
\pgfsetdash{}{0pt}%
\pgfpathmoveto{\pgfqpoint{0.000000in}{0.000000in}}%
\pgfpathlineto{\pgfqpoint{5.845526in}{0.000000in}}%
\pgfpathlineto{\pgfqpoint{5.845526in}{3.133457in}}%
\pgfpathlineto{\pgfqpoint{0.000000in}{3.133457in}}%
\pgfpathclose%
\pgfusepath{fill}%
\end{pgfscope}%
\begin{pgfscope}%
\pgfsetbuttcap%
\pgfsetmiterjoin%
\definecolor{currentfill}{rgb}{1.000000,1.000000,1.000000}%
\pgfsetfillcolor{currentfill}%
\pgfsetlinewidth{0.000000pt}%
\definecolor{currentstroke}{rgb}{0.000000,0.000000,0.000000}%
\pgfsetstrokecolor{currentstroke}%
\pgfsetstrokeopacity{0.000000}%
\pgfsetdash{}{0pt}%
\pgfpathmoveto{\pgfqpoint{0.708026in}{0.515123in}}%
\pgfpathlineto{\pgfqpoint{5.745526in}{0.515123in}}%
\pgfpathlineto{\pgfqpoint{5.745526in}{2.825123in}}%
\pgfpathlineto{\pgfqpoint{0.708026in}{2.825123in}}%
\pgfpathclose%
\pgfusepath{fill}%
\end{pgfscope}%
\begin{pgfscope}%
\pgfsetbuttcap%
\pgfsetroundjoin%
\definecolor{currentfill}{rgb}{0.000000,0.000000,0.000000}%
\pgfsetfillcolor{currentfill}%
\pgfsetlinewidth{0.803000pt}%
\definecolor{currentstroke}{rgb}{0.000000,0.000000,0.000000}%
\pgfsetstrokecolor{currentstroke}%
\pgfsetdash{}{0pt}%
\pgfsys@defobject{currentmarker}{\pgfqpoint{0.000000in}{-0.048611in}}{\pgfqpoint{0.000000in}{0.000000in}}{%
\pgfpathmoveto{\pgfqpoint{0.000000in}{0.000000in}}%
\pgfpathlineto{\pgfqpoint{0.000000in}{-0.048611in}}%
\pgfusepath{stroke,fill}%
}%
\begin{pgfscope}%
\pgfsys@transformshift{0.937003in}{0.515123in}%
\pgfsys@useobject{currentmarker}{}%
\end{pgfscope}%
\end{pgfscope}%
\begin{pgfscope}%
\definecolor{textcolor}{rgb}{0.000000,0.000000,0.000000}%
\pgfsetstrokecolor{textcolor}%
\pgfsetfillcolor{textcolor}%
\pgftext[x=0.937003in,y=0.417901in,,top]{\color{textcolor}\rmfamily\fontsize{10.000000}{12.000000}\selectfont \(\displaystyle 0\)}%
\end{pgfscope}%
\begin{pgfscope}%
\pgfsetbuttcap%
\pgfsetroundjoin%
\definecolor{currentfill}{rgb}{0.000000,0.000000,0.000000}%
\pgfsetfillcolor{currentfill}%
\pgfsetlinewidth{0.803000pt}%
\definecolor{currentstroke}{rgb}{0.000000,0.000000,0.000000}%
\pgfsetstrokecolor{currentstroke}%
\pgfsetdash{}{0pt}%
\pgfsys@defobject{currentmarker}{\pgfqpoint{0.000000in}{-0.048611in}}{\pgfqpoint{0.000000in}{0.000000in}}{%
\pgfpathmoveto{\pgfqpoint{0.000000in}{0.000000in}}%
\pgfpathlineto{\pgfqpoint{0.000000in}{-0.048611in}}%
\pgfusepath{stroke,fill}%
}%
\begin{pgfscope}%
\pgfsys@transformshift{1.633268in}{0.515123in}%
\pgfsys@useobject{currentmarker}{}%
\end{pgfscope}%
\end{pgfscope}%
\begin{pgfscope}%
\definecolor{textcolor}{rgb}{0.000000,0.000000,0.000000}%
\pgfsetstrokecolor{textcolor}%
\pgfsetfillcolor{textcolor}%
\pgftext[x=1.633268in,y=0.417901in,,top]{\color{textcolor}\rmfamily\fontsize{10.000000}{12.000000}\selectfont \(\displaystyle 20\)}%
\end{pgfscope}%
\begin{pgfscope}%
\pgfsetbuttcap%
\pgfsetroundjoin%
\definecolor{currentfill}{rgb}{0.000000,0.000000,0.000000}%
\pgfsetfillcolor{currentfill}%
\pgfsetlinewidth{0.803000pt}%
\definecolor{currentstroke}{rgb}{0.000000,0.000000,0.000000}%
\pgfsetstrokecolor{currentstroke}%
\pgfsetdash{}{0pt}%
\pgfsys@defobject{currentmarker}{\pgfqpoint{0.000000in}{-0.048611in}}{\pgfqpoint{0.000000in}{0.000000in}}{%
\pgfpathmoveto{\pgfqpoint{0.000000in}{0.000000in}}%
\pgfpathlineto{\pgfqpoint{0.000000in}{-0.048611in}}%
\pgfusepath{stroke,fill}%
}%
\begin{pgfscope}%
\pgfsys@transformshift{2.329533in}{0.515123in}%
\pgfsys@useobject{currentmarker}{}%
\end{pgfscope}%
\end{pgfscope}%
\begin{pgfscope}%
\definecolor{textcolor}{rgb}{0.000000,0.000000,0.000000}%
\pgfsetstrokecolor{textcolor}%
\pgfsetfillcolor{textcolor}%
\pgftext[x=2.329533in,y=0.417901in,,top]{\color{textcolor}\rmfamily\fontsize{10.000000}{12.000000}\selectfont \(\displaystyle 40\)}%
\end{pgfscope}%
\begin{pgfscope}%
\pgfsetbuttcap%
\pgfsetroundjoin%
\definecolor{currentfill}{rgb}{0.000000,0.000000,0.000000}%
\pgfsetfillcolor{currentfill}%
\pgfsetlinewidth{0.803000pt}%
\definecolor{currentstroke}{rgb}{0.000000,0.000000,0.000000}%
\pgfsetstrokecolor{currentstroke}%
\pgfsetdash{}{0pt}%
\pgfsys@defobject{currentmarker}{\pgfqpoint{0.000000in}{-0.048611in}}{\pgfqpoint{0.000000in}{0.000000in}}{%
\pgfpathmoveto{\pgfqpoint{0.000000in}{0.000000in}}%
\pgfpathlineto{\pgfqpoint{0.000000in}{-0.048611in}}%
\pgfusepath{stroke,fill}%
}%
\begin{pgfscope}%
\pgfsys@transformshift{3.025799in}{0.515123in}%
\pgfsys@useobject{currentmarker}{}%
\end{pgfscope}%
\end{pgfscope}%
\begin{pgfscope}%
\definecolor{textcolor}{rgb}{0.000000,0.000000,0.000000}%
\pgfsetstrokecolor{textcolor}%
\pgfsetfillcolor{textcolor}%
\pgftext[x=3.025799in,y=0.417901in,,top]{\color{textcolor}\rmfamily\fontsize{10.000000}{12.000000}\selectfont \(\displaystyle 60\)}%
\end{pgfscope}%
\begin{pgfscope}%
\pgfsetbuttcap%
\pgfsetroundjoin%
\definecolor{currentfill}{rgb}{0.000000,0.000000,0.000000}%
\pgfsetfillcolor{currentfill}%
\pgfsetlinewidth{0.803000pt}%
\definecolor{currentstroke}{rgb}{0.000000,0.000000,0.000000}%
\pgfsetstrokecolor{currentstroke}%
\pgfsetdash{}{0pt}%
\pgfsys@defobject{currentmarker}{\pgfqpoint{0.000000in}{-0.048611in}}{\pgfqpoint{0.000000in}{0.000000in}}{%
\pgfpathmoveto{\pgfqpoint{0.000000in}{0.000000in}}%
\pgfpathlineto{\pgfqpoint{0.000000in}{-0.048611in}}%
\pgfusepath{stroke,fill}%
}%
\begin{pgfscope}%
\pgfsys@transformshift{3.722064in}{0.515123in}%
\pgfsys@useobject{currentmarker}{}%
\end{pgfscope}%
\end{pgfscope}%
\begin{pgfscope}%
\definecolor{textcolor}{rgb}{0.000000,0.000000,0.000000}%
\pgfsetstrokecolor{textcolor}%
\pgfsetfillcolor{textcolor}%
\pgftext[x=3.722064in,y=0.417901in,,top]{\color{textcolor}\rmfamily\fontsize{10.000000}{12.000000}\selectfont \(\displaystyle 80\)}%
\end{pgfscope}%
\begin{pgfscope}%
\pgfsetbuttcap%
\pgfsetroundjoin%
\definecolor{currentfill}{rgb}{0.000000,0.000000,0.000000}%
\pgfsetfillcolor{currentfill}%
\pgfsetlinewidth{0.803000pt}%
\definecolor{currentstroke}{rgb}{0.000000,0.000000,0.000000}%
\pgfsetstrokecolor{currentstroke}%
\pgfsetdash{}{0pt}%
\pgfsys@defobject{currentmarker}{\pgfqpoint{0.000000in}{-0.048611in}}{\pgfqpoint{0.000000in}{0.000000in}}{%
\pgfpathmoveto{\pgfqpoint{0.000000in}{0.000000in}}%
\pgfpathlineto{\pgfqpoint{0.000000in}{-0.048611in}}%
\pgfusepath{stroke,fill}%
}%
\begin{pgfscope}%
\pgfsys@transformshift{4.418329in}{0.515123in}%
\pgfsys@useobject{currentmarker}{}%
\end{pgfscope}%
\end{pgfscope}%
\begin{pgfscope}%
\definecolor{textcolor}{rgb}{0.000000,0.000000,0.000000}%
\pgfsetstrokecolor{textcolor}%
\pgfsetfillcolor{textcolor}%
\pgftext[x=4.418329in,y=0.417901in,,top]{\color{textcolor}\rmfamily\fontsize{10.000000}{12.000000}\selectfont \(\displaystyle 100\)}%
\end{pgfscope}%
\begin{pgfscope}%
\pgfsetbuttcap%
\pgfsetroundjoin%
\definecolor{currentfill}{rgb}{0.000000,0.000000,0.000000}%
\pgfsetfillcolor{currentfill}%
\pgfsetlinewidth{0.803000pt}%
\definecolor{currentstroke}{rgb}{0.000000,0.000000,0.000000}%
\pgfsetstrokecolor{currentstroke}%
\pgfsetdash{}{0pt}%
\pgfsys@defobject{currentmarker}{\pgfqpoint{0.000000in}{-0.048611in}}{\pgfqpoint{0.000000in}{0.000000in}}{%
\pgfpathmoveto{\pgfqpoint{0.000000in}{0.000000in}}%
\pgfpathlineto{\pgfqpoint{0.000000in}{-0.048611in}}%
\pgfusepath{stroke,fill}%
}%
\begin{pgfscope}%
\pgfsys@transformshift{5.114594in}{0.515123in}%
\pgfsys@useobject{currentmarker}{}%
\end{pgfscope}%
\end{pgfscope}%
\begin{pgfscope}%
\definecolor{textcolor}{rgb}{0.000000,0.000000,0.000000}%
\pgfsetstrokecolor{textcolor}%
\pgfsetfillcolor{textcolor}%
\pgftext[x=5.114594in,y=0.417901in,,top]{\color{textcolor}\rmfamily\fontsize{10.000000}{12.000000}\selectfont \(\displaystyle 120\)}%
\end{pgfscope}%
\begin{pgfscope}%
\definecolor{textcolor}{rgb}{0.000000,0.000000,0.000000}%
\pgfsetstrokecolor{textcolor}%
\pgfsetfillcolor{textcolor}%
\pgftext[x=3.226776in,y=0.238889in,,top]{\color{textcolor}\rmfamily\fontsize{10.000000}{12.000000}\selectfont Time (seconds)}%
\end{pgfscope}%
\begin{pgfscope}%
\pgfsetbuttcap%
\pgfsetroundjoin%
\definecolor{currentfill}{rgb}{0.000000,0.000000,0.000000}%
\pgfsetfillcolor{currentfill}%
\pgfsetlinewidth{0.803000pt}%
\definecolor{currentstroke}{rgb}{0.000000,0.000000,0.000000}%
\pgfsetstrokecolor{currentstroke}%
\pgfsetdash{}{0pt}%
\pgfsys@defobject{currentmarker}{\pgfqpoint{-0.048611in}{0.000000in}}{\pgfqpoint{0.000000in}{0.000000in}}{%
\pgfpathmoveto{\pgfqpoint{0.000000in}{0.000000in}}%
\pgfpathlineto{\pgfqpoint{-0.048611in}{0.000000in}}%
\pgfusepath{stroke,fill}%
}%
\begin{pgfscope}%
\pgfsys@transformshift{0.708026in}{0.524859in}%
\pgfsys@useobject{currentmarker}{}%
\end{pgfscope}%
\end{pgfscope}%
\begin{pgfscope}%
\definecolor{textcolor}{rgb}{0.000000,0.000000,0.000000}%
\pgfsetstrokecolor{textcolor}%
\pgfsetfillcolor{textcolor}%
\pgftext[x=0.294444in,y=0.476634in,left,base]{\color{textcolor}\rmfamily\fontsize{10.000000}{12.000000}\selectfont \(\displaystyle 22.50\)}%
\end{pgfscope}%
\begin{pgfscope}%
\pgfsetbuttcap%
\pgfsetroundjoin%
\definecolor{currentfill}{rgb}{0.000000,0.000000,0.000000}%
\pgfsetfillcolor{currentfill}%
\pgfsetlinewidth{0.803000pt}%
\definecolor{currentstroke}{rgb}{0.000000,0.000000,0.000000}%
\pgfsetstrokecolor{currentstroke}%
\pgfsetdash{}{0pt}%
\pgfsys@defobject{currentmarker}{\pgfqpoint{-0.048611in}{0.000000in}}{\pgfqpoint{0.000000in}{0.000000in}}{%
\pgfpathmoveto{\pgfqpoint{0.000000in}{0.000000in}}%
\pgfpathlineto{\pgfqpoint{-0.048611in}{0.000000in}}%
\pgfusepath{stroke,fill}%
}%
\begin{pgfscope}%
\pgfsys@transformshift{0.708026in}{0.823017in}%
\pgfsys@useobject{currentmarker}{}%
\end{pgfscope}%
\end{pgfscope}%
\begin{pgfscope}%
\definecolor{textcolor}{rgb}{0.000000,0.000000,0.000000}%
\pgfsetstrokecolor{textcolor}%
\pgfsetfillcolor{textcolor}%
\pgftext[x=0.294444in,y=0.774792in,left,base]{\color{textcolor}\rmfamily\fontsize{10.000000}{12.000000}\selectfont \(\displaystyle 22.75\)}%
\end{pgfscope}%
\begin{pgfscope}%
\pgfsetbuttcap%
\pgfsetroundjoin%
\definecolor{currentfill}{rgb}{0.000000,0.000000,0.000000}%
\pgfsetfillcolor{currentfill}%
\pgfsetlinewidth{0.803000pt}%
\definecolor{currentstroke}{rgb}{0.000000,0.000000,0.000000}%
\pgfsetstrokecolor{currentstroke}%
\pgfsetdash{}{0pt}%
\pgfsys@defobject{currentmarker}{\pgfqpoint{-0.048611in}{0.000000in}}{\pgfqpoint{0.000000in}{0.000000in}}{%
\pgfpathmoveto{\pgfqpoint{0.000000in}{0.000000in}}%
\pgfpathlineto{\pgfqpoint{-0.048611in}{0.000000in}}%
\pgfusepath{stroke,fill}%
}%
\begin{pgfscope}%
\pgfsys@transformshift{0.708026in}{1.121175in}%
\pgfsys@useobject{currentmarker}{}%
\end{pgfscope}%
\end{pgfscope}%
\begin{pgfscope}%
\definecolor{textcolor}{rgb}{0.000000,0.000000,0.000000}%
\pgfsetstrokecolor{textcolor}%
\pgfsetfillcolor{textcolor}%
\pgftext[x=0.294444in,y=1.072949in,left,base]{\color{textcolor}\rmfamily\fontsize{10.000000}{12.000000}\selectfont \(\displaystyle 23.00\)}%
\end{pgfscope}%
\begin{pgfscope}%
\pgfsetbuttcap%
\pgfsetroundjoin%
\definecolor{currentfill}{rgb}{0.000000,0.000000,0.000000}%
\pgfsetfillcolor{currentfill}%
\pgfsetlinewidth{0.803000pt}%
\definecolor{currentstroke}{rgb}{0.000000,0.000000,0.000000}%
\pgfsetstrokecolor{currentstroke}%
\pgfsetdash{}{0pt}%
\pgfsys@defobject{currentmarker}{\pgfqpoint{-0.048611in}{0.000000in}}{\pgfqpoint{0.000000in}{0.000000in}}{%
\pgfpathmoveto{\pgfqpoint{0.000000in}{0.000000in}}%
\pgfpathlineto{\pgfqpoint{-0.048611in}{0.000000in}}%
\pgfusepath{stroke,fill}%
}%
\begin{pgfscope}%
\pgfsys@transformshift{0.708026in}{1.419332in}%
\pgfsys@useobject{currentmarker}{}%
\end{pgfscope}%
\end{pgfscope}%
\begin{pgfscope}%
\definecolor{textcolor}{rgb}{0.000000,0.000000,0.000000}%
\pgfsetstrokecolor{textcolor}%
\pgfsetfillcolor{textcolor}%
\pgftext[x=0.294444in,y=1.371107in,left,base]{\color{textcolor}\rmfamily\fontsize{10.000000}{12.000000}\selectfont \(\displaystyle 23.25\)}%
\end{pgfscope}%
\begin{pgfscope}%
\pgfsetbuttcap%
\pgfsetroundjoin%
\definecolor{currentfill}{rgb}{0.000000,0.000000,0.000000}%
\pgfsetfillcolor{currentfill}%
\pgfsetlinewidth{0.803000pt}%
\definecolor{currentstroke}{rgb}{0.000000,0.000000,0.000000}%
\pgfsetstrokecolor{currentstroke}%
\pgfsetdash{}{0pt}%
\pgfsys@defobject{currentmarker}{\pgfqpoint{-0.048611in}{0.000000in}}{\pgfqpoint{0.000000in}{0.000000in}}{%
\pgfpathmoveto{\pgfqpoint{0.000000in}{0.000000in}}%
\pgfpathlineto{\pgfqpoint{-0.048611in}{0.000000in}}%
\pgfusepath{stroke,fill}%
}%
\begin{pgfscope}%
\pgfsys@transformshift{0.708026in}{1.717490in}%
\pgfsys@useobject{currentmarker}{}%
\end{pgfscope}%
\end{pgfscope}%
\begin{pgfscope}%
\definecolor{textcolor}{rgb}{0.000000,0.000000,0.000000}%
\pgfsetstrokecolor{textcolor}%
\pgfsetfillcolor{textcolor}%
\pgftext[x=0.294444in,y=1.669265in,left,base]{\color{textcolor}\rmfamily\fontsize{10.000000}{12.000000}\selectfont \(\displaystyle 23.50\)}%
\end{pgfscope}%
\begin{pgfscope}%
\pgfsetbuttcap%
\pgfsetroundjoin%
\definecolor{currentfill}{rgb}{0.000000,0.000000,0.000000}%
\pgfsetfillcolor{currentfill}%
\pgfsetlinewidth{0.803000pt}%
\definecolor{currentstroke}{rgb}{0.000000,0.000000,0.000000}%
\pgfsetstrokecolor{currentstroke}%
\pgfsetdash{}{0pt}%
\pgfsys@defobject{currentmarker}{\pgfqpoint{-0.048611in}{0.000000in}}{\pgfqpoint{0.000000in}{0.000000in}}{%
\pgfpathmoveto{\pgfqpoint{0.000000in}{0.000000in}}%
\pgfpathlineto{\pgfqpoint{-0.048611in}{0.000000in}}%
\pgfusepath{stroke,fill}%
}%
\begin{pgfscope}%
\pgfsys@transformshift{0.708026in}{2.015647in}%
\pgfsys@useobject{currentmarker}{}%
\end{pgfscope}%
\end{pgfscope}%
\begin{pgfscope}%
\definecolor{textcolor}{rgb}{0.000000,0.000000,0.000000}%
\pgfsetstrokecolor{textcolor}%
\pgfsetfillcolor{textcolor}%
\pgftext[x=0.294444in,y=1.967422in,left,base]{\color{textcolor}\rmfamily\fontsize{10.000000}{12.000000}\selectfont \(\displaystyle 23.75\)}%
\end{pgfscope}%
\begin{pgfscope}%
\pgfsetbuttcap%
\pgfsetroundjoin%
\definecolor{currentfill}{rgb}{0.000000,0.000000,0.000000}%
\pgfsetfillcolor{currentfill}%
\pgfsetlinewidth{0.803000pt}%
\definecolor{currentstroke}{rgb}{0.000000,0.000000,0.000000}%
\pgfsetstrokecolor{currentstroke}%
\pgfsetdash{}{0pt}%
\pgfsys@defobject{currentmarker}{\pgfqpoint{-0.048611in}{0.000000in}}{\pgfqpoint{0.000000in}{0.000000in}}{%
\pgfpathmoveto{\pgfqpoint{0.000000in}{0.000000in}}%
\pgfpathlineto{\pgfqpoint{-0.048611in}{0.000000in}}%
\pgfusepath{stroke,fill}%
}%
\begin{pgfscope}%
\pgfsys@transformshift{0.708026in}{2.313805in}%
\pgfsys@useobject{currentmarker}{}%
\end{pgfscope}%
\end{pgfscope}%
\begin{pgfscope}%
\definecolor{textcolor}{rgb}{0.000000,0.000000,0.000000}%
\pgfsetstrokecolor{textcolor}%
\pgfsetfillcolor{textcolor}%
\pgftext[x=0.294444in,y=2.265580in,left,base]{\color{textcolor}\rmfamily\fontsize{10.000000}{12.000000}\selectfont \(\displaystyle 24.00\)}%
\end{pgfscope}%
\begin{pgfscope}%
\pgfsetbuttcap%
\pgfsetroundjoin%
\definecolor{currentfill}{rgb}{0.000000,0.000000,0.000000}%
\pgfsetfillcolor{currentfill}%
\pgfsetlinewidth{0.803000pt}%
\definecolor{currentstroke}{rgb}{0.000000,0.000000,0.000000}%
\pgfsetstrokecolor{currentstroke}%
\pgfsetdash{}{0pt}%
\pgfsys@defobject{currentmarker}{\pgfqpoint{-0.048611in}{0.000000in}}{\pgfqpoint{0.000000in}{0.000000in}}{%
\pgfpathmoveto{\pgfqpoint{0.000000in}{0.000000in}}%
\pgfpathlineto{\pgfqpoint{-0.048611in}{0.000000in}}%
\pgfusepath{stroke,fill}%
}%
\begin{pgfscope}%
\pgfsys@transformshift{0.708026in}{2.611963in}%
\pgfsys@useobject{currentmarker}{}%
\end{pgfscope}%
\end{pgfscope}%
\begin{pgfscope}%
\definecolor{textcolor}{rgb}{0.000000,0.000000,0.000000}%
\pgfsetstrokecolor{textcolor}%
\pgfsetfillcolor{textcolor}%
\pgftext[x=0.294444in,y=2.563737in,left,base]{\color{textcolor}\rmfamily\fontsize{10.000000}{12.000000}\selectfont \(\displaystyle 24.25\)}%
\end{pgfscope}%
\begin{pgfscope}%
\definecolor{textcolor}{rgb}{0.000000,0.000000,0.000000}%
\pgfsetstrokecolor{textcolor}%
\pgfsetfillcolor{textcolor}%
\pgftext[x=0.238889in,y=1.670123in,,bottom,rotate=90.000000]{\color{textcolor}\rmfamily\fontsize{10.000000}{12.000000}\selectfont Temperature (C)}%
\end{pgfscope}%
\begin{pgfscope}%
\pgfpathrectangle{\pgfqpoint{0.708026in}{0.515123in}}{\pgfqpoint{5.037500in}{2.310000in}}%
\pgfusepath{clip}%
\pgfsetrectcap%
\pgfsetroundjoin%
\pgfsetlinewidth{1.505625pt}%
\definecolor{currentstroke}{rgb}{1.000000,0.000000,0.000000}%
\pgfsetstrokecolor{currentstroke}%
\pgfsetdash{}{0pt}%
\pgfpathmoveto{\pgfqpoint{0.937003in}{2.601870in}}%
\pgfpathlineto{\pgfqpoint{0.958204in}{2.649226in}}%
\pgfpathlineto{\pgfqpoint{0.971816in}{2.672876in}}%
\pgfpathlineto{\pgfqpoint{0.992460in}{2.720123in}}%
\pgfpathlineto{\pgfqpoint{1.006629in}{2.649226in}}%
\pgfpathlineto{\pgfqpoint{1.024036in}{2.672876in}}%
\pgfpathlineto{\pgfqpoint{1.041443in}{2.601870in}}%
\pgfpathlineto{\pgfqpoint{1.058849in}{2.601870in}}%
\pgfpathlineto{\pgfqpoint{1.076778in}{2.530701in}}%
\pgfpathlineto{\pgfqpoint{1.111069in}{2.292279in}}%
\pgfpathlineto{\pgfqpoint{1.128476in}{2.220392in}}%
\pgfpathlineto{\pgfqpoint{1.146405in}{2.268335in}}%
\pgfpathlineto{\pgfqpoint{1.163289in}{2.292279in}}%
\pgfpathlineto{\pgfqpoint{1.180696in}{2.220392in}}%
\pgfpathlineto{\pgfqpoint{1.198625in}{2.124282in}}%
\pgfpathlineto{\pgfqpoint{1.216031in}{2.100207in}}%
\pgfpathlineto{\pgfqpoint{1.232916in}{2.220392in}}%
\pgfpathlineto{\pgfqpoint{1.250322in}{2.292279in}}%
\pgfpathlineto{\pgfqpoint{1.268251in}{2.268335in}}%
\pgfpathlineto{\pgfqpoint{1.320471in}{2.268335in}}%
\pgfpathlineto{\pgfqpoint{1.337878in}{2.316204in}}%
\pgfpathlineto{\pgfqpoint{1.372169in}{2.459367in}}%
\pgfpathlineto{\pgfqpoint{1.390098in}{2.530701in}}%
\pgfpathlineto{\pgfqpoint{1.407504in}{2.578165in}}%
\pgfpathlineto{\pgfqpoint{1.424389in}{2.601870in}}%
\pgfpathlineto{\pgfqpoint{1.459724in}{2.601870in}}%
\pgfpathlineto{\pgfqpoint{1.494537in}{2.649226in}}%
\pgfpathlineto{\pgfqpoint{1.564164in}{2.649226in}}%
\pgfpathlineto{\pgfqpoint{1.581571in}{2.672876in}}%
\pgfpathlineto{\pgfqpoint{1.598977in}{2.672876in}}%
\pgfpathlineto{\pgfqpoint{1.616941in}{2.696509in}}%
\pgfpathlineto{\pgfqpoint{1.651197in}{2.696509in}}%
\pgfpathlineto{\pgfqpoint{1.668604in}{2.720123in}}%
\pgfpathlineto{\pgfqpoint{1.738787in}{2.720123in}}%
\pgfpathlineto{\pgfqpoint{1.756194in}{2.696509in}}%
\pgfpathlineto{\pgfqpoint{1.773043in}{2.720123in}}%
\pgfpathlineto{\pgfqpoint{1.791007in}{2.696509in}}%
\pgfpathlineto{\pgfqpoint{1.808414in}{2.720123in}}%
\pgfpathlineto{\pgfqpoint{1.894890in}{2.720123in}}%
\pgfpathlineto{\pgfqpoint{1.930260in}{2.672876in}}%
\pgfpathlineto{\pgfqpoint{1.947667in}{2.625557in}}%
\pgfpathlineto{\pgfqpoint{1.982480in}{2.578165in}}%
\pgfpathlineto{\pgfqpoint{1.999887in}{2.530701in}}%
\pgfpathlineto{\pgfqpoint{2.052107in}{2.316204in}}%
\pgfpathlineto{\pgfqpoint{2.069513in}{2.196392in}}%
\pgfpathlineto{\pgfqpoint{2.086920in}{2.124282in}}%
\pgfpathlineto{\pgfqpoint{2.104326in}{2.196392in}}%
\pgfpathlineto{\pgfqpoint{2.139140in}{2.435553in}}%
\pgfpathlineto{\pgfqpoint{2.157069in}{2.506941in}}%
\pgfpathlineto{\pgfqpoint{2.173953in}{2.554442in}}%
\pgfpathlineto{\pgfqpoint{2.243580in}{2.649226in}}%
\pgfpathlineto{\pgfqpoint{2.260986in}{2.578165in}}%
\pgfpathlineto{\pgfqpoint{2.278915in}{2.483163in}}%
\pgfpathlineto{\pgfqpoint{2.296322in}{2.435553in}}%
\pgfpathlineto{\pgfqpoint{2.313206in}{2.363999in}}%
\pgfpathlineto{\pgfqpoint{2.331135in}{2.316204in}}%
\pgfpathlineto{\pgfqpoint{2.348542in}{2.292279in}}%
\pgfpathlineto{\pgfqpoint{2.365948in}{2.244373in}}%
\pgfpathlineto{\pgfqpoint{2.400761in}{2.196392in}}%
\pgfpathlineto{\pgfqpoint{2.418168in}{2.172374in}}%
\pgfpathlineto{\pgfqpoint{2.435575in}{2.124282in}}%
\pgfpathlineto{\pgfqpoint{2.452981in}{2.124282in}}%
\pgfpathlineto{\pgfqpoint{2.470388in}{2.148337in}}%
\pgfpathlineto{\pgfqpoint{2.540014in}{2.148337in}}%
\pgfpathlineto{\pgfqpoint{2.557421in}{2.100207in}}%
\pgfpathlineto{\pgfqpoint{2.574828in}{2.076114in}}%
\pgfpathlineto{\pgfqpoint{2.592234in}{2.003721in}}%
\pgfpathlineto{\pgfqpoint{2.609641in}{1.858423in}}%
\pgfpathlineto{\pgfqpoint{2.627605in}{1.614725in}}%
\pgfpathlineto{\pgfqpoint{2.644454in}{1.369081in}}%
\pgfpathlineto{\pgfqpoint{2.661861in}{1.146309in}}%
\pgfpathlineto{\pgfqpoint{2.679268in}{1.195954in}}%
\pgfpathlineto{\pgfqpoint{2.697231in}{1.344409in}}%
\pgfpathlineto{\pgfqpoint{2.714081in}{1.442981in}}%
\pgfpathlineto{\pgfqpoint{2.731487in}{1.467575in}}%
\pgfpathlineto{\pgfqpoint{2.749451in}{1.516703in}}%
\pgfpathlineto{\pgfqpoint{2.766858in}{1.492149in}}%
\pgfpathlineto{\pgfqpoint{2.783707in}{1.442981in}}%
\pgfpathlineto{\pgfqpoint{2.801114in}{1.418368in}}%
\pgfpathlineto{\pgfqpoint{2.819078in}{1.418368in}}%
\pgfpathlineto{\pgfqpoint{2.836484in}{1.393734in}}%
\pgfpathlineto{\pgfqpoint{2.853334in}{1.393734in}}%
\pgfpathlineto{\pgfqpoint{2.871297in}{1.418368in}}%
\pgfpathlineto{\pgfqpoint{2.888704in}{1.467575in}}%
\pgfpathlineto{\pgfqpoint{2.906111in}{1.492149in}}%
\pgfpathlineto{\pgfqpoint{2.923517in}{1.565753in}}%
\pgfpathlineto{\pgfqpoint{2.975737in}{1.639182in}}%
\pgfpathlineto{\pgfqpoint{3.010551in}{1.639182in}}%
\pgfpathlineto{\pgfqpoint{3.045921in}{1.541238in}}%
\pgfpathlineto{\pgfqpoint{3.062770in}{1.418368in}}%
\pgfpathlineto{\pgfqpoint{3.080177in}{1.195954in}}%
\pgfpathlineto{\pgfqpoint{3.097584in}{1.046778in}}%
\pgfpathlineto{\pgfqpoint{3.115547in}{1.096584in}}%
\pgfpathlineto{\pgfqpoint{3.149804in}{1.393734in}}%
\pgfpathlineto{\pgfqpoint{3.167767in}{1.492149in}}%
\pgfpathlineto{\pgfqpoint{3.185174in}{1.541238in}}%
\pgfpathlineto{\pgfqpoint{3.202023in}{1.614725in}}%
\pgfpathlineto{\pgfqpoint{3.219987in}{1.639182in}}%
\pgfpathlineto{\pgfqpoint{3.237394in}{1.614725in}}%
\pgfpathlineto{\pgfqpoint{3.254243in}{1.565753in}}%
\pgfpathlineto{\pgfqpoint{3.271650in}{1.492149in}}%
\pgfpathlineto{\pgfqpoint{3.289614in}{1.295004in}}%
\pgfpathlineto{\pgfqpoint{3.307020in}{1.071691in}}%
\pgfpathlineto{\pgfqpoint{3.323870in}{1.096584in}}%
\pgfpathlineto{\pgfqpoint{3.341834in}{1.270271in}}%
\pgfpathlineto{\pgfqpoint{3.359240in}{1.418368in}}%
\pgfpathlineto{\pgfqpoint{3.376647in}{1.442981in}}%
\pgfpathlineto{\pgfqpoint{3.393496in}{1.393734in}}%
\pgfpathlineto{\pgfqpoint{3.411460in}{1.369081in}}%
\pgfpathlineto{\pgfqpoint{3.428867in}{1.319716in}}%
\pgfpathlineto{\pgfqpoint{3.446273in}{1.295004in}}%
\pgfpathlineto{\pgfqpoint{3.463680in}{1.245519in}}%
\pgfpathlineto{\pgfqpoint{3.481087in}{1.245519in}}%
\pgfpathlineto{\pgfqpoint{3.515900in}{1.195954in}}%
\pgfpathlineto{\pgfqpoint{3.533306in}{1.195954in}}%
\pgfpathlineto{\pgfqpoint{3.550713in}{1.245519in}}%
\pgfpathlineto{\pgfqpoint{3.568120in}{1.270271in}}%
\pgfpathlineto{\pgfqpoint{3.581697in}{1.270271in}}%
\pgfpathlineto{\pgfqpoint{3.602933in}{1.344409in}}%
\pgfpathlineto{\pgfqpoint{3.620340in}{1.418368in}}%
\pgfpathlineto{\pgfqpoint{3.634474in}{1.492149in}}%
\pgfpathlineto{\pgfqpoint{3.651880in}{1.565753in}}%
\pgfpathlineto{\pgfqpoint{3.672560in}{1.688037in}}%
\pgfpathlineto{\pgfqpoint{3.689966in}{1.761175in}}%
\pgfpathlineto{\pgfqpoint{3.704100in}{1.809837in}}%
\pgfpathlineto{\pgfqpoint{3.721507in}{1.858423in}}%
\pgfpathlineto{\pgfqpoint{3.742186in}{1.906932in}}%
\pgfpathlineto{\pgfqpoint{3.756320in}{1.979552in}}%
\pgfpathlineto{\pgfqpoint{3.774284in}{2.027871in}}%
\pgfpathlineto{\pgfqpoint{3.794406in}{2.100207in}}%
\pgfpathlineto{\pgfqpoint{3.811813in}{2.076114in}}%
\pgfpathlineto{\pgfqpoint{3.826504in}{2.076114in}}%
\pgfpathlineto{\pgfqpoint{3.843910in}{2.027871in}}%
\pgfpathlineto{\pgfqpoint{3.864032in}{1.931157in}}%
\pgfpathlineto{\pgfqpoint{3.878167in}{1.736815in}}%
\pgfpathlineto{\pgfqpoint{3.896130in}{1.712436in}}%
\pgfpathlineto{\pgfqpoint{3.930387in}{1.955364in}}%
\pgfpathlineto{\pgfqpoint{3.948350in}{2.052002in}}%
\pgfpathlineto{\pgfqpoint{3.965757in}{2.172374in}}%
\pgfpathlineto{\pgfqpoint{3.983163in}{2.220392in}}%
\pgfpathlineto{\pgfqpoint{4.017977in}{2.268335in}}%
\pgfpathlineto{\pgfqpoint{4.052233in}{2.268335in}}%
\pgfpathlineto{\pgfqpoint{4.105010in}{2.340111in}}%
\pgfpathlineto{\pgfqpoint{4.139823in}{2.387869in}}%
\pgfpathlineto{\pgfqpoint{4.157230in}{2.411720in}}%
\pgfpathlineto{\pgfqpoint{4.174636in}{2.411720in}}%
\pgfpathlineto{\pgfqpoint{4.192043in}{2.387869in}}%
\pgfpathlineto{\pgfqpoint{4.226856in}{2.387869in}}%
\pgfpathlineto{\pgfqpoint{4.244785in}{2.363999in}}%
\pgfpathlineto{\pgfqpoint{4.261670in}{2.363999in}}%
\pgfpathlineto{\pgfqpoint{4.279076in}{2.387869in}}%
\pgfpathlineto{\pgfqpoint{4.314412in}{2.387869in}}%
\pgfpathlineto{\pgfqpoint{4.331296in}{2.411720in}}%
\pgfpathlineto{\pgfqpoint{4.348703in}{2.387869in}}%
\pgfpathlineto{\pgfqpoint{4.366632in}{2.411720in}}%
\pgfpathlineto{\pgfqpoint{4.384038in}{2.387869in}}%
\pgfpathlineto{\pgfqpoint{4.400923in}{2.411720in}}%
\pgfpathlineto{\pgfqpoint{4.418329in}{2.411720in}}%
\pgfpathlineto{\pgfqpoint{4.453665in}{2.363999in}}%
\pgfpathlineto{\pgfqpoint{4.523291in}{2.363999in}}%
\pgfpathlineto{\pgfqpoint{4.540698in}{2.387869in}}%
\pgfpathlineto{\pgfqpoint{4.558104in}{2.363999in}}%
\pgfpathlineto{\pgfqpoint{4.575511in}{2.363999in}}%
\pgfpathlineto{\pgfqpoint{4.592396in}{2.316204in}}%
\pgfpathlineto{\pgfqpoint{4.610324in}{2.292279in}}%
\pgfpathlineto{\pgfqpoint{4.627731in}{2.244373in}}%
\pgfpathlineto{\pgfqpoint{4.645138in}{2.244373in}}%
\pgfpathlineto{\pgfqpoint{4.663101in}{2.220392in}}%
\pgfpathlineto{\pgfqpoint{4.697358in}{2.220392in}}%
\pgfpathlineto{\pgfqpoint{4.714764in}{2.172374in}}%
\pgfpathlineto{\pgfqpoint{4.854574in}{2.363999in}}%
\pgfpathlineto{\pgfqpoint{4.871424in}{2.340111in}}%
\pgfpathlineto{\pgfqpoint{4.888830in}{2.340111in}}%
\pgfpathlineto{\pgfqpoint{4.906794in}{2.316204in}}%
\pgfpathlineto{\pgfqpoint{4.959014in}{2.387869in}}%
\pgfpathlineto{\pgfqpoint{5.028641in}{2.387869in}}%
\pgfpathlineto{\pgfqpoint{5.046047in}{2.363999in}}%
\pgfpathlineto{\pgfqpoint{5.063454in}{2.363999in}}%
\pgfpathlineto{\pgfqpoint{5.080860in}{2.340111in}}%
\pgfpathlineto{\pgfqpoint{5.098267in}{2.292279in}}%
\pgfpathlineto{\pgfqpoint{5.115674in}{2.292279in}}%
\pgfpathlineto{\pgfqpoint{5.150487in}{2.244373in}}%
\pgfpathlineto{\pgfqpoint{5.167894in}{2.124282in}}%
\pgfpathlineto{\pgfqpoint{5.185300in}{2.124282in}}%
\pgfpathlineto{\pgfqpoint{5.203229in}{2.100207in}}%
\pgfpathlineto{\pgfqpoint{5.220113in}{2.100207in}}%
\pgfpathlineto{\pgfqpoint{5.237520in}{2.076114in}}%
\pgfpathlineto{\pgfqpoint{5.255449in}{2.003721in}}%
\pgfpathlineto{\pgfqpoint{5.272333in}{1.906932in}}%
\pgfpathlineto{\pgfqpoint{5.289740in}{1.882687in}}%
\pgfpathlineto{\pgfqpoint{5.307147in}{2.052002in}}%
\pgfpathlineto{\pgfqpoint{5.325075in}{2.100207in}}%
\pgfpathlineto{\pgfqpoint{5.341960in}{2.076114in}}%
\pgfpathlineto{\pgfqpoint{5.359367in}{1.955364in}}%
\pgfpathlineto{\pgfqpoint{5.394702in}{1.809837in}}%
\pgfpathlineto{\pgfqpoint{5.411586in}{1.761175in}}%
\pgfpathlineto{\pgfqpoint{5.446922in}{1.565753in}}%
\pgfpathlineto{\pgfqpoint{5.464329in}{1.393734in}}%
\pgfpathlineto{\pgfqpoint{5.481213in}{1.171142in}}%
\pgfpathlineto{\pgfqpoint{5.516548in}{0.620123in}}%
\pgfpathlineto{\pgfqpoint{5.516548in}{0.620123in}}%
\pgfusepath{stroke}%
\end{pgfscope}%
\begin{pgfscope}%
\pgfsetrectcap%
\pgfsetmiterjoin%
\pgfsetlinewidth{0.803000pt}%
\definecolor{currentstroke}{rgb}{0.000000,0.000000,0.000000}%
\pgfsetstrokecolor{currentstroke}%
\pgfsetdash{}{0pt}%
\pgfpathmoveto{\pgfqpoint{0.708026in}{0.515123in}}%
\pgfpathlineto{\pgfqpoint{0.708026in}{2.825123in}}%
\pgfusepath{stroke}%
\end{pgfscope}%
\begin{pgfscope}%
\pgfsetrectcap%
\pgfsetmiterjoin%
\pgfsetlinewidth{0.803000pt}%
\definecolor{currentstroke}{rgb}{0.000000,0.000000,0.000000}%
\pgfsetstrokecolor{currentstroke}%
\pgfsetdash{}{0pt}%
\pgfpathmoveto{\pgfqpoint{5.745526in}{0.515123in}}%
\pgfpathlineto{\pgfqpoint{5.745526in}{2.825123in}}%
\pgfusepath{stroke}%
\end{pgfscope}%
\begin{pgfscope}%
\pgfsetrectcap%
\pgfsetmiterjoin%
\pgfsetlinewidth{0.803000pt}%
\definecolor{currentstroke}{rgb}{0.000000,0.000000,0.000000}%
\pgfsetstrokecolor{currentstroke}%
\pgfsetdash{}{0pt}%
\pgfpathmoveto{\pgfqpoint{0.708026in}{0.515123in}}%
\pgfpathlineto{\pgfqpoint{5.745526in}{0.515123in}}%
\pgfusepath{stroke}%
\end{pgfscope}%
\begin{pgfscope}%
\pgfsetrectcap%
\pgfsetmiterjoin%
\pgfsetlinewidth{0.803000pt}%
\definecolor{currentstroke}{rgb}{0.000000,0.000000,0.000000}%
\pgfsetstrokecolor{currentstroke}%
\pgfsetdash{}{0pt}%
\pgfpathmoveto{\pgfqpoint{0.708026in}{2.825123in}}%
\pgfpathlineto{\pgfqpoint{5.745526in}{2.825123in}}%
\pgfusepath{stroke}%
\end{pgfscope}%
\begin{pgfscope}%
\definecolor{textcolor}{rgb}{0.000000,0.000000,0.000000}%
\pgfsetstrokecolor{textcolor}%
\pgfsetfillcolor{textcolor}%
\pgftext[x=3.226776in,y=2.908457in,,base]{\color{textcolor}\rmfamily\fontsize{12.000000}{14.400000}\selectfont Part III, KOH(aq)}%
\end{pgfscope}%
\end{pgfpicture}%
\makeatother%
\endgroup%

    \end{center}
    \caption{Temperature profile of stirred, aqueous \koh over 2 minute duration. }\label{fig:partIIIa}
\end{figure}


\begin{figure}[H]
    \begin{center}
	%% Creator: Matplotlib, PGF backend
%%
%% To include the figure in your LaTeX document, write
%%   \input{<filename>.pgf}
%%
%% Make sure the required packages are loaded in your preamble
%%   \usepackage{pgf}
%%
%% Figures using additional raster images can only be included by \input if
%% they are in the same directory as the main LaTeX file. For loading figures
%% from other directories you can use the `import` package
%%   \usepackage{import}
%% and then include the figures with
%%   \import{<path to file>}{<filename>.pgf}
%%
%% Matplotlib used the following preamble
%%
\begingroup%
\makeatletter%
\begin{pgfpicture}%
\pgfpathrectangle{\pgfpointorigin}{\pgfqpoint{5.668056in}{3.133457in}}%
\pgfusepath{use as bounding box, clip}%
\begin{pgfscope}%
\pgfsetbuttcap%
\pgfsetmiterjoin%
\definecolor{currentfill}{rgb}{1.000000,1.000000,1.000000}%
\pgfsetfillcolor{currentfill}%
\pgfsetlinewidth{0.000000pt}%
\definecolor{currentstroke}{rgb}{1.000000,1.000000,1.000000}%
\pgfsetstrokecolor{currentstroke}%
\pgfsetdash{}{0pt}%
\pgfpathmoveto{\pgfqpoint{0.000000in}{0.000000in}}%
\pgfpathlineto{\pgfqpoint{5.668056in}{0.000000in}}%
\pgfpathlineto{\pgfqpoint{5.668056in}{3.133457in}}%
\pgfpathlineto{\pgfqpoint{0.000000in}{3.133457in}}%
\pgfpathclose%
\pgfusepath{fill}%
\end{pgfscope}%
\begin{pgfscope}%
\pgfsetbuttcap%
\pgfsetmiterjoin%
\definecolor{currentfill}{rgb}{1.000000,1.000000,1.000000}%
\pgfsetfillcolor{currentfill}%
\pgfsetlinewidth{0.000000pt}%
\definecolor{currentstroke}{rgb}{0.000000,0.000000,0.000000}%
\pgfsetstrokecolor{currentstroke}%
\pgfsetstrokeopacity{0.000000}%
\pgfsetdash{}{0pt}%
\pgfpathmoveto{\pgfqpoint{0.530556in}{0.515123in}}%
\pgfpathlineto{\pgfqpoint{5.568056in}{0.515123in}}%
\pgfpathlineto{\pgfqpoint{5.568056in}{2.825123in}}%
\pgfpathlineto{\pgfqpoint{0.530556in}{2.825123in}}%
\pgfpathclose%
\pgfusepath{fill}%
\end{pgfscope}%
\begin{pgfscope}%
\pgfsetbuttcap%
\pgfsetroundjoin%
\definecolor{currentfill}{rgb}{0.000000,0.000000,0.000000}%
\pgfsetfillcolor{currentfill}%
\pgfsetlinewidth{0.803000pt}%
\definecolor{currentstroke}{rgb}{0.000000,0.000000,0.000000}%
\pgfsetstrokecolor{currentstroke}%
\pgfsetdash{}{0pt}%
\pgfsys@defobject{currentmarker}{\pgfqpoint{0.000000in}{-0.048611in}}{\pgfqpoint{0.000000in}{0.000000in}}{%
\pgfpathmoveto{\pgfqpoint{0.000000in}{0.000000in}}%
\pgfpathlineto{\pgfqpoint{0.000000in}{-0.048611in}}%
\pgfusepath{stroke,fill}%
}%
\begin{pgfscope}%
\pgfsys@transformshift{0.759533in}{0.515123in}%
\pgfsys@useobject{currentmarker}{}%
\end{pgfscope}%
\end{pgfscope}%
\begin{pgfscope}%
\definecolor{textcolor}{rgb}{0.000000,0.000000,0.000000}%
\pgfsetstrokecolor{textcolor}%
\pgfsetfillcolor{textcolor}%
\pgftext[x=0.759533in,y=0.417901in,,top]{\color{textcolor}\rmfamily\fontsize{10.000000}{12.000000}\selectfont \(\displaystyle 0\)}%
\end{pgfscope}%
\begin{pgfscope}%
\pgfsetbuttcap%
\pgfsetroundjoin%
\definecolor{currentfill}{rgb}{0.000000,0.000000,0.000000}%
\pgfsetfillcolor{currentfill}%
\pgfsetlinewidth{0.803000pt}%
\definecolor{currentstroke}{rgb}{0.000000,0.000000,0.000000}%
\pgfsetstrokecolor{currentstroke}%
\pgfsetdash{}{0pt}%
\pgfsys@defobject{currentmarker}{\pgfqpoint{0.000000in}{-0.048611in}}{\pgfqpoint{0.000000in}{0.000000in}}{%
\pgfpathmoveto{\pgfqpoint{0.000000in}{0.000000in}}%
\pgfpathlineto{\pgfqpoint{0.000000in}{-0.048611in}}%
\pgfusepath{stroke,fill}%
}%
\begin{pgfscope}%
\pgfsys@transformshift{1.452222in}{0.515123in}%
\pgfsys@useobject{currentmarker}{}%
\end{pgfscope}%
\end{pgfscope}%
\begin{pgfscope}%
\definecolor{textcolor}{rgb}{0.000000,0.000000,0.000000}%
\pgfsetstrokecolor{textcolor}%
\pgfsetfillcolor{textcolor}%
\pgftext[x=1.452222in,y=0.417901in,,top]{\color{textcolor}\rmfamily\fontsize{10.000000}{12.000000}\selectfont \(\displaystyle 50\)}%
\end{pgfscope}%
\begin{pgfscope}%
\pgfsetbuttcap%
\pgfsetroundjoin%
\definecolor{currentfill}{rgb}{0.000000,0.000000,0.000000}%
\pgfsetfillcolor{currentfill}%
\pgfsetlinewidth{0.803000pt}%
\definecolor{currentstroke}{rgb}{0.000000,0.000000,0.000000}%
\pgfsetstrokecolor{currentstroke}%
\pgfsetdash{}{0pt}%
\pgfsys@defobject{currentmarker}{\pgfqpoint{0.000000in}{-0.048611in}}{\pgfqpoint{0.000000in}{0.000000in}}{%
\pgfpathmoveto{\pgfqpoint{0.000000in}{0.000000in}}%
\pgfpathlineto{\pgfqpoint{0.000000in}{-0.048611in}}%
\pgfusepath{stroke,fill}%
}%
\begin{pgfscope}%
\pgfsys@transformshift{2.144911in}{0.515123in}%
\pgfsys@useobject{currentmarker}{}%
\end{pgfscope}%
\end{pgfscope}%
\begin{pgfscope}%
\definecolor{textcolor}{rgb}{0.000000,0.000000,0.000000}%
\pgfsetstrokecolor{textcolor}%
\pgfsetfillcolor{textcolor}%
\pgftext[x=2.144911in,y=0.417901in,,top]{\color{textcolor}\rmfamily\fontsize{10.000000}{12.000000}\selectfont \(\displaystyle 100\)}%
\end{pgfscope}%
\begin{pgfscope}%
\pgfsetbuttcap%
\pgfsetroundjoin%
\definecolor{currentfill}{rgb}{0.000000,0.000000,0.000000}%
\pgfsetfillcolor{currentfill}%
\pgfsetlinewidth{0.803000pt}%
\definecolor{currentstroke}{rgb}{0.000000,0.000000,0.000000}%
\pgfsetstrokecolor{currentstroke}%
\pgfsetdash{}{0pt}%
\pgfsys@defobject{currentmarker}{\pgfqpoint{0.000000in}{-0.048611in}}{\pgfqpoint{0.000000in}{0.000000in}}{%
\pgfpathmoveto{\pgfqpoint{0.000000in}{0.000000in}}%
\pgfpathlineto{\pgfqpoint{0.000000in}{-0.048611in}}%
\pgfusepath{stroke,fill}%
}%
\begin{pgfscope}%
\pgfsys@transformshift{2.837600in}{0.515123in}%
\pgfsys@useobject{currentmarker}{}%
\end{pgfscope}%
\end{pgfscope}%
\begin{pgfscope}%
\definecolor{textcolor}{rgb}{0.000000,0.000000,0.000000}%
\pgfsetstrokecolor{textcolor}%
\pgfsetfillcolor{textcolor}%
\pgftext[x=2.837600in,y=0.417901in,,top]{\color{textcolor}\rmfamily\fontsize{10.000000}{12.000000}\selectfont \(\displaystyle 150\)}%
\end{pgfscope}%
\begin{pgfscope}%
\pgfsetbuttcap%
\pgfsetroundjoin%
\definecolor{currentfill}{rgb}{0.000000,0.000000,0.000000}%
\pgfsetfillcolor{currentfill}%
\pgfsetlinewidth{0.803000pt}%
\definecolor{currentstroke}{rgb}{0.000000,0.000000,0.000000}%
\pgfsetstrokecolor{currentstroke}%
\pgfsetdash{}{0pt}%
\pgfsys@defobject{currentmarker}{\pgfqpoint{0.000000in}{-0.048611in}}{\pgfqpoint{0.000000in}{0.000000in}}{%
\pgfpathmoveto{\pgfqpoint{0.000000in}{0.000000in}}%
\pgfpathlineto{\pgfqpoint{0.000000in}{-0.048611in}}%
\pgfusepath{stroke,fill}%
}%
\begin{pgfscope}%
\pgfsys@transformshift{3.530288in}{0.515123in}%
\pgfsys@useobject{currentmarker}{}%
\end{pgfscope}%
\end{pgfscope}%
\begin{pgfscope}%
\definecolor{textcolor}{rgb}{0.000000,0.000000,0.000000}%
\pgfsetstrokecolor{textcolor}%
\pgfsetfillcolor{textcolor}%
\pgftext[x=3.530288in,y=0.417901in,,top]{\color{textcolor}\rmfamily\fontsize{10.000000}{12.000000}\selectfont \(\displaystyle 200\)}%
\end{pgfscope}%
\begin{pgfscope}%
\pgfsetbuttcap%
\pgfsetroundjoin%
\definecolor{currentfill}{rgb}{0.000000,0.000000,0.000000}%
\pgfsetfillcolor{currentfill}%
\pgfsetlinewidth{0.803000pt}%
\definecolor{currentstroke}{rgb}{0.000000,0.000000,0.000000}%
\pgfsetstrokecolor{currentstroke}%
\pgfsetdash{}{0pt}%
\pgfsys@defobject{currentmarker}{\pgfqpoint{0.000000in}{-0.048611in}}{\pgfqpoint{0.000000in}{0.000000in}}{%
\pgfpathmoveto{\pgfqpoint{0.000000in}{0.000000in}}%
\pgfpathlineto{\pgfqpoint{0.000000in}{-0.048611in}}%
\pgfusepath{stroke,fill}%
}%
\begin{pgfscope}%
\pgfsys@transformshift{4.222977in}{0.515123in}%
\pgfsys@useobject{currentmarker}{}%
\end{pgfscope}%
\end{pgfscope}%
\begin{pgfscope}%
\definecolor{textcolor}{rgb}{0.000000,0.000000,0.000000}%
\pgfsetstrokecolor{textcolor}%
\pgfsetfillcolor{textcolor}%
\pgftext[x=4.222977in,y=0.417901in,,top]{\color{textcolor}\rmfamily\fontsize{10.000000}{12.000000}\selectfont \(\displaystyle 250\)}%
\end{pgfscope}%
\begin{pgfscope}%
\pgfsetbuttcap%
\pgfsetroundjoin%
\definecolor{currentfill}{rgb}{0.000000,0.000000,0.000000}%
\pgfsetfillcolor{currentfill}%
\pgfsetlinewidth{0.803000pt}%
\definecolor{currentstroke}{rgb}{0.000000,0.000000,0.000000}%
\pgfsetstrokecolor{currentstroke}%
\pgfsetdash{}{0pt}%
\pgfsys@defobject{currentmarker}{\pgfqpoint{0.000000in}{-0.048611in}}{\pgfqpoint{0.000000in}{0.000000in}}{%
\pgfpathmoveto{\pgfqpoint{0.000000in}{0.000000in}}%
\pgfpathlineto{\pgfqpoint{0.000000in}{-0.048611in}}%
\pgfusepath{stroke,fill}%
}%
\begin{pgfscope}%
\pgfsys@transformshift{4.915666in}{0.515123in}%
\pgfsys@useobject{currentmarker}{}%
\end{pgfscope}%
\end{pgfscope}%
\begin{pgfscope}%
\definecolor{textcolor}{rgb}{0.000000,0.000000,0.000000}%
\pgfsetstrokecolor{textcolor}%
\pgfsetfillcolor{textcolor}%
\pgftext[x=4.915666in,y=0.417901in,,top]{\color{textcolor}\rmfamily\fontsize{10.000000}{12.000000}\selectfont \(\displaystyle 300\)}%
\end{pgfscope}%
\begin{pgfscope}%
\definecolor{textcolor}{rgb}{0.000000,0.000000,0.000000}%
\pgfsetstrokecolor{textcolor}%
\pgfsetfillcolor{textcolor}%
\pgftext[x=3.049306in,y=0.238889in,,top]{\color{textcolor}\rmfamily\fontsize{10.000000}{12.000000}\selectfont Time (seconds)}%
\end{pgfscope}%
\begin{pgfscope}%
\pgfsetbuttcap%
\pgfsetroundjoin%
\definecolor{currentfill}{rgb}{0.000000,0.000000,0.000000}%
\pgfsetfillcolor{currentfill}%
\pgfsetlinewidth{0.803000pt}%
\definecolor{currentstroke}{rgb}{0.000000,0.000000,0.000000}%
\pgfsetstrokecolor{currentstroke}%
\pgfsetdash{}{0pt}%
\pgfsys@defobject{currentmarker}{\pgfqpoint{-0.048611in}{0.000000in}}{\pgfqpoint{0.000000in}{0.000000in}}{%
\pgfpathmoveto{\pgfqpoint{0.000000in}{0.000000in}}%
\pgfpathlineto{\pgfqpoint{-0.048611in}{0.000000in}}%
\pgfusepath{stroke,fill}%
}%
\begin{pgfscope}%
\pgfsys@transformshift{0.530556in}{0.667186in}%
\pgfsys@useobject{currentmarker}{}%
\end{pgfscope}%
\end{pgfscope}%
\begin{pgfscope}%
\definecolor{textcolor}{rgb}{0.000000,0.000000,0.000000}%
\pgfsetstrokecolor{textcolor}%
\pgfsetfillcolor{textcolor}%
\pgftext[x=0.294444in,y=0.618961in,left,base]{\color{textcolor}\rmfamily\fontsize{10.000000}{12.000000}\selectfont \(\displaystyle 16\)}%
\end{pgfscope}%
\begin{pgfscope}%
\pgfsetbuttcap%
\pgfsetroundjoin%
\definecolor{currentfill}{rgb}{0.000000,0.000000,0.000000}%
\pgfsetfillcolor{currentfill}%
\pgfsetlinewidth{0.803000pt}%
\definecolor{currentstroke}{rgb}{0.000000,0.000000,0.000000}%
\pgfsetstrokecolor{currentstroke}%
\pgfsetdash{}{0pt}%
\pgfsys@defobject{currentmarker}{\pgfqpoint{-0.048611in}{0.000000in}}{\pgfqpoint{0.000000in}{0.000000in}}{%
\pgfpathmoveto{\pgfqpoint{0.000000in}{0.000000in}}%
\pgfpathlineto{\pgfqpoint{-0.048611in}{0.000000in}}%
\pgfusepath{stroke,fill}%
}%
\begin{pgfscope}%
\pgfsys@transformshift{0.530556in}{1.059405in}%
\pgfsys@useobject{currentmarker}{}%
\end{pgfscope}%
\end{pgfscope}%
\begin{pgfscope}%
\definecolor{textcolor}{rgb}{0.000000,0.000000,0.000000}%
\pgfsetstrokecolor{textcolor}%
\pgfsetfillcolor{textcolor}%
\pgftext[x=0.294444in,y=1.011180in,left,base]{\color{textcolor}\rmfamily\fontsize{10.000000}{12.000000}\selectfont \(\displaystyle 18\)}%
\end{pgfscope}%
\begin{pgfscope}%
\pgfsetbuttcap%
\pgfsetroundjoin%
\definecolor{currentfill}{rgb}{0.000000,0.000000,0.000000}%
\pgfsetfillcolor{currentfill}%
\pgfsetlinewidth{0.803000pt}%
\definecolor{currentstroke}{rgb}{0.000000,0.000000,0.000000}%
\pgfsetstrokecolor{currentstroke}%
\pgfsetdash{}{0pt}%
\pgfsys@defobject{currentmarker}{\pgfqpoint{-0.048611in}{0.000000in}}{\pgfqpoint{0.000000in}{0.000000in}}{%
\pgfpathmoveto{\pgfqpoint{0.000000in}{0.000000in}}%
\pgfpathlineto{\pgfqpoint{-0.048611in}{0.000000in}}%
\pgfusepath{stroke,fill}%
}%
\begin{pgfscope}%
\pgfsys@transformshift{0.530556in}{1.451624in}%
\pgfsys@useobject{currentmarker}{}%
\end{pgfscope}%
\end{pgfscope}%
\begin{pgfscope}%
\definecolor{textcolor}{rgb}{0.000000,0.000000,0.000000}%
\pgfsetstrokecolor{textcolor}%
\pgfsetfillcolor{textcolor}%
\pgftext[x=0.294444in,y=1.403399in,left,base]{\color{textcolor}\rmfamily\fontsize{10.000000}{12.000000}\selectfont \(\displaystyle 20\)}%
\end{pgfscope}%
\begin{pgfscope}%
\pgfsetbuttcap%
\pgfsetroundjoin%
\definecolor{currentfill}{rgb}{0.000000,0.000000,0.000000}%
\pgfsetfillcolor{currentfill}%
\pgfsetlinewidth{0.803000pt}%
\definecolor{currentstroke}{rgb}{0.000000,0.000000,0.000000}%
\pgfsetstrokecolor{currentstroke}%
\pgfsetdash{}{0pt}%
\pgfsys@defobject{currentmarker}{\pgfqpoint{-0.048611in}{0.000000in}}{\pgfqpoint{0.000000in}{0.000000in}}{%
\pgfpathmoveto{\pgfqpoint{0.000000in}{0.000000in}}%
\pgfpathlineto{\pgfqpoint{-0.048611in}{0.000000in}}%
\pgfusepath{stroke,fill}%
}%
\begin{pgfscope}%
\pgfsys@transformshift{0.530556in}{1.843844in}%
\pgfsys@useobject{currentmarker}{}%
\end{pgfscope}%
\end{pgfscope}%
\begin{pgfscope}%
\definecolor{textcolor}{rgb}{0.000000,0.000000,0.000000}%
\pgfsetstrokecolor{textcolor}%
\pgfsetfillcolor{textcolor}%
\pgftext[x=0.294444in,y=1.795618in,left,base]{\color{textcolor}\rmfamily\fontsize{10.000000}{12.000000}\selectfont \(\displaystyle 22\)}%
\end{pgfscope}%
\begin{pgfscope}%
\pgfsetbuttcap%
\pgfsetroundjoin%
\definecolor{currentfill}{rgb}{0.000000,0.000000,0.000000}%
\pgfsetfillcolor{currentfill}%
\pgfsetlinewidth{0.803000pt}%
\definecolor{currentstroke}{rgb}{0.000000,0.000000,0.000000}%
\pgfsetstrokecolor{currentstroke}%
\pgfsetdash{}{0pt}%
\pgfsys@defobject{currentmarker}{\pgfqpoint{-0.048611in}{0.000000in}}{\pgfqpoint{0.000000in}{0.000000in}}{%
\pgfpathmoveto{\pgfqpoint{0.000000in}{0.000000in}}%
\pgfpathlineto{\pgfqpoint{-0.048611in}{0.000000in}}%
\pgfusepath{stroke,fill}%
}%
\begin{pgfscope}%
\pgfsys@transformshift{0.530556in}{2.236063in}%
\pgfsys@useobject{currentmarker}{}%
\end{pgfscope}%
\end{pgfscope}%
\begin{pgfscope}%
\definecolor{textcolor}{rgb}{0.000000,0.000000,0.000000}%
\pgfsetstrokecolor{textcolor}%
\pgfsetfillcolor{textcolor}%
\pgftext[x=0.294444in,y=2.187837in,left,base]{\color{textcolor}\rmfamily\fontsize{10.000000}{12.000000}\selectfont \(\displaystyle 24\)}%
\end{pgfscope}%
\begin{pgfscope}%
\pgfsetbuttcap%
\pgfsetroundjoin%
\definecolor{currentfill}{rgb}{0.000000,0.000000,0.000000}%
\pgfsetfillcolor{currentfill}%
\pgfsetlinewidth{0.803000pt}%
\definecolor{currentstroke}{rgb}{0.000000,0.000000,0.000000}%
\pgfsetstrokecolor{currentstroke}%
\pgfsetdash{}{0pt}%
\pgfsys@defobject{currentmarker}{\pgfqpoint{-0.048611in}{0.000000in}}{\pgfqpoint{0.000000in}{0.000000in}}{%
\pgfpathmoveto{\pgfqpoint{0.000000in}{0.000000in}}%
\pgfpathlineto{\pgfqpoint{-0.048611in}{0.000000in}}%
\pgfusepath{stroke,fill}%
}%
\begin{pgfscope}%
\pgfsys@transformshift{0.530556in}{2.628282in}%
\pgfsys@useobject{currentmarker}{}%
\end{pgfscope}%
\end{pgfscope}%
\begin{pgfscope}%
\definecolor{textcolor}{rgb}{0.000000,0.000000,0.000000}%
\pgfsetstrokecolor{textcolor}%
\pgfsetfillcolor{textcolor}%
\pgftext[x=0.294444in,y=2.580057in,left,base]{\color{textcolor}\rmfamily\fontsize{10.000000}{12.000000}\selectfont \(\displaystyle 26\)}%
\end{pgfscope}%
\begin{pgfscope}%
\definecolor{textcolor}{rgb}{0.000000,0.000000,0.000000}%
\pgfsetstrokecolor{textcolor}%
\pgfsetfillcolor{textcolor}%
\pgftext[x=0.238889in,y=1.670123in,,bottom,rotate=90.000000]{\color{textcolor}\rmfamily\fontsize{10.000000}{12.000000}\selectfont Temperature (C)}%
\end{pgfscope}%
\begin{pgfscope}%
\pgfpathrectangle{\pgfqpoint{0.530556in}{0.515123in}}{\pgfqpoint{5.037500in}{2.310000in}}%
\pgfusepath{clip}%
\pgfsetrectcap%
\pgfsetroundjoin%
\pgfsetlinewidth{1.505625pt}%
\definecolor{currentstroke}{rgb}{1.000000,0.000000,0.000000}%
\pgfsetstrokecolor{currentstroke}%
\pgfsetdash{}{0pt}%
\pgfpathmoveto{\pgfqpoint{0.759533in}{1.657870in}}%
\pgfpathlineto{\pgfqpoint{0.767984in}{1.684275in}}%
\pgfpathlineto{\pgfqpoint{0.773387in}{1.732322in}}%
\pgfpathlineto{\pgfqpoint{0.781616in}{1.775604in}}%
\pgfpathlineto{\pgfqpoint{0.787241in}{1.797105in}}%
\pgfpathlineto{\pgfqpoint{0.794168in}{1.818514in}}%
\pgfpathlineto{\pgfqpoint{0.802397in}{1.827052in}}%
\pgfpathlineto{\pgfqpoint{0.809324in}{1.827052in}}%
\pgfpathlineto{\pgfqpoint{0.814948in}{1.399754in}}%
\pgfpathlineto{\pgfqpoint{0.821875in}{0.620123in}}%
\pgfpathlineto{\pgfqpoint{0.830104in}{1.541799in}}%
\pgfpathlineto{\pgfqpoint{0.835729in}{1.766978in}}%
\pgfpathlineto{\pgfqpoint{0.842878in}{1.546314in}}%
\pgfpathlineto{\pgfqpoint{0.850885in}{1.801394in}}%
\pgfpathlineto{\pgfqpoint{0.857812in}{1.805680in}}%
\pgfpathlineto{\pgfqpoint{0.863437in}{1.805680in}}%
\pgfpathlineto{\pgfqpoint{0.871887in}{1.814240in}}%
\pgfpathlineto{\pgfqpoint{0.884217in}{1.814240in}}%
\pgfpathlineto{\pgfqpoint{0.891366in}{1.818514in}}%
\pgfpathlineto{\pgfqpoint{0.906300in}{1.818514in}}%
\pgfpathlineto{\pgfqpoint{0.911925in}{1.814240in}}%
\pgfpathlineto{\pgfqpoint{0.946781in}{1.814240in}}%
\pgfpathlineto{\pgfqpoint{0.953708in}{1.818514in}}%
\pgfpathlineto{\pgfqpoint{0.981194in}{1.818514in}}%
\pgfpathlineto{\pgfqpoint{0.995269in}{1.809961in}}%
\pgfpathlineto{\pgfqpoint{1.002196in}{1.809961in}}%
\pgfpathlineto{\pgfqpoint{1.009345in}{1.814240in}}%
\pgfpathlineto{\pgfqpoint{1.071465in}{1.814240in}}%
\pgfpathlineto{\pgfqpoint{1.078613in}{1.818514in}}%
\pgfpathlineto{\pgfqpoint{1.154809in}{1.818514in}}%
\pgfpathlineto{\pgfqpoint{1.161736in}{1.822785in}}%
\pgfpathlineto{\pgfqpoint{1.210224in}{1.822785in}}%
\pgfpathlineto{\pgfqpoint{1.217151in}{1.818514in}}%
\pgfpathlineto{\pgfqpoint{1.231005in}{1.818514in}}%
\pgfpathlineto{\pgfqpoint{1.237932in}{1.822785in}}%
\pgfpathlineto{\pgfqpoint{1.245067in}{1.818514in}}%
\pgfpathlineto{\pgfqpoint{1.300482in}{1.818514in}}%
\pgfpathlineto{\pgfqpoint{1.307201in}{1.822785in}}%
\pgfpathlineto{\pgfqpoint{1.314128in}{1.818514in}}%
\pgfpathlineto{\pgfqpoint{1.321262in}{1.818514in}}%
\pgfpathlineto{\pgfqpoint{1.328189in}{1.822785in}}%
\pgfpathlineto{\pgfqpoint{1.404385in}{1.822785in}}%
\pgfpathlineto{\pgfqpoint{1.411533in}{1.818514in}}%
\pgfpathlineto{\pgfqpoint{1.418239in}{1.822785in}}%
\pgfpathlineto{\pgfqpoint{1.425166in}{1.822785in}}%
\pgfpathlineto{\pgfqpoint{1.432092in}{1.827052in}}%
\pgfpathlineto{\pgfqpoint{1.439241in}{1.822785in}}%
\pgfpathlineto{\pgfqpoint{1.445946in}{1.827052in}}%
\pgfpathlineto{\pgfqpoint{1.452873in}{1.822785in}}%
\pgfpathlineto{\pgfqpoint{1.460022in}{1.827052in}}%
\pgfpathlineto{\pgfqpoint{1.563925in}{1.827052in}}%
\pgfpathlineto{\pgfqpoint{1.570852in}{1.822785in}}%
\pgfpathlineto{\pgfqpoint{1.584706in}{1.822785in}}%
\pgfpathlineto{\pgfqpoint{1.591633in}{1.818514in}}%
\pgfpathlineto{\pgfqpoint{1.605486in}{1.818514in}}%
\pgfpathlineto{\pgfqpoint{1.612413in}{1.822785in}}%
\pgfpathlineto{\pgfqpoint{1.619340in}{1.822785in}}%
\pgfpathlineto{\pgfqpoint{1.626489in}{1.818514in}}%
\pgfpathlineto{\pgfqpoint{1.702684in}{1.818514in}}%
\pgfpathlineto{\pgfqpoint{1.709611in}{1.822785in}}%
\pgfpathlineto{\pgfqpoint{1.716317in}{1.818514in}}%
\pgfpathlineto{\pgfqpoint{1.723465in}{1.822785in}}%
\pgfpathlineto{\pgfqpoint{1.744246in}{1.822785in}}%
\pgfpathlineto{\pgfqpoint{1.751173in}{1.818514in}}%
\pgfpathlineto{\pgfqpoint{1.758099in}{1.818514in}}%
\pgfpathlineto{\pgfqpoint{1.764805in}{1.822785in}}%
\pgfpathlineto{\pgfqpoint{1.771953in}{1.818514in}}%
\pgfpathlineto{\pgfqpoint{1.792734in}{1.818514in}}%
\pgfpathlineto{\pgfqpoint{1.799661in}{1.822785in}}%
\pgfpathlineto{\pgfqpoint{1.820441in}{1.822785in}}%
\pgfpathlineto{\pgfqpoint{1.827368in}{1.818514in}}%
\pgfpathlineto{\pgfqpoint{1.834295in}{1.822785in}}%
\pgfpathlineto{\pgfqpoint{1.841430in}{1.822785in}}%
\pgfpathlineto{\pgfqpoint{1.848149in}{1.818514in}}%
\pgfpathlineto{\pgfqpoint{1.855076in}{1.822785in}}%
\pgfpathlineto{\pgfqpoint{1.882783in}{1.822785in}}%
\pgfpathlineto{\pgfqpoint{1.888616in}{1.818514in}}%
\pgfpathlineto{\pgfqpoint{1.903564in}{1.818514in}}%
\pgfpathlineto{\pgfqpoint{1.910699in}{1.822785in}}%
\pgfpathlineto{\pgfqpoint{1.929969in}{1.822785in}}%
\pgfpathlineto{\pgfqpoint{1.938406in}{1.818514in}}%
\pgfpathlineto{\pgfqpoint{1.945333in}{1.822785in}}%
\pgfpathlineto{\pgfqpoint{1.952052in}{1.818514in}}%
\pgfpathlineto{\pgfqpoint{1.959187in}{1.818514in}}%
\pgfpathlineto{\pgfqpoint{1.966114in}{1.822785in}}%
\pgfpathlineto{\pgfqpoint{2.070017in}{1.822785in}}%
\pgfpathlineto{\pgfqpoint{2.075642in}{1.818514in}}%
\pgfpathlineto{\pgfqpoint{2.082569in}{1.822785in}}%
\pgfpathlineto{\pgfqpoint{2.200548in}{1.822785in}}%
\pgfpathlineto{\pgfqpoint{2.207474in}{1.818514in}}%
\pgfpathlineto{\pgfqpoint{2.241887in}{1.818514in}}%
\pgfpathlineto{\pgfqpoint{2.249036in}{1.814240in}}%
\pgfpathlineto{\pgfqpoint{2.255963in}{1.818514in}}%
\pgfpathlineto{\pgfqpoint{2.270038in}{1.818514in}}%
\pgfpathlineto{\pgfqpoint{2.276743in}{1.814240in}}%
\pgfpathlineto{\pgfqpoint{2.283670in}{1.818514in}}%
\pgfpathlineto{\pgfqpoint{2.290597in}{1.818514in}}%
\pgfpathlineto{\pgfqpoint{2.297524in}{1.814240in}}%
\pgfpathlineto{\pgfqpoint{2.304451in}{1.818514in}}%
\pgfpathlineto{\pgfqpoint{2.318526in}{1.818514in}}%
\pgfpathlineto{\pgfqpoint{2.325231in}{1.814240in}}%
\pgfpathlineto{\pgfqpoint{2.332158in}{1.818514in}}%
\pgfpathlineto{\pgfqpoint{2.339307in}{1.814240in}}%
\pgfpathlineto{\pgfqpoint{2.367014in}{1.814240in}}%
\pgfpathlineto{\pgfqpoint{2.373941in}{1.809961in}}%
\pgfpathlineto{\pgfqpoint{2.380647in}{1.809961in}}%
\pgfpathlineto{\pgfqpoint{2.387795in}{1.814240in}}%
\pgfpathlineto{\pgfqpoint{2.394722in}{1.809961in}}%
\pgfpathlineto{\pgfqpoint{2.457272in}{1.809961in}}%
\pgfpathlineto{\pgfqpoint{2.463991in}{1.814240in}}%
\pgfpathlineto{\pgfqpoint{2.519406in}{1.814240in}}%
\pgfpathlineto{\pgfqpoint{2.526333in}{1.809961in}}%
\pgfpathlineto{\pgfqpoint{2.554248in}{1.809961in}}%
\pgfpathlineto{\pgfqpoint{2.561175in}{1.814240in}}%
\pgfpathlineto{\pgfqpoint{2.567894in}{1.809961in}}%
\pgfpathlineto{\pgfqpoint{2.574821in}{1.831315in}}%
\pgfpathlineto{\pgfqpoint{2.581956in}{2.088822in}}%
\pgfpathlineto{\pgfqpoint{2.588883in}{2.455821in}}%
\pgfpathlineto{\pgfqpoint{2.595602in}{2.600668in}}%
\pgfpathlineto{\pgfqpoint{2.602736in}{2.662551in}}%
\pgfpathlineto{\pgfqpoint{2.609663in}{2.702201in}}%
\pgfpathlineto{\pgfqpoint{2.616590in}{2.702201in}}%
\pgfpathlineto{\pgfqpoint{2.623517in}{2.712962in}}%
\pgfpathlineto{\pgfqpoint{2.630444in}{2.720123in}}%
\pgfpathlineto{\pgfqpoint{2.637371in}{2.716544in}}%
\pgfpathlineto{\pgfqpoint{2.651225in}{2.702201in}}%
\pgfpathlineto{\pgfqpoint{2.658152in}{2.687818in}}%
\pgfpathlineto{\pgfqpoint{2.665078in}{2.684216in}}%
\pgfpathlineto{\pgfqpoint{2.672227in}{2.691418in}}%
\pgfpathlineto{\pgfqpoint{2.685859in}{2.691418in}}%
\pgfpathlineto{\pgfqpoint{2.692786in}{2.687818in}}%
\pgfpathlineto{\pgfqpoint{2.762055in}{2.687818in}}%
\pgfpathlineto{\pgfqpoint{2.769203in}{2.684216in}}%
\pgfpathlineto{\pgfqpoint{2.776130in}{2.687818in}}%
\pgfpathlineto{\pgfqpoint{2.782836in}{2.684216in}}%
\pgfpathlineto{\pgfqpoint{2.845399in}{2.684216in}}%
\pgfpathlineto{\pgfqpoint{2.852326in}{2.680612in}}%
\pgfpathlineto{\pgfqpoint{2.859253in}{2.684216in}}%
\pgfpathlineto{\pgfqpoint{2.873107in}{2.684216in}}%
\pgfpathlineto{\pgfqpoint{2.880034in}{2.680612in}}%
\pgfpathlineto{\pgfqpoint{2.887182in}{2.684216in}}%
\pgfpathlineto{\pgfqpoint{2.893887in}{2.684216in}}%
\pgfpathlineto{\pgfqpoint{2.900814in}{2.680612in}}%
\pgfpathlineto{\pgfqpoint{2.907963in}{2.680612in}}%
\pgfpathlineto{\pgfqpoint{2.914890in}{2.684216in}}%
\pgfpathlineto{\pgfqpoint{2.921595in}{2.684216in}}%
\pgfpathlineto{\pgfqpoint{2.928522in}{2.680612in}}%
\pgfpathlineto{\pgfqpoint{2.984159in}{2.680612in}}%
\pgfpathlineto{\pgfqpoint{2.991085in}{2.677005in}}%
\pgfpathlineto{\pgfqpoint{2.997791in}{2.680612in}}%
\pgfpathlineto{\pgfqpoint{3.004939in}{2.680612in}}%
\pgfpathlineto{\pgfqpoint{3.011866in}{2.677005in}}%
\pgfpathlineto{\pgfqpoint{3.018793in}{2.680612in}}%
\pgfpathlineto{\pgfqpoint{3.039574in}{2.680612in}}%
\pgfpathlineto{\pgfqpoint{3.046501in}{2.677005in}}%
\pgfpathlineto{\pgfqpoint{3.129831in}{2.677005in}}%
\pgfpathlineto{\pgfqpoint{3.136550in}{2.673395in}}%
\pgfpathlineto{\pgfqpoint{3.143477in}{2.677005in}}%
\pgfpathlineto{\pgfqpoint{3.150612in}{2.673395in}}%
\pgfpathlineto{\pgfqpoint{3.157539in}{2.673395in}}%
\pgfpathlineto{\pgfqpoint{3.164258in}{2.677005in}}%
\pgfpathlineto{\pgfqpoint{3.171392in}{2.673395in}}%
\pgfpathlineto{\pgfqpoint{3.226807in}{2.673395in}}%
\pgfpathlineto{\pgfqpoint{3.233734in}{2.669783in}}%
\pgfpathlineto{\pgfqpoint{3.253213in}{2.669783in}}%
\pgfpathlineto{\pgfqpoint{3.261442in}{2.666168in}}%
\pgfpathlineto{\pgfqpoint{3.268369in}{2.669783in}}%
\pgfpathlineto{\pgfqpoint{3.287847in}{2.669783in}}%
\pgfpathlineto{\pgfqpoint{3.296076in}{2.666168in}}%
\pgfpathlineto{\pgfqpoint{3.303003in}{2.669783in}}%
\pgfpathlineto{\pgfqpoint{3.308628in}{2.666168in}}%
\pgfpathlineto{\pgfqpoint{3.343484in}{2.666168in}}%
\pgfpathlineto{\pgfqpoint{3.351491in}{2.662551in}}%
\pgfpathlineto{\pgfqpoint{3.357116in}{2.666168in}}%
\pgfpathlineto{\pgfqpoint{3.370970in}{2.666168in}}%
\pgfpathlineto{\pgfqpoint{3.379199in}{2.662551in}}%
\pgfpathlineto{\pgfqpoint{3.384824in}{2.666168in}}%
\pgfpathlineto{\pgfqpoint{3.398677in}{2.666168in}}%
\pgfpathlineto{\pgfqpoint{3.405604in}{2.662551in}}%
\pgfpathlineto{\pgfqpoint{3.509729in}{2.662551in}}%
\pgfpathlineto{\pgfqpoint{3.516656in}{2.658931in}}%
\pgfpathlineto{\pgfqpoint{3.523583in}{2.662551in}}%
\pgfpathlineto{\pgfqpoint{3.537437in}{2.662551in}}%
\pgfpathlineto{\pgfqpoint{3.544364in}{2.658931in}}%
\pgfpathlineto{\pgfqpoint{3.572071in}{2.658931in}}%
\pgfpathlineto{\pgfqpoint{3.579220in}{2.662551in}}%
\pgfpathlineto{\pgfqpoint{3.586147in}{2.658931in}}%
\pgfpathlineto{\pgfqpoint{3.634635in}{2.658931in}}%
\pgfpathlineto{\pgfqpoint{3.641340in}{2.662551in}}%
\pgfpathlineto{\pgfqpoint{3.648489in}{2.658931in}}%
\pgfpathlineto{\pgfqpoint{3.717965in}{2.658931in}}%
\pgfpathlineto{\pgfqpoint{3.724684in}{2.655309in}}%
\pgfpathlineto{\pgfqpoint{3.835722in}{2.655309in}}%
\pgfpathlineto{\pgfqpoint{3.842649in}{2.651684in}}%
\pgfpathlineto{\pgfqpoint{3.849576in}{2.651684in}}%
\pgfpathlineto{\pgfqpoint{3.856503in}{2.655309in}}%
\pgfpathlineto{\pgfqpoint{3.863430in}{2.655309in}}%
\pgfpathlineto{\pgfqpoint{3.870357in}{2.651684in}}%
\pgfpathlineto{\pgfqpoint{3.898064in}{2.651684in}}%
\pgfpathlineto{\pgfqpoint{3.904991in}{2.655309in}}%
\pgfpathlineto{\pgfqpoint{3.911918in}{2.651684in}}%
\pgfpathlineto{\pgfqpoint{3.918845in}{2.651684in}}%
\pgfpathlineto{\pgfqpoint{3.925772in}{2.648057in}}%
\pgfpathlineto{\pgfqpoint{3.932921in}{2.651684in}}%
\pgfpathlineto{\pgfqpoint{3.939626in}{2.648057in}}%
\pgfpathlineto{\pgfqpoint{3.946553in}{2.651684in}}%
\pgfpathlineto{\pgfqpoint{3.953480in}{2.648057in}}%
\pgfpathlineto{\pgfqpoint{4.064531in}{2.648057in}}%
\pgfpathlineto{\pgfqpoint{4.071458in}{2.644427in}}%
\pgfpathlineto{\pgfqpoint{4.078385in}{2.648057in}}%
\pgfpathlineto{\pgfqpoint{4.085312in}{2.644427in}}%
\pgfpathlineto{\pgfqpoint{4.161508in}{2.644427in}}%
\pgfpathlineto{\pgfqpoint{4.168656in}{2.640795in}}%
\pgfpathlineto{\pgfqpoint{4.314121in}{2.640795in}}%
\pgfpathlineto{\pgfqpoint{4.321048in}{2.644427in}}%
\pgfpathlineto{\pgfqpoint{4.327975in}{2.640795in}}%
\pgfpathlineto{\pgfqpoint{4.362817in}{2.640795in}}%
\pgfpathlineto{\pgfqpoint{4.369536in}{2.637160in}}%
\pgfpathlineto{\pgfqpoint{4.383598in}{2.637160in}}%
\pgfpathlineto{\pgfqpoint{4.390525in}{2.640795in}}%
\pgfpathlineto{\pgfqpoint{4.424951in}{2.640795in}}%
\pgfpathlineto{\pgfqpoint{4.432086in}{2.637160in}}%
\pgfpathlineto{\pgfqpoint{4.439013in}{2.640795in}}%
\pgfpathlineto{\pgfqpoint{4.452867in}{2.640795in}}%
\pgfpathlineto{\pgfqpoint{4.459793in}{2.637160in}}%
\pgfpathlineto{\pgfqpoint{4.466720in}{2.637160in}}%
\pgfpathlineto{\pgfqpoint{4.473647in}{2.633522in}}%
\pgfpathlineto{\pgfqpoint{4.480574in}{2.637160in}}%
\pgfpathlineto{\pgfqpoint{4.487501in}{2.637160in}}%
\pgfpathlineto{\pgfqpoint{4.494428in}{2.633522in}}%
\pgfpathlineto{\pgfqpoint{4.501576in}{2.637160in}}%
\pgfpathlineto{\pgfqpoint{4.522135in}{2.637160in}}%
\pgfpathlineto{\pgfqpoint{4.527760in}{2.633522in}}%
\pgfpathlineto{\pgfqpoint{4.555468in}{2.633522in}}%
\pgfpathlineto{\pgfqpoint{4.563697in}{2.629882in}}%
\pgfpathlineto{\pgfqpoint{4.569321in}{2.633522in}}%
\pgfpathlineto{\pgfqpoint{4.584477in}{2.626239in}}%
\pgfpathlineto{\pgfqpoint{4.604177in}{2.626239in}}%
\pgfpathlineto{\pgfqpoint{4.612185in}{2.629882in}}%
\pgfpathlineto{\pgfqpoint{4.652666in}{2.629882in}}%
\pgfpathlineto{\pgfqpoint{4.659593in}{2.633522in}}%
\pgfpathlineto{\pgfqpoint{4.666298in}{2.629882in}}%
\pgfpathlineto{\pgfqpoint{4.673446in}{2.629882in}}%
\pgfpathlineto{\pgfqpoint{4.680373in}{2.633522in}}%
\pgfpathlineto{\pgfqpoint{4.805057in}{2.633522in}}%
\pgfpathlineto{\pgfqpoint{4.812206in}{2.629882in}}%
\pgfpathlineto{\pgfqpoint{4.819133in}{2.633522in}}%
\pgfpathlineto{\pgfqpoint{4.825838in}{2.633522in}}%
\pgfpathlineto{\pgfqpoint{4.832765in}{2.629882in}}%
\pgfpathlineto{\pgfqpoint{4.839913in}{2.629882in}}%
\pgfpathlineto{\pgfqpoint{4.846840in}{2.626239in}}%
\pgfpathlineto{\pgfqpoint{4.853545in}{2.626239in}}%
\pgfpathlineto{\pgfqpoint{4.860694in}{2.629882in}}%
\pgfpathlineto{\pgfqpoint{4.867621in}{2.626239in}}%
\pgfpathlineto{\pgfqpoint{4.874548in}{2.629882in}}%
\pgfpathlineto{\pgfqpoint{4.881253in}{2.626239in}}%
\pgfpathlineto{\pgfqpoint{4.909182in}{2.626239in}}%
\pgfpathlineto{\pgfqpoint{4.916109in}{2.622594in}}%
\pgfpathlineto{\pgfqpoint{4.923036in}{2.622594in}}%
\pgfpathlineto{\pgfqpoint{4.929963in}{2.626239in}}%
\pgfpathlineto{\pgfqpoint{4.936890in}{2.622594in}}%
\pgfpathlineto{\pgfqpoint{4.950744in}{2.622594in}}%
\pgfpathlineto{\pgfqpoint{4.957670in}{2.618946in}}%
\pgfpathlineto{\pgfqpoint{4.964597in}{2.618946in}}%
\pgfpathlineto{\pgfqpoint{4.971524in}{2.622594in}}%
\pgfpathlineto{\pgfqpoint{4.978659in}{2.578648in}}%
\pgfpathlineto{\pgfqpoint{4.992305in}{2.593339in}}%
\pgfpathlineto{\pgfqpoint{4.999232in}{2.604329in}}%
\pgfpathlineto{\pgfqpoint{5.006366in}{2.607987in}}%
\pgfpathlineto{\pgfqpoint{5.013086in}{2.607987in}}%
\pgfpathlineto{\pgfqpoint{5.020012in}{2.604329in}}%
\pgfpathlineto{\pgfqpoint{5.034074in}{2.604329in}}%
\pgfpathlineto{\pgfqpoint{5.054855in}{2.615296in}}%
\pgfpathlineto{\pgfqpoint{5.061782in}{2.615296in}}%
\pgfpathlineto{\pgfqpoint{5.068501in}{2.622594in}}%
\pgfpathlineto{\pgfqpoint{5.075635in}{2.618946in}}%
\pgfpathlineto{\pgfqpoint{5.103343in}{2.618946in}}%
\pgfpathlineto{\pgfqpoint{5.110270in}{2.622594in}}%
\pgfpathlineto{\pgfqpoint{5.117197in}{2.618946in}}%
\pgfpathlineto{\pgfqpoint{5.131050in}{2.618946in}}%
\pgfpathlineto{\pgfqpoint{5.137977in}{2.622594in}}%
\pgfpathlineto{\pgfqpoint{5.145126in}{2.618946in}}%
\pgfpathlineto{\pgfqpoint{5.179539in}{2.618946in}}%
\pgfpathlineto{\pgfqpoint{5.186466in}{2.545440in}}%
\pgfpathlineto{\pgfqpoint{5.193614in}{2.425596in}}%
\pgfpathlineto{\pgfqpoint{5.200319in}{2.333834in}}%
\pgfpathlineto{\pgfqpoint{5.214173in}{2.240388in}}%
\pgfpathlineto{\pgfqpoint{5.221322in}{2.149194in}}%
\pgfpathlineto{\pgfqpoint{5.228027in}{2.165172in}}%
\pgfpathlineto{\pgfqpoint{5.234954in}{2.368438in}}%
\pgfpathlineto{\pgfqpoint{5.242102in}{2.500827in}}%
\pgfpathlineto{\pgfqpoint{5.249029in}{2.541737in}}%
\pgfpathlineto{\pgfqpoint{5.255734in}{2.556534in}}%
\pgfpathlineto{\pgfqpoint{5.262883in}{2.563916in}}%
\pgfpathlineto{\pgfqpoint{5.269810in}{2.560226in}}%
\pgfpathlineto{\pgfqpoint{5.276737in}{2.560226in}}%
\pgfpathlineto{\pgfqpoint{5.297517in}{2.571287in}}%
\pgfpathlineto{\pgfqpoint{5.304444in}{2.567603in}}%
\pgfpathlineto{\pgfqpoint{5.311371in}{2.571287in}}%
\pgfpathlineto{\pgfqpoint{5.318298in}{2.571287in}}%
\pgfpathlineto{\pgfqpoint{5.332152in}{2.578648in}}%
\pgfpathlineto{\pgfqpoint{5.339079in}{2.585999in}}%
\pgfpathlineto{\pgfqpoint{5.339079in}{2.585999in}}%
\pgfusepath{stroke}%
\end{pgfscope}%
\begin{pgfscope}%
\pgfsetrectcap%
\pgfsetmiterjoin%
\pgfsetlinewidth{0.803000pt}%
\definecolor{currentstroke}{rgb}{0.000000,0.000000,0.000000}%
\pgfsetstrokecolor{currentstroke}%
\pgfsetdash{}{0pt}%
\pgfpathmoveto{\pgfqpoint{0.530556in}{0.515123in}}%
\pgfpathlineto{\pgfqpoint{0.530556in}{2.825123in}}%
\pgfusepath{stroke}%
\end{pgfscope}%
\begin{pgfscope}%
\pgfsetrectcap%
\pgfsetmiterjoin%
\pgfsetlinewidth{0.803000pt}%
\definecolor{currentstroke}{rgb}{0.000000,0.000000,0.000000}%
\pgfsetstrokecolor{currentstroke}%
\pgfsetdash{}{0pt}%
\pgfpathmoveto{\pgfqpoint{5.568056in}{0.515123in}}%
\pgfpathlineto{\pgfqpoint{5.568056in}{2.825123in}}%
\pgfusepath{stroke}%
\end{pgfscope}%
\begin{pgfscope}%
\pgfsetrectcap%
\pgfsetmiterjoin%
\pgfsetlinewidth{0.803000pt}%
\definecolor{currentstroke}{rgb}{0.000000,0.000000,0.000000}%
\pgfsetstrokecolor{currentstroke}%
\pgfsetdash{}{0pt}%
\pgfpathmoveto{\pgfqpoint{0.530556in}{0.515123in}}%
\pgfpathlineto{\pgfqpoint{5.568056in}{0.515123in}}%
\pgfusepath{stroke}%
\end{pgfscope}%
\begin{pgfscope}%
\pgfsetrectcap%
\pgfsetmiterjoin%
\pgfsetlinewidth{0.803000pt}%
\definecolor{currentstroke}{rgb}{0.000000,0.000000,0.000000}%
\pgfsetstrokecolor{currentstroke}%
\pgfsetdash{}{0pt}%
\pgfpathmoveto{\pgfqpoint{0.530556in}{2.825123in}}%
\pgfpathlineto{\pgfqpoint{5.568056in}{2.825123in}}%
\pgfusepath{stroke}%
\end{pgfscope}%
\begin{pgfscope}%
\definecolor{textcolor}{rgb}{0.000000,0.000000,0.000000}%
\pgfsetstrokecolor{textcolor}%
\pgfsetfillcolor{textcolor}%
\pgftext[x=3.049306in,y=2.908457in,,base]{\color{textcolor}\rmfamily\fontsize{12.000000}{14.400000}\selectfont Part III, KOH(aq) + HCl(aq)\(\displaystyle \to\) KCl(aq) + H\(\displaystyle _2\)O(l)}%
\end{pgfscope}%
\end{pgfpicture}%
\makeatother%
\endgroup%

    \end{center}
    \caption{Temperature profile of reaction, when both reactants start as aqueous solutions.
    		Larger dips are result of experimental operator error:
		the thermistor was partially removed from the solution momentarily}\label{fig:partIIIb}
\end{figure}

\begin{table}[H]
	\doublespacing
	\centering
	\caption{Data collected from all six experiment groups, including some calculations} \label{tab:main}
	\begin{tabular}{rrrrrrr}%
	\hline
                                     & Group 1 & Group 2 & Group 3 & Group 4 & Group 5 & Group 6 \\\hline
\textbf{Part I}                      &         &         &         &         &         & \\
Mass \koh (g)                        & 2.991   & 2.230   & 3.332   & 2.643   & 3.090   & 2.327 \\
Moles \koh                           & 0.05331 & 0.03975 & 0.05939 & 0.04711 & 0.05508 & 0.04148\\
Solution Volume (mL)                 & 100.0   & 100.0   & 98.7    & 100.0   & 100.0   & 100.0 \\
$\Delta T$ (\textdegree C)           & 4.644   & 3.680   & 5.123   & 5.042   & 6.160   & 3.498 \\
Heat (J)                             & 1943.   & 1539.   & 2120.   & 2110.   & 2577.   & 1464. \\
Heat (J) per mol \koh (J)            & 36450.  & 38740.  & 35620.  & 44780.  & 46800.  & 35290. \\\hline
\textbf{Part II}                     &         &         &         &         &         & \\
Mass \koh (g)                        & 2.228   & 2.388   & 2.708   & 2.058   & 2.515   & 1.838 \\
Moles \koh                           & 0.0397  & 0.04256 & 0.04827 & 0.03668 & 0.04483 & 0.03276 \\
Moles \hcl                           & 0.150   & 0.150   & 0.148   & 0.150   & 0.150   & 0.150 \\
Solution Volume (mL)                 & 100.0   & 100.0   & 98.5    & 99.95   & 100.0   & 100.0 \\
pH test                              & acidic  & acidic  & acidic  & acidic  & acidic  & acidic \\
Limiting reagent                     & \koh    & \koh    & \koh    & \koh    & \koh    & \koh \\
$\Delta T$ (\textdegree C)           & 8.650   & 8.888   & 10.526  & 8.364   & 10.6015 & 6.922 \\
Heat (J)                             & 3620    & 3720    & 4340    & 3498    & 4440    & $2.90\times10^3$ \\
Heat (J) per mol \koh                & 91100   & 87400   & 89876   & 95355   & 98952   & 88405 \\\hline
\textbf{Part III}                    &         &         &         &         &         & \\
Mass \koh (g)                        & 1.496   & 1.115   & 1.673   & 1.3215  & 1.545   & 1.1635 \\
Moles \koh                           & 0.02666 & 0.01987 & 0.02982 & 0.02355 & 0.02754 & 0.02074\\
Moles \hcl                           & 0.075   & 0.075   & 0.0753  & 0.0738  & 0.075   & 0.075 \\
Final solution volume (mL)           & 100.0   & 100.0   & 99.9    & 98.5    & 100.0   & 100.0 \\
pH test                              & acidic  & acidic  & acidic  & acidic  & acidic  & acidic \\
Limiting reagent                     & \koh    & \koh    & \koh    & \koh    & \koh    & \koh \\
$\Delta T$ (\textdegree C)           & 3.874   & 2.777   & 4.199   & 3.831   & 5.417   & 3.262 \\
Heat (J)                             & 1621.   & 1162.   & 1755.   & 1579.   & 2266.   & 1365. \\
Heat (J) per mol \koh                & 21610   & 15490   & 23310   & 21390   & 30220   & $1.820\times10^3$ \\\hline

\end{tabular}

\end{table}
\begin{figure}[H]
	\begin{center}
		%% Creator: Matplotlib, PGF backend
%%
%% To include the figure in your LaTeX document, write
%%   \input{<filename>.pgf}
%%
%% Make sure the required packages are loaded in your preamble
%%   \usepackage{pgf}
%%
%% Figures using additional raster images can only be included by \input if
%% they are in the same directory as the main LaTeX file. For loading figures
%% from other directories you can use the `import` package
%%   \usepackage{import}
%% and then include the figures with
%%   \import{<path to file>}{<filename>.pgf}
%%
%% Matplotlib used the following preamble
%%
\begingroup%
\makeatletter%
\begin{pgfpicture}%
\pgfpathrectangle{\pgfpointorigin}{\pgfqpoint{5.612501in}{2.730679in}}%
\pgfusepath{use as bounding box, clip}%
\begin{pgfscope}%
\pgfsetbuttcap%
\pgfsetmiterjoin%
\definecolor{currentfill}{rgb}{1.000000,1.000000,1.000000}%
\pgfsetfillcolor{currentfill}%
\pgfsetlinewidth{0.000000pt}%
\definecolor{currentstroke}{rgb}{1.000000,1.000000,1.000000}%
\pgfsetstrokecolor{currentstroke}%
\pgfsetdash{}{0pt}%
\pgfpathmoveto{\pgfqpoint{0.000000in}{0.000000in}}%
\pgfpathlineto{\pgfqpoint{5.612501in}{0.000000in}}%
\pgfpathlineto{\pgfqpoint{5.612501in}{2.730679in}}%
\pgfpathlineto{\pgfqpoint{0.000000in}{2.730679in}}%
\pgfpathclose%
\pgfusepath{fill}%
\end{pgfscope}%
\begin{pgfscope}%
\pgfsetbuttcap%
\pgfsetmiterjoin%
\definecolor{currentfill}{rgb}{1.000000,1.000000,1.000000}%
\pgfsetfillcolor{currentfill}%
\pgfsetlinewidth{0.000000pt}%
\definecolor{currentstroke}{rgb}{0.000000,0.000000,0.000000}%
\pgfsetstrokecolor{currentstroke}%
\pgfsetstrokeopacity{0.000000}%
\pgfsetdash{}{0pt}%
\pgfpathmoveto{\pgfqpoint{0.475001in}{0.320679in}}%
\pgfpathlineto{\pgfqpoint{5.512501in}{0.320679in}}%
\pgfpathlineto{\pgfqpoint{5.512501in}{2.630679in}}%
\pgfpathlineto{\pgfqpoint{0.475001in}{2.630679in}}%
\pgfpathclose%
\pgfusepath{fill}%
\end{pgfscope}%
\begin{pgfscope}%
\pgfpathrectangle{\pgfqpoint{0.475001in}{0.320679in}}{\pgfqpoint{5.037500in}{2.310000in}}%
\pgfusepath{clip}%
\pgfsetbuttcap%
\pgfsetroundjoin%
\definecolor{currentfill}{rgb}{0.121569,0.466667,0.705882}%
\pgfsetfillcolor{currentfill}%
\pgfsetlinewidth{1.003750pt}%
\definecolor{currentstroke}{rgb}{0.121569,0.466667,0.705882}%
\pgfsetstrokecolor{currentstroke}%
\pgfsetdash{}{0pt}%
\pgfpathmoveto{\pgfqpoint{3.097439in}{1.944039in}}%
\pgfpathcurveto{\pgfqpoint{3.108489in}{1.944039in}}{\pgfqpoint{3.119088in}{1.948429in}}{\pgfqpoint{3.126901in}{1.956243in}}%
\pgfpathcurveto{\pgfqpoint{3.134715in}{1.964056in}}{\pgfqpoint{3.139105in}{1.974655in}}{\pgfqpoint{3.139105in}{1.985705in}}%
\pgfpathcurveto{\pgfqpoint{3.139105in}{1.996755in}}{\pgfqpoint{3.134715in}{2.007354in}}{\pgfqpoint{3.126901in}{2.015168in}}%
\pgfpathcurveto{\pgfqpoint{3.119088in}{2.022982in}}{\pgfqpoint{3.108489in}{2.027372in}}{\pgfqpoint{3.097439in}{2.027372in}}%
\pgfpathcurveto{\pgfqpoint{3.086389in}{2.027372in}}{\pgfqpoint{3.075789in}{2.022982in}}{\pgfqpoint{3.067976in}{2.015168in}}%
\pgfpathcurveto{\pgfqpoint{3.060162in}{2.007354in}}{\pgfqpoint{3.055772in}{1.996755in}}{\pgfqpoint{3.055772in}{1.985705in}}%
\pgfpathcurveto{\pgfqpoint{3.055772in}{1.974655in}}{\pgfqpoint{3.060162in}{1.964056in}}{\pgfqpoint{3.067976in}{1.956243in}}%
\pgfpathcurveto{\pgfqpoint{3.075789in}{1.948429in}}{\pgfqpoint{3.086389in}{1.944039in}}{\pgfqpoint{3.097439in}{1.944039in}}%
\pgfpathclose%
\pgfusepath{stroke,fill}%
\end{pgfscope}%
\begin{pgfscope}%
\pgfpathrectangle{\pgfqpoint{0.475001in}{0.320679in}}{\pgfqpoint{5.037500in}{2.310000in}}%
\pgfusepath{clip}%
\pgfsetbuttcap%
\pgfsetroundjoin%
\definecolor{currentfill}{rgb}{0.121569,0.466667,0.705882}%
\pgfsetfillcolor{currentfill}%
\pgfsetlinewidth{1.003750pt}%
\definecolor{currentstroke}{rgb}{0.121569,0.466667,0.705882}%
\pgfsetstrokecolor{currentstroke}%
\pgfsetdash{}{0pt}%
\pgfpathmoveto{\pgfqpoint{3.344561in}{1.997219in}}%
\pgfpathcurveto{\pgfqpoint{3.355611in}{1.997219in}}{\pgfqpoint{3.366210in}{2.001609in}}{\pgfqpoint{3.374024in}{2.009422in}}%
\pgfpathcurveto{\pgfqpoint{3.381838in}{2.017236in}}{\pgfqpoint{3.386228in}{2.027835in}}{\pgfqpoint{3.386228in}{2.038885in}}%
\pgfpathcurveto{\pgfqpoint{3.386228in}{2.049935in}}{\pgfqpoint{3.381838in}{2.060534in}}{\pgfqpoint{3.374024in}{2.068348in}}%
\pgfpathcurveto{\pgfqpoint{3.366210in}{2.076162in}}{\pgfqpoint{3.355611in}{2.080552in}}{\pgfqpoint{3.344561in}{2.080552in}}%
\pgfpathcurveto{\pgfqpoint{3.333511in}{2.080552in}}{\pgfqpoint{3.322912in}{2.076162in}}{\pgfqpoint{3.315099in}{2.068348in}}%
\pgfpathcurveto{\pgfqpoint{3.307285in}{2.060534in}}{\pgfqpoint{3.302895in}{2.049935in}}{\pgfqpoint{3.302895in}{2.038885in}}%
\pgfpathcurveto{\pgfqpoint{3.302895in}{2.027835in}}{\pgfqpoint{3.307285in}{2.017236in}}{\pgfqpoint{3.315099in}{2.009422in}}%
\pgfpathcurveto{\pgfqpoint{3.322912in}{2.001609in}}{\pgfqpoint{3.333511in}{1.997219in}}{\pgfqpoint{3.344561in}{1.997219in}}%
\pgfpathclose%
\pgfusepath{stroke,fill}%
\end{pgfscope}%
\begin{pgfscope}%
\pgfpathrectangle{\pgfqpoint{0.475001in}{0.320679in}}{\pgfqpoint{5.037500in}{2.310000in}}%
\pgfusepath{clip}%
\pgfsetbuttcap%
\pgfsetroundjoin%
\definecolor{currentfill}{rgb}{0.121569,0.466667,0.705882}%
\pgfsetfillcolor{currentfill}%
\pgfsetlinewidth{1.003750pt}%
\definecolor{currentstroke}{rgb}{0.121569,0.466667,0.705882}%
\pgfsetstrokecolor{currentstroke}%
\pgfsetdash{}{0pt}%
\pgfpathmoveto{\pgfqpoint{3.837943in}{2.326934in}}%
\pgfpathcurveto{\pgfqpoint{3.848993in}{2.326934in}}{\pgfqpoint{3.859592in}{2.331324in}}{\pgfqpoint{3.867405in}{2.339138in}}%
\pgfpathcurveto{\pgfqpoint{3.875219in}{2.346952in}}{\pgfqpoint{3.879609in}{2.357551in}}{\pgfqpoint{3.879609in}{2.368601in}}%
\pgfpathcurveto{\pgfqpoint{3.879609in}{2.379651in}}{\pgfqpoint{3.875219in}{2.390250in}}{\pgfqpoint{3.867405in}{2.398064in}}%
\pgfpathcurveto{\pgfqpoint{3.859592in}{2.405877in}}{\pgfqpoint{3.848993in}{2.410268in}}{\pgfqpoint{3.837943in}{2.410268in}}%
\pgfpathcurveto{\pgfqpoint{3.826892in}{2.410268in}}{\pgfqpoint{3.816293in}{2.405877in}}{\pgfqpoint{3.808480in}{2.398064in}}%
\pgfpathcurveto{\pgfqpoint{3.800666in}{2.390250in}}{\pgfqpoint{3.796276in}{2.379651in}}{\pgfqpoint{3.796276in}{2.368601in}}%
\pgfpathcurveto{\pgfqpoint{3.796276in}{2.357551in}}{\pgfqpoint{3.800666in}{2.346952in}}{\pgfqpoint{3.808480in}{2.339138in}}%
\pgfpathcurveto{\pgfqpoint{3.816293in}{2.331324in}}{\pgfqpoint{3.826892in}{2.326934in}}{\pgfqpoint{3.837943in}{2.326934in}}%
\pgfpathclose%
\pgfusepath{stroke,fill}%
\end{pgfscope}%
\begin{pgfscope}%
\pgfpathrectangle{\pgfqpoint{0.475001in}{0.320679in}}{\pgfqpoint{5.037500in}{2.310000in}}%
\pgfusepath{clip}%
\pgfsetbuttcap%
\pgfsetroundjoin%
\definecolor{currentfill}{rgb}{0.121569,0.466667,0.705882}%
\pgfsetfillcolor{currentfill}%
\pgfsetlinewidth{1.003750pt}%
\definecolor{currentstroke}{rgb}{0.121569,0.466667,0.705882}%
\pgfsetstrokecolor{currentstroke}%
\pgfsetdash{}{0pt}%
\pgfpathmoveto{\pgfqpoint{2.836491in}{1.879159in}}%
\pgfpathcurveto{\pgfqpoint{2.847541in}{1.879159in}}{\pgfqpoint{2.858140in}{1.883549in}}{\pgfqpoint{2.865954in}{1.891363in}}%
\pgfpathcurveto{\pgfqpoint{2.873767in}{1.899177in}}{\pgfqpoint{2.878158in}{1.909776in}}{\pgfqpoint{2.878158in}{1.920826in}}%
\pgfpathcurveto{\pgfqpoint{2.878158in}{1.931876in}}{\pgfqpoint{2.873767in}{1.942475in}}{\pgfqpoint{2.865954in}{1.950289in}}%
\pgfpathcurveto{\pgfqpoint{2.858140in}{1.958102in}}{\pgfqpoint{2.847541in}{1.962492in}}{\pgfqpoint{2.836491in}{1.962492in}}%
\pgfpathcurveto{\pgfqpoint{2.825441in}{1.962492in}}{\pgfqpoint{2.814842in}{1.958102in}}{\pgfqpoint{2.807028in}{1.950289in}}%
\pgfpathcurveto{\pgfqpoint{2.799215in}{1.942475in}}{\pgfqpoint{2.794824in}{1.931876in}}{\pgfqpoint{2.794824in}{1.920826in}}%
\pgfpathcurveto{\pgfqpoint{2.794824in}{1.909776in}}{\pgfqpoint{2.799215in}{1.899177in}}{\pgfqpoint{2.807028in}{1.891363in}}%
\pgfpathcurveto{\pgfqpoint{2.814842in}{1.883549in}}{\pgfqpoint{2.825441in}{1.879159in}}{\pgfqpoint{2.836491in}{1.879159in}}%
\pgfpathclose%
\pgfusepath{stroke,fill}%
\end{pgfscope}%
\begin{pgfscope}%
\pgfpathrectangle{\pgfqpoint{0.475001in}{0.320679in}}{\pgfqpoint{5.037500in}{2.310000in}}%
\pgfusepath{clip}%
\pgfsetbuttcap%
\pgfsetroundjoin%
\definecolor{currentfill}{rgb}{0.121569,0.466667,0.705882}%
\pgfsetfillcolor{currentfill}%
\pgfsetlinewidth{1.003750pt}%
\definecolor{currentstroke}{rgb}{0.121569,0.466667,0.705882}%
\pgfsetstrokecolor{currentstroke}%
\pgfsetdash{}{0pt}%
\pgfpathmoveto{\pgfqpoint{3.540704in}{2.380114in}}%
\pgfpathcurveto{\pgfqpoint{3.551754in}{2.380114in}}{\pgfqpoint{3.562353in}{2.384504in}}{\pgfqpoint{3.570167in}{2.392318in}}%
\pgfpathcurveto{\pgfqpoint{3.577980in}{2.400132in}}{\pgfqpoint{3.582371in}{2.410731in}}{\pgfqpoint{3.582371in}{2.421781in}}%
\pgfpathcurveto{\pgfqpoint{3.582371in}{2.432831in}}{\pgfqpoint{3.577980in}{2.443430in}}{\pgfqpoint{3.570167in}{2.451244in}}%
\pgfpathcurveto{\pgfqpoint{3.562353in}{2.459057in}}{\pgfqpoint{3.551754in}{2.463448in}}{\pgfqpoint{3.540704in}{2.463448in}}%
\pgfpathcurveto{\pgfqpoint{3.529654in}{2.463448in}}{\pgfqpoint{3.519055in}{2.459057in}}{\pgfqpoint{3.511241in}{2.451244in}}%
\pgfpathcurveto{\pgfqpoint{3.503428in}{2.443430in}}{\pgfqpoint{3.499037in}{2.432831in}}{\pgfqpoint{3.499037in}{2.421781in}}%
\pgfpathcurveto{\pgfqpoint{3.499037in}{2.410731in}}{\pgfqpoint{3.503428in}{2.400132in}}{\pgfqpoint{3.511241in}{2.392318in}}%
\pgfpathcurveto{\pgfqpoint{3.519055in}{2.384504in}}{\pgfqpoint{3.529654in}{2.380114in}}{\pgfqpoint{3.540704in}{2.380114in}}%
\pgfpathclose%
\pgfusepath{stroke,fill}%
\end{pgfscope}%
\begin{pgfscope}%
\pgfpathrectangle{\pgfqpoint{0.475001in}{0.320679in}}{\pgfqpoint{5.037500in}{2.310000in}}%
\pgfusepath{clip}%
\pgfsetbuttcap%
\pgfsetroundjoin%
\definecolor{currentfill}{rgb}{0.121569,0.466667,0.705882}%
\pgfsetfillcolor{currentfill}%
\pgfsetlinewidth{1.003750pt}%
\definecolor{currentstroke}{rgb}{0.121569,0.466667,0.705882}%
\pgfsetstrokecolor{currentstroke}%
\pgfsetdash{}{0pt}%
\pgfpathmoveto{\pgfqpoint{2.497777in}{1.561143in}}%
\pgfpathcurveto{\pgfqpoint{2.508828in}{1.561143in}}{\pgfqpoint{2.519427in}{1.565533in}}{\pgfqpoint{2.527240in}{1.573347in}}%
\pgfpathcurveto{\pgfqpoint{2.535054in}{1.581161in}}{\pgfqpoint{2.539444in}{1.591760in}}{\pgfqpoint{2.539444in}{1.602810in}}%
\pgfpathcurveto{\pgfqpoint{2.539444in}{1.613860in}}{\pgfqpoint{2.535054in}{1.624459in}}{\pgfqpoint{2.527240in}{1.632273in}}%
\pgfpathcurveto{\pgfqpoint{2.519427in}{1.640086in}}{\pgfqpoint{2.508828in}{1.644476in}}{\pgfqpoint{2.497777in}{1.644476in}}%
\pgfpathcurveto{\pgfqpoint{2.486727in}{1.644476in}}{\pgfqpoint{2.476128in}{1.640086in}}{\pgfqpoint{2.468315in}{1.632273in}}%
\pgfpathcurveto{\pgfqpoint{2.460501in}{1.624459in}}{\pgfqpoint{2.456111in}{1.613860in}}{\pgfqpoint{2.456111in}{1.602810in}}%
\pgfpathcurveto{\pgfqpoint{2.456111in}{1.591760in}}{\pgfqpoint{2.460501in}{1.581161in}}{\pgfqpoint{2.468315in}{1.573347in}}%
\pgfpathcurveto{\pgfqpoint{2.476128in}{1.565533in}}{\pgfqpoint{2.486727in}{1.561143in}}{\pgfqpoint{2.497777in}{1.561143in}}%
\pgfpathclose%
\pgfusepath{stroke,fill}%
\end{pgfscope}%
\begin{pgfscope}%
\pgfpathrectangle{\pgfqpoint{0.475001in}{0.320679in}}{\pgfqpoint{5.037500in}{2.310000in}}%
\pgfusepath{clip}%
\pgfsetbuttcap%
\pgfsetroundjoin%
\definecolor{currentfill}{rgb}{1.000000,0.498039,0.054902}%
\pgfsetfillcolor{currentfill}%
\pgfsetlinewidth{1.003750pt}%
\definecolor{currentstroke}{rgb}{1.000000,0.498039,0.054902}%
\pgfsetstrokecolor{currentstroke}%
\pgfsetdash{}{0pt}%
\pgfpathmoveto{\pgfqpoint{4.273431in}{1.052211in}}%
\pgfpathcurveto{\pgfqpoint{4.284481in}{1.052211in}}{\pgfqpoint{4.295081in}{1.056601in}}{\pgfqpoint{4.302894in}{1.064415in}}%
\pgfpathcurveto{\pgfqpoint{4.310708in}{1.072229in}}{\pgfqpoint{4.315098in}{1.082828in}}{\pgfqpoint{4.315098in}{1.093878in}}%
\pgfpathcurveto{\pgfqpoint{4.315098in}{1.104928in}}{\pgfqpoint{4.310708in}{1.115527in}}{\pgfqpoint{4.302894in}{1.123340in}}%
\pgfpathcurveto{\pgfqpoint{4.295081in}{1.131154in}}{\pgfqpoint{4.284481in}{1.135544in}}{\pgfqpoint{4.273431in}{1.135544in}}%
\pgfpathcurveto{\pgfqpoint{4.262381in}{1.135544in}}{\pgfqpoint{4.251782in}{1.131154in}}{\pgfqpoint{4.243969in}{1.123340in}}%
\pgfpathcurveto{\pgfqpoint{4.236155in}{1.115527in}}{\pgfqpoint{4.231765in}{1.104928in}}{\pgfqpoint{4.231765in}{1.093878in}}%
\pgfpathcurveto{\pgfqpoint{4.231765in}{1.082828in}}{\pgfqpoint{4.236155in}{1.072229in}}{\pgfqpoint{4.243969in}{1.064415in}}%
\pgfpathcurveto{\pgfqpoint{4.251782in}{1.056601in}}{\pgfqpoint{4.262381in}{1.052211in}}{\pgfqpoint{4.273431in}{1.052211in}}%
\pgfpathclose%
\pgfusepath{stroke,fill}%
\end{pgfscope}%
\begin{pgfscope}%
\pgfpathrectangle{\pgfqpoint{0.475001in}{0.320679in}}{\pgfqpoint{5.037500in}{2.310000in}}%
\pgfusepath{clip}%
\pgfsetbuttcap%
\pgfsetroundjoin%
\definecolor{currentfill}{rgb}{1.000000,0.498039,0.054902}%
\pgfsetfillcolor{currentfill}%
\pgfsetlinewidth{1.003750pt}%
\definecolor{currentstroke}{rgb}{1.000000,0.498039,0.054902}%
\pgfsetstrokecolor{currentstroke}%
\pgfsetdash{}{0pt}%
\pgfpathmoveto{\pgfqpoint{3.101759in}{0.837364in}}%
\pgfpathcurveto{\pgfqpoint{3.112809in}{0.837364in}}{\pgfqpoint{3.123408in}{0.841754in}}{\pgfqpoint{3.131222in}{0.849568in}}%
\pgfpathcurveto{\pgfqpoint{3.139035in}{0.857382in}}{\pgfqpoint{3.143426in}{0.867981in}}{\pgfqpoint{3.143426in}{0.879031in}}%
\pgfpathcurveto{\pgfqpoint{3.143426in}{0.890081in}}{\pgfqpoint{3.139035in}{0.900680in}}{\pgfqpoint{3.131222in}{0.908493in}}%
\pgfpathcurveto{\pgfqpoint{3.123408in}{0.916307in}}{\pgfqpoint{3.112809in}{0.920697in}}{\pgfqpoint{3.101759in}{0.920697in}}%
\pgfpathcurveto{\pgfqpoint{3.090709in}{0.920697in}}{\pgfqpoint{3.080110in}{0.916307in}}{\pgfqpoint{3.072296in}{0.908493in}}%
\pgfpathcurveto{\pgfqpoint{3.064483in}{0.900680in}}{\pgfqpoint{3.060092in}{0.890081in}}{\pgfqpoint{3.060092in}{0.879031in}}%
\pgfpathcurveto{\pgfqpoint{3.060092in}{0.867981in}}{\pgfqpoint{3.064483in}{0.857382in}}{\pgfqpoint{3.072296in}{0.849568in}}%
\pgfpathcurveto{\pgfqpoint{3.080110in}{0.841754in}}{\pgfqpoint{3.090709in}{0.837364in}}{\pgfqpoint{3.101759in}{0.837364in}}%
\pgfpathclose%
\pgfusepath{stroke,fill}%
\end{pgfscope}%
\begin{pgfscope}%
\pgfpathrectangle{\pgfqpoint{0.475001in}{0.320679in}}{\pgfqpoint{5.037500in}{2.310000in}}%
\pgfusepath{clip}%
\pgfsetbuttcap%
\pgfsetroundjoin%
\definecolor{currentfill}{rgb}{1.000000,0.498039,0.054902}%
\pgfsetfillcolor{currentfill}%
\pgfsetlinewidth{1.003750pt}%
\definecolor{currentstroke}{rgb}{1.000000,0.498039,0.054902}%
\pgfsetstrokecolor{currentstroke}%
\pgfsetdash{}{0pt}%
\pgfpathmoveto{\pgfqpoint{4.798783in}{1.146339in}}%
\pgfpathcurveto{\pgfqpoint{4.809833in}{1.146339in}}{\pgfqpoint{4.820432in}{1.150730in}}{\pgfqpoint{4.828246in}{1.158543in}}%
\pgfpathcurveto{\pgfqpoint{4.836059in}{1.166357in}}{\pgfqpoint{4.840450in}{1.176956in}}{\pgfqpoint{4.840450in}{1.188006in}}%
\pgfpathcurveto{\pgfqpoint{4.840450in}{1.199056in}}{\pgfqpoint{4.836059in}{1.209655in}}{\pgfqpoint{4.828246in}{1.217469in}}%
\pgfpathcurveto{\pgfqpoint{4.820432in}{1.225283in}}{\pgfqpoint{4.809833in}{1.229673in}}{\pgfqpoint{4.798783in}{1.229673in}}%
\pgfpathcurveto{\pgfqpoint{4.787733in}{1.229673in}}{\pgfqpoint{4.777134in}{1.225283in}}{\pgfqpoint{4.769320in}{1.217469in}}%
\pgfpathcurveto{\pgfqpoint{4.761507in}{1.209655in}}{\pgfqpoint{4.757116in}{1.199056in}}{\pgfqpoint{4.757116in}{1.188006in}}%
\pgfpathcurveto{\pgfqpoint{4.757116in}{1.176956in}}{\pgfqpoint{4.761507in}{1.166357in}}{\pgfqpoint{4.769320in}{1.158543in}}%
\pgfpathcurveto{\pgfqpoint{4.777134in}{1.150730in}}{\pgfqpoint{4.787733in}{1.146339in}}{\pgfqpoint{4.798783in}{1.146339in}}%
\pgfpathclose%
\pgfusepath{stroke,fill}%
\end{pgfscope}%
\begin{pgfscope}%
\pgfpathrectangle{\pgfqpoint{0.475001in}{0.320679in}}{\pgfqpoint{5.037500in}{2.310000in}}%
\pgfusepath{clip}%
\pgfsetbuttcap%
\pgfsetroundjoin%
\definecolor{currentfill}{rgb}{1.000000,0.498039,0.054902}%
\pgfsetfillcolor{currentfill}%
\pgfsetlinewidth{1.003750pt}%
\definecolor{currentstroke}{rgb}{1.000000,0.498039,0.054902}%
\pgfsetstrokecolor{currentstroke}%
\pgfsetdash{}{0pt}%
\pgfpathmoveto{\pgfqpoint{3.737711in}{1.141022in}}%
\pgfpathcurveto{\pgfqpoint{3.748761in}{1.141022in}}{\pgfqpoint{3.759360in}{1.145412in}}{\pgfqpoint{3.767174in}{1.153225in}}%
\pgfpathcurveto{\pgfqpoint{3.774987in}{1.161039in}}{\pgfqpoint{3.779378in}{1.171638in}}{\pgfqpoint{3.779378in}{1.182688in}}%
\pgfpathcurveto{\pgfqpoint{3.779378in}{1.193738in}}{\pgfqpoint{3.774987in}{1.204337in}}{\pgfqpoint{3.767174in}{1.212151in}}%
\pgfpathcurveto{\pgfqpoint{3.759360in}{1.219965in}}{\pgfqpoint{3.748761in}{1.224355in}}{\pgfqpoint{3.737711in}{1.224355in}}%
\pgfpathcurveto{\pgfqpoint{3.726661in}{1.224355in}}{\pgfqpoint{3.716062in}{1.219965in}}{\pgfqpoint{3.708248in}{1.212151in}}%
\pgfpathcurveto{\pgfqpoint{3.700435in}{1.204337in}}{\pgfqpoint{3.696044in}{1.193738in}}{\pgfqpoint{3.696044in}{1.182688in}}%
\pgfpathcurveto{\pgfqpoint{3.696044in}{1.171638in}}{\pgfqpoint{3.700435in}{1.161039in}}{\pgfqpoint{3.708248in}{1.153225in}}%
\pgfpathcurveto{\pgfqpoint{3.716062in}{1.145412in}}{\pgfqpoint{3.726661in}{1.141022in}}{\pgfqpoint{3.737711in}{1.141022in}}%
\pgfpathclose%
\pgfusepath{stroke,fill}%
\end{pgfscope}%
\begin{pgfscope}%
\pgfpathrectangle{\pgfqpoint{0.475001in}{0.320679in}}{\pgfqpoint{5.037500in}{2.310000in}}%
\pgfusepath{clip}%
\pgfsetbuttcap%
\pgfsetroundjoin%
\definecolor{currentfill}{rgb}{1.000000,0.498039,0.054902}%
\pgfsetfillcolor{currentfill}%
\pgfsetlinewidth{1.003750pt}%
\definecolor{currentstroke}{rgb}{1.000000,0.498039,0.054902}%
\pgfsetstrokecolor{currentstroke}%
\pgfsetdash{}{0pt}%
\pgfpathmoveto{\pgfqpoint{4.426371in}{1.389372in}}%
\pgfpathcurveto{\pgfqpoint{4.437421in}{1.389372in}}{\pgfqpoint{4.448020in}{1.393762in}}{\pgfqpoint{4.455834in}{1.401576in}}%
\pgfpathcurveto{\pgfqpoint{4.463647in}{1.409389in}}{\pgfqpoint{4.468038in}{1.419988in}}{\pgfqpoint{4.468038in}{1.431039in}}%
\pgfpathcurveto{\pgfqpoint{4.468038in}{1.442089in}}{\pgfqpoint{4.463647in}{1.452688in}}{\pgfqpoint{4.455834in}{1.460501in}}%
\pgfpathcurveto{\pgfqpoint{4.448020in}{1.468315in}}{\pgfqpoint{4.437421in}{1.472705in}}{\pgfqpoint{4.426371in}{1.472705in}}%
\pgfpathcurveto{\pgfqpoint{4.415321in}{1.472705in}}{\pgfqpoint{4.404722in}{1.468315in}}{\pgfqpoint{4.396908in}{1.460501in}}%
\pgfpathcurveto{\pgfqpoint{4.389094in}{1.452688in}}{\pgfqpoint{4.384704in}{1.442089in}}{\pgfqpoint{4.384704in}{1.431039in}}%
\pgfpathcurveto{\pgfqpoint{4.384704in}{1.419988in}}{\pgfqpoint{4.389094in}{1.409389in}}{\pgfqpoint{4.396908in}{1.401576in}}%
\pgfpathcurveto{\pgfqpoint{4.404722in}{1.393762in}}{\pgfqpoint{4.415321in}{1.389372in}}{\pgfqpoint{4.426371in}{1.389372in}}%
\pgfpathclose%
\pgfusepath{stroke,fill}%
\end{pgfscope}%
\begin{pgfscope}%
\pgfpathrectangle{\pgfqpoint{0.475001in}{0.320679in}}{\pgfqpoint{5.037500in}{2.310000in}}%
\pgfusepath{clip}%
\pgfsetbuttcap%
\pgfsetroundjoin%
\definecolor{currentfill}{rgb}{1.000000,0.498039,0.054902}%
\pgfsetfillcolor{currentfill}%
\pgfsetlinewidth{1.003750pt}%
\definecolor{currentstroke}{rgb}{1.000000,0.498039,0.054902}%
\pgfsetstrokecolor{currentstroke}%
\pgfsetdash{}{0pt}%
\pgfpathmoveto{\pgfqpoint{3.251242in}{0.797479in}}%
\pgfpathcurveto{\pgfqpoint{3.262292in}{0.797479in}}{\pgfqpoint{3.272891in}{0.801869in}}{\pgfqpoint{3.280705in}{0.809683in}}%
\pgfpathcurveto{\pgfqpoint{3.288519in}{0.817497in}}{\pgfqpoint{3.292909in}{0.828096in}}{\pgfqpoint{3.292909in}{0.839146in}}%
\pgfpathcurveto{\pgfqpoint{3.292909in}{0.850196in}}{\pgfqpoint{3.288519in}{0.860795in}}{\pgfqpoint{3.280705in}{0.868609in}}%
\pgfpathcurveto{\pgfqpoint{3.272891in}{0.876422in}}{\pgfqpoint{3.262292in}{0.880812in}}{\pgfqpoint{3.251242in}{0.880812in}}%
\pgfpathcurveto{\pgfqpoint{3.240192in}{0.880812in}}{\pgfqpoint{3.229593in}{0.876422in}}{\pgfqpoint{3.221779in}{0.868609in}}%
\pgfpathcurveto{\pgfqpoint{3.213966in}{0.860795in}}{\pgfqpoint{3.209576in}{0.850196in}}{\pgfqpoint{3.209576in}{0.839146in}}%
\pgfpathcurveto{\pgfqpoint{3.209576in}{0.828096in}}{\pgfqpoint{3.213966in}{0.817497in}}{\pgfqpoint{3.221779in}{0.809683in}}%
\pgfpathcurveto{\pgfqpoint{3.229593in}{0.801869in}}{\pgfqpoint{3.240192in}{0.797479in}}{\pgfqpoint{3.251242in}{0.797479in}}%
\pgfpathclose%
\pgfusepath{stroke,fill}%
\end{pgfscope}%
\begin{pgfscope}%
\pgfpathrectangle{\pgfqpoint{0.475001in}{0.320679in}}{\pgfqpoint{5.037500in}{2.310000in}}%
\pgfusepath{clip}%
\pgfsetbuttcap%
\pgfsetroundjoin%
\definecolor{currentfill}{rgb}{0.172549,0.627451,0.172549}%
\pgfsetfillcolor{currentfill}%
\pgfsetlinewidth{1.003750pt}%
\definecolor{currentstroke}{rgb}{0.172549,0.627451,0.172549}%
\pgfsetstrokecolor{currentstroke}%
\pgfsetdash{}{0pt}%
\pgfpathmoveto{\pgfqpoint{1.970698in}{0.880972in}}%
\pgfpathcurveto{\pgfqpoint{1.981748in}{0.880972in}}{\pgfqpoint{1.992347in}{0.885362in}}{\pgfqpoint{2.000160in}{0.893175in}}%
\pgfpathcurveto{\pgfqpoint{2.007974in}{0.900989in}}{\pgfqpoint{2.012364in}{0.911588in}}{\pgfqpoint{2.012364in}{0.922638in}}%
\pgfpathcurveto{\pgfqpoint{2.012364in}{0.933688in}}{\pgfqpoint{2.007974in}{0.944287in}}{\pgfqpoint{2.000160in}{0.952101in}}%
\pgfpathcurveto{\pgfqpoint{1.992347in}{0.959915in}}{\pgfqpoint{1.981748in}{0.964305in}}{\pgfqpoint{1.970698in}{0.964305in}}%
\pgfpathcurveto{\pgfqpoint{1.959648in}{0.964305in}}{\pgfqpoint{1.949048in}{0.959915in}}{\pgfqpoint{1.941235in}{0.952101in}}%
\pgfpathcurveto{\pgfqpoint{1.933421in}{0.944287in}}{\pgfqpoint{1.929031in}{0.933688in}}{\pgfqpoint{1.929031in}{0.922638in}}%
\pgfpathcurveto{\pgfqpoint{1.929031in}{0.911588in}}{\pgfqpoint{1.933421in}{0.900989in}}{\pgfqpoint{1.941235in}{0.893175in}}%
\pgfpathcurveto{\pgfqpoint{1.949048in}{0.885362in}}{\pgfqpoint{1.959648in}{0.880972in}}{\pgfqpoint{1.970698in}{0.880972in}}%
\pgfpathclose%
\pgfusepath{stroke,fill}%
\end{pgfscope}%
\begin{pgfscope}%
\pgfpathrectangle{\pgfqpoint{0.475001in}{0.320679in}}{\pgfqpoint{5.037500in}{2.310000in}}%
\pgfusepath{clip}%
\pgfsetbuttcap%
\pgfsetroundjoin%
\definecolor{currentfill}{rgb}{0.172549,0.627451,0.172549}%
\pgfsetfillcolor{currentfill}%
\pgfsetlinewidth{1.003750pt}%
\definecolor{currentstroke}{rgb}{0.172549,0.627451,0.172549}%
\pgfsetstrokecolor{currentstroke}%
\pgfsetdash{}{0pt}%
\pgfpathmoveto{\pgfqpoint{1.383997in}{0.636876in}}%
\pgfpathcurveto{\pgfqpoint{1.395048in}{0.636876in}}{\pgfqpoint{1.405647in}{0.641266in}}{\pgfqpoint{1.413460in}{0.649080in}}%
\pgfpathcurveto{\pgfqpoint{1.421274in}{0.656893in}}{\pgfqpoint{1.425664in}{0.667492in}}{\pgfqpoint{1.425664in}{0.678542in}}%
\pgfpathcurveto{\pgfqpoint{1.425664in}{0.689592in}}{\pgfqpoint{1.421274in}{0.700191in}}{\pgfqpoint{1.413460in}{0.708005in}}%
\pgfpathcurveto{\pgfqpoint{1.405647in}{0.715819in}}{\pgfqpoint{1.395048in}{0.720209in}}{\pgfqpoint{1.383997in}{0.720209in}}%
\pgfpathcurveto{\pgfqpoint{1.372947in}{0.720209in}}{\pgfqpoint{1.362348in}{0.715819in}}{\pgfqpoint{1.354535in}{0.708005in}}%
\pgfpathcurveto{\pgfqpoint{1.346721in}{0.700191in}}{\pgfqpoint{1.342331in}{0.689592in}}{\pgfqpoint{1.342331in}{0.678542in}}%
\pgfpathcurveto{\pgfqpoint{1.342331in}{0.667492in}}{\pgfqpoint{1.346721in}{0.656893in}}{\pgfqpoint{1.354535in}{0.649080in}}%
\pgfpathcurveto{\pgfqpoint{1.362348in}{0.641266in}}{\pgfqpoint{1.372947in}{0.636876in}}{\pgfqpoint{1.383997in}{0.636876in}}%
\pgfpathclose%
\pgfusepath{stroke,fill}%
\end{pgfscope}%
\begin{pgfscope}%
\pgfpathrectangle{\pgfqpoint{0.475001in}{0.320679in}}{\pgfqpoint{5.037500in}{2.310000in}}%
\pgfusepath{clip}%
\pgfsetbuttcap%
\pgfsetroundjoin%
\definecolor{currentfill}{rgb}{0.172549,0.627451,0.172549}%
\pgfsetfillcolor{currentfill}%
\pgfsetlinewidth{1.003750pt}%
\definecolor{currentstroke}{rgb}{0.172549,0.627451,0.172549}%
\pgfsetstrokecolor{currentstroke}%
\pgfsetdash{}{0pt}%
\pgfpathmoveto{\pgfqpoint{2.243742in}{0.952233in}}%
\pgfpathcurveto{\pgfqpoint{2.254792in}{0.952233in}}{\pgfqpoint{2.265391in}{0.956623in}}{\pgfqpoint{2.273205in}{0.964437in}}%
\pgfpathcurveto{\pgfqpoint{2.281019in}{0.972250in}}{\pgfqpoint{2.285409in}{0.982849in}}{\pgfqpoint{2.285409in}{0.993899in}}%
\pgfpathcurveto{\pgfqpoint{2.285409in}{1.004950in}}{\pgfqpoint{2.281019in}{1.015549in}}{\pgfqpoint{2.273205in}{1.023362in}}%
\pgfpathcurveto{\pgfqpoint{2.265391in}{1.031176in}}{\pgfqpoint{2.254792in}{1.035566in}}{\pgfqpoint{2.243742in}{1.035566in}}%
\pgfpathcurveto{\pgfqpoint{2.232692in}{1.035566in}}{\pgfqpoint{2.222093in}{1.031176in}}{\pgfqpoint{2.214279in}{1.023362in}}%
\pgfpathcurveto{\pgfqpoint{2.206466in}{1.015549in}}{\pgfqpoint{2.202076in}{1.004950in}}{\pgfqpoint{2.202076in}{0.993899in}}%
\pgfpathcurveto{\pgfqpoint{2.202076in}{0.982849in}}{\pgfqpoint{2.206466in}{0.972250in}}{\pgfqpoint{2.214279in}{0.964437in}}%
\pgfpathcurveto{\pgfqpoint{2.222093in}{0.956623in}}{\pgfqpoint{2.232692in}{0.952233in}}{\pgfqpoint{2.243742in}{0.952233in}}%
\pgfpathclose%
\pgfusepath{stroke,fill}%
\end{pgfscope}%
\begin{pgfscope}%
\pgfpathrectangle{\pgfqpoint{0.475001in}{0.320679in}}{\pgfqpoint{5.037500in}{2.310000in}}%
\pgfusepath{clip}%
\pgfsetbuttcap%
\pgfsetroundjoin%
\definecolor{currentfill}{rgb}{0.172549,0.627451,0.172549}%
\pgfsetfillcolor{currentfill}%
\pgfsetlinewidth{1.003750pt}%
\definecolor{currentstroke}{rgb}{0.172549,0.627451,0.172549}%
\pgfsetstrokecolor{currentstroke}%
\pgfsetdash{}{0pt}%
\pgfpathmoveto{\pgfqpoint{1.701973in}{0.858636in}}%
\pgfpathcurveto{\pgfqpoint{1.713024in}{0.858636in}}{\pgfqpoint{1.723623in}{0.863026in}}{\pgfqpoint{1.731436in}{0.870840in}}%
\pgfpathcurveto{\pgfqpoint{1.739250in}{0.878654in}}{\pgfqpoint{1.743640in}{0.889253in}}{\pgfqpoint{1.743640in}{0.900303in}}%
\pgfpathcurveto{\pgfqpoint{1.743640in}{0.911353in}}{\pgfqpoint{1.739250in}{0.921952in}}{\pgfqpoint{1.731436in}{0.929765in}}%
\pgfpathcurveto{\pgfqpoint{1.723623in}{0.937579in}}{\pgfqpoint{1.713024in}{0.941969in}}{\pgfqpoint{1.701973in}{0.941969in}}%
\pgfpathcurveto{\pgfqpoint{1.690923in}{0.941969in}}{\pgfqpoint{1.680324in}{0.937579in}}{\pgfqpoint{1.672511in}{0.929765in}}%
\pgfpathcurveto{\pgfqpoint{1.664697in}{0.921952in}}{\pgfqpoint{1.660307in}{0.911353in}}{\pgfqpoint{1.660307in}{0.900303in}}%
\pgfpathcurveto{\pgfqpoint{1.660307in}{0.889253in}}{\pgfqpoint{1.664697in}{0.878654in}}{\pgfqpoint{1.672511in}{0.870840in}}%
\pgfpathcurveto{\pgfqpoint{1.680324in}{0.863026in}}{\pgfqpoint{1.690923in}{0.858636in}}{\pgfqpoint{1.701973in}{0.858636in}}%
\pgfpathclose%
\pgfusepath{stroke,fill}%
\end{pgfscope}%
\begin{pgfscope}%
\pgfpathrectangle{\pgfqpoint{0.475001in}{0.320679in}}{\pgfqpoint{5.037500in}{2.310000in}}%
\pgfusepath{clip}%
\pgfsetbuttcap%
\pgfsetroundjoin%
\definecolor{currentfill}{rgb}{0.172549,0.627451,0.172549}%
\pgfsetfillcolor{currentfill}%
\pgfsetlinewidth{1.003750pt}%
\definecolor{currentstroke}{rgb}{0.172549,0.627451,0.172549}%
\pgfsetstrokecolor{currentstroke}%
\pgfsetdash{}{0pt}%
\pgfpathmoveto{\pgfqpoint{2.046735in}{1.223982in}}%
\pgfpathcurveto{\pgfqpoint{2.057786in}{1.223982in}}{\pgfqpoint{2.068385in}{1.228372in}}{\pgfqpoint{2.076198in}{1.236186in}}%
\pgfpathcurveto{\pgfqpoint{2.084012in}{1.244000in}}{\pgfqpoint{2.088402in}{1.254599in}}{\pgfqpoint{2.088402in}{1.265649in}}%
\pgfpathcurveto{\pgfqpoint{2.088402in}{1.276699in}}{\pgfqpoint{2.084012in}{1.287298in}}{\pgfqpoint{2.076198in}{1.295112in}}%
\pgfpathcurveto{\pgfqpoint{2.068385in}{1.302925in}}{\pgfqpoint{2.057786in}{1.307316in}}{\pgfqpoint{2.046735in}{1.307316in}}%
\pgfpathcurveto{\pgfqpoint{2.035685in}{1.307316in}}{\pgfqpoint{2.025086in}{1.302925in}}{\pgfqpoint{2.017273in}{1.295112in}}%
\pgfpathcurveto{\pgfqpoint{2.009459in}{1.287298in}}{\pgfqpoint{2.005069in}{1.276699in}}{\pgfqpoint{2.005069in}{1.265649in}}%
\pgfpathcurveto{\pgfqpoint{2.005069in}{1.254599in}}{\pgfqpoint{2.009459in}{1.244000in}}{\pgfqpoint{2.017273in}{1.236186in}}%
\pgfpathcurveto{\pgfqpoint{2.025086in}{1.228372in}}{\pgfqpoint{2.035685in}{1.223982in}}{\pgfqpoint{2.046735in}{1.223982in}}%
\pgfpathclose%
\pgfusepath{stroke,fill}%
\end{pgfscope}%
\begin{pgfscope}%
\pgfpathrectangle{\pgfqpoint{0.475001in}{0.320679in}}{\pgfqpoint{5.037500in}{2.310000in}}%
\pgfusepath{clip}%
\pgfsetbuttcap%
\pgfsetroundjoin%
\definecolor{currentfill}{rgb}{0.172549,0.627451,0.172549}%
\pgfsetfillcolor{currentfill}%
\pgfsetlinewidth{1.003750pt}%
\definecolor{currentstroke}{rgb}{0.172549,0.627451,0.172549}%
\pgfsetstrokecolor{currentstroke}%
\pgfsetdash{}{0pt}%
\pgfpathmoveto{\pgfqpoint{1.459171in}{0.744831in}}%
\pgfpathcurveto{\pgfqpoint{1.470221in}{0.744831in}}{\pgfqpoint{1.480820in}{0.749221in}}{\pgfqpoint{1.488634in}{0.757035in}}%
\pgfpathcurveto{\pgfqpoint{1.496447in}{0.764848in}}{\pgfqpoint{1.500838in}{0.775447in}}{\pgfqpoint{1.500838in}{0.786498in}}%
\pgfpathcurveto{\pgfqpoint{1.500838in}{0.797548in}}{\pgfqpoint{1.496447in}{0.808147in}}{\pgfqpoint{1.488634in}{0.815960in}}%
\pgfpathcurveto{\pgfqpoint{1.480820in}{0.823774in}}{\pgfqpoint{1.470221in}{0.828164in}}{\pgfqpoint{1.459171in}{0.828164in}}%
\pgfpathcurveto{\pgfqpoint{1.448121in}{0.828164in}}{\pgfqpoint{1.437522in}{0.823774in}}{\pgfqpoint{1.429708in}{0.815960in}}%
\pgfpathcurveto{\pgfqpoint{1.421895in}{0.808147in}}{\pgfqpoint{1.417504in}{0.797548in}}{\pgfqpoint{1.417504in}{0.786498in}}%
\pgfpathcurveto{\pgfqpoint{1.417504in}{0.775447in}}{\pgfqpoint{1.421895in}{0.764848in}}{\pgfqpoint{1.429708in}{0.757035in}}%
\pgfpathcurveto{\pgfqpoint{1.437522in}{0.749221in}}{\pgfqpoint{1.448121in}{0.744831in}}{\pgfqpoint{1.459171in}{0.744831in}}%
\pgfpathclose%
\pgfusepath{stroke,fill}%
\end{pgfscope}%
\begin{pgfscope}%
\pgfsetbuttcap%
\pgfsetroundjoin%
\definecolor{currentfill}{rgb}{0.000000,0.000000,0.000000}%
\pgfsetfillcolor{currentfill}%
\pgfsetlinewidth{0.803000pt}%
\definecolor{currentstroke}{rgb}{0.000000,0.000000,0.000000}%
\pgfsetstrokecolor{currentstroke}%
\pgfsetdash{}{0pt}%
\pgfsys@defobject{currentmarker}{\pgfqpoint{0.000000in}{-0.048611in}}{\pgfqpoint{0.000000in}{0.000000in}}{%
\pgfpathmoveto{\pgfqpoint{0.000000in}{0.000000in}}%
\pgfpathlineto{\pgfqpoint{0.000000in}{-0.048611in}}%
\pgfusepath{stroke,fill}%
}%
\begin{pgfscope}%
\pgfsys@transformshift{0.531165in}{0.320679in}%
\pgfsys@useobject{currentmarker}{}%
\end{pgfscope}%
\end{pgfscope}%
\begin{pgfscope}%
\definecolor{textcolor}{rgb}{0.000000,0.000000,0.000000}%
\pgfsetstrokecolor{textcolor}%
\pgfsetfillcolor{textcolor}%
\pgftext[x=0.531165in,y=0.223457in,,top]{\color{textcolor}\rmfamily\fontsize{10.000000}{12.000000}\selectfont \(\displaystyle 0.01\)}%
\end{pgfscope}%
\begin{pgfscope}%
\pgfsetbuttcap%
\pgfsetroundjoin%
\definecolor{currentfill}{rgb}{0.000000,0.000000,0.000000}%
\pgfsetfillcolor{currentfill}%
\pgfsetlinewidth{0.803000pt}%
\definecolor{currentstroke}{rgb}{0.000000,0.000000,0.000000}%
\pgfsetstrokecolor{currentstroke}%
\pgfsetdash{}{0pt}%
\pgfsys@defobject{currentmarker}{\pgfqpoint{0.000000in}{-0.048611in}}{\pgfqpoint{0.000000in}{0.000000in}}{%
\pgfpathmoveto{\pgfqpoint{0.000000in}{0.000000in}}%
\pgfpathlineto{\pgfqpoint{0.000000in}{-0.048611in}}%
\pgfusepath{stroke,fill}%
}%
\begin{pgfscope}%
\pgfsys@transformshift{1.395230in}{0.320679in}%
\pgfsys@useobject{currentmarker}{}%
\end{pgfscope}%
\end{pgfscope}%
\begin{pgfscope}%
\definecolor{textcolor}{rgb}{0.000000,0.000000,0.000000}%
\pgfsetstrokecolor{textcolor}%
\pgfsetfillcolor{textcolor}%
\pgftext[x=1.395230in,y=0.223457in,,top]{\color{textcolor}\rmfamily\fontsize{10.000000}{12.000000}\selectfont \(\displaystyle 0.02\)}%
\end{pgfscope}%
\begin{pgfscope}%
\pgfsetbuttcap%
\pgfsetroundjoin%
\definecolor{currentfill}{rgb}{0.000000,0.000000,0.000000}%
\pgfsetfillcolor{currentfill}%
\pgfsetlinewidth{0.803000pt}%
\definecolor{currentstroke}{rgb}{0.000000,0.000000,0.000000}%
\pgfsetstrokecolor{currentstroke}%
\pgfsetdash{}{0pt}%
\pgfsys@defobject{currentmarker}{\pgfqpoint{0.000000in}{-0.048611in}}{\pgfqpoint{0.000000in}{0.000000in}}{%
\pgfpathmoveto{\pgfqpoint{0.000000in}{0.000000in}}%
\pgfpathlineto{\pgfqpoint{0.000000in}{-0.048611in}}%
\pgfusepath{stroke,fill}%
}%
\begin{pgfscope}%
\pgfsys@transformshift{2.259295in}{0.320679in}%
\pgfsys@useobject{currentmarker}{}%
\end{pgfscope}%
\end{pgfscope}%
\begin{pgfscope}%
\definecolor{textcolor}{rgb}{0.000000,0.000000,0.000000}%
\pgfsetstrokecolor{textcolor}%
\pgfsetfillcolor{textcolor}%
\pgftext[x=2.259295in,y=0.223457in,,top]{\color{textcolor}\rmfamily\fontsize{10.000000}{12.000000}\selectfont \(\displaystyle 0.03\)}%
\end{pgfscope}%
\begin{pgfscope}%
\pgfsetbuttcap%
\pgfsetroundjoin%
\definecolor{currentfill}{rgb}{0.000000,0.000000,0.000000}%
\pgfsetfillcolor{currentfill}%
\pgfsetlinewidth{0.803000pt}%
\definecolor{currentstroke}{rgb}{0.000000,0.000000,0.000000}%
\pgfsetstrokecolor{currentstroke}%
\pgfsetdash{}{0pt}%
\pgfsys@defobject{currentmarker}{\pgfqpoint{0.000000in}{-0.048611in}}{\pgfqpoint{0.000000in}{0.000000in}}{%
\pgfpathmoveto{\pgfqpoint{0.000000in}{0.000000in}}%
\pgfpathlineto{\pgfqpoint{0.000000in}{-0.048611in}}%
\pgfusepath{stroke,fill}%
}%
\begin{pgfscope}%
\pgfsys@transformshift{3.123361in}{0.320679in}%
\pgfsys@useobject{currentmarker}{}%
\end{pgfscope}%
\end{pgfscope}%
\begin{pgfscope}%
\definecolor{textcolor}{rgb}{0.000000,0.000000,0.000000}%
\pgfsetstrokecolor{textcolor}%
\pgfsetfillcolor{textcolor}%
\pgftext[x=3.123361in,y=0.223457in,,top]{\color{textcolor}\rmfamily\fontsize{10.000000}{12.000000}\selectfont \(\displaystyle 0.04\)}%
\end{pgfscope}%
\begin{pgfscope}%
\pgfsetbuttcap%
\pgfsetroundjoin%
\definecolor{currentfill}{rgb}{0.000000,0.000000,0.000000}%
\pgfsetfillcolor{currentfill}%
\pgfsetlinewidth{0.803000pt}%
\definecolor{currentstroke}{rgb}{0.000000,0.000000,0.000000}%
\pgfsetstrokecolor{currentstroke}%
\pgfsetdash{}{0pt}%
\pgfsys@defobject{currentmarker}{\pgfqpoint{0.000000in}{-0.048611in}}{\pgfqpoint{0.000000in}{0.000000in}}{%
\pgfpathmoveto{\pgfqpoint{0.000000in}{0.000000in}}%
\pgfpathlineto{\pgfqpoint{0.000000in}{-0.048611in}}%
\pgfusepath{stroke,fill}%
}%
\begin{pgfscope}%
\pgfsys@transformshift{3.987426in}{0.320679in}%
\pgfsys@useobject{currentmarker}{}%
\end{pgfscope}%
\end{pgfscope}%
\begin{pgfscope}%
\definecolor{textcolor}{rgb}{0.000000,0.000000,0.000000}%
\pgfsetstrokecolor{textcolor}%
\pgfsetfillcolor{textcolor}%
\pgftext[x=3.987426in,y=0.223457in,,top]{\color{textcolor}\rmfamily\fontsize{10.000000}{12.000000}\selectfont \(\displaystyle 0.05\)}%
\end{pgfscope}%
\begin{pgfscope}%
\pgfsetbuttcap%
\pgfsetroundjoin%
\definecolor{currentfill}{rgb}{0.000000,0.000000,0.000000}%
\pgfsetfillcolor{currentfill}%
\pgfsetlinewidth{0.803000pt}%
\definecolor{currentstroke}{rgb}{0.000000,0.000000,0.000000}%
\pgfsetstrokecolor{currentstroke}%
\pgfsetdash{}{0pt}%
\pgfsys@defobject{currentmarker}{\pgfqpoint{0.000000in}{-0.048611in}}{\pgfqpoint{0.000000in}{0.000000in}}{%
\pgfpathmoveto{\pgfqpoint{0.000000in}{0.000000in}}%
\pgfpathlineto{\pgfqpoint{0.000000in}{-0.048611in}}%
\pgfusepath{stroke,fill}%
}%
\begin{pgfscope}%
\pgfsys@transformshift{4.851491in}{0.320679in}%
\pgfsys@useobject{currentmarker}{}%
\end{pgfscope}%
\end{pgfscope}%
\begin{pgfscope}%
\definecolor{textcolor}{rgb}{0.000000,0.000000,0.000000}%
\pgfsetstrokecolor{textcolor}%
\pgfsetfillcolor{textcolor}%
\pgftext[x=4.851491in,y=0.223457in,,top]{\color{textcolor}\rmfamily\fontsize{10.000000}{12.000000}\selectfont \(\displaystyle 0.06\)}%
\end{pgfscope}%
\begin{pgfscope}%
\pgfsetbuttcap%
\pgfsetroundjoin%
\definecolor{currentfill}{rgb}{0.000000,0.000000,0.000000}%
\pgfsetfillcolor{currentfill}%
\pgfsetlinewidth{0.803000pt}%
\definecolor{currentstroke}{rgb}{0.000000,0.000000,0.000000}%
\pgfsetstrokecolor{currentstroke}%
\pgfsetdash{}{0pt}%
\pgfsys@defobject{currentmarker}{\pgfqpoint{-0.048611in}{0.000000in}}{\pgfqpoint{0.000000in}{0.000000in}}{%
\pgfpathmoveto{\pgfqpoint{0.000000in}{0.000000in}}%
\pgfpathlineto{\pgfqpoint{-0.048611in}{0.000000in}}%
\pgfusepath{stroke,fill}%
}%
\begin{pgfscope}%
\pgfsys@transformshift{0.475001in}{0.592391in}%
\pgfsys@useobject{currentmarker}{}%
\end{pgfscope}%
\end{pgfscope}%
\begin{pgfscope}%
\definecolor{textcolor}{rgb}{0.000000,0.000000,0.000000}%
\pgfsetstrokecolor{textcolor}%
\pgfsetfillcolor{textcolor}%
\pgftext[x=0.100000in,y=0.544166in,left,base]{\color{textcolor}\rmfamily\fontsize{10.000000}{12.000000}\selectfont \(\displaystyle 1000\)}%
\end{pgfscope}%
\begin{pgfscope}%
\pgfsetbuttcap%
\pgfsetroundjoin%
\definecolor{currentfill}{rgb}{0.000000,0.000000,0.000000}%
\pgfsetfillcolor{currentfill}%
\pgfsetlinewidth{0.803000pt}%
\definecolor{currentstroke}{rgb}{0.000000,0.000000,0.000000}%
\pgfsetstrokecolor{currentstroke}%
\pgfsetdash{}{0pt}%
\pgfsys@defobject{currentmarker}{\pgfqpoint{-0.048611in}{0.000000in}}{\pgfqpoint{0.000000in}{0.000000in}}{%
\pgfpathmoveto{\pgfqpoint{0.000000in}{0.000000in}}%
\pgfpathlineto{\pgfqpoint{-0.048611in}{0.000000in}}%
\pgfusepath{stroke,fill}%
}%
\begin{pgfscope}%
\pgfsys@transformshift{0.475001in}{1.124190in}%
\pgfsys@useobject{currentmarker}{}%
\end{pgfscope}%
\end{pgfscope}%
\begin{pgfscope}%
\definecolor{textcolor}{rgb}{0.000000,0.000000,0.000000}%
\pgfsetstrokecolor{textcolor}%
\pgfsetfillcolor{textcolor}%
\pgftext[x=0.100000in,y=1.075965in,left,base]{\color{textcolor}\rmfamily\fontsize{10.000000}{12.000000}\selectfont \(\displaystyle 2000\)}%
\end{pgfscope}%
\begin{pgfscope}%
\pgfsetbuttcap%
\pgfsetroundjoin%
\definecolor{currentfill}{rgb}{0.000000,0.000000,0.000000}%
\pgfsetfillcolor{currentfill}%
\pgfsetlinewidth{0.803000pt}%
\definecolor{currentstroke}{rgb}{0.000000,0.000000,0.000000}%
\pgfsetstrokecolor{currentstroke}%
\pgfsetdash{}{0pt}%
\pgfsys@defobject{currentmarker}{\pgfqpoint{-0.048611in}{0.000000in}}{\pgfqpoint{0.000000in}{0.000000in}}{%
\pgfpathmoveto{\pgfqpoint{0.000000in}{0.000000in}}%
\pgfpathlineto{\pgfqpoint{-0.048611in}{0.000000in}}%
\pgfusepath{stroke,fill}%
}%
\begin{pgfscope}%
\pgfsys@transformshift{0.475001in}{1.655990in}%
\pgfsys@useobject{currentmarker}{}%
\end{pgfscope}%
\end{pgfscope}%
\begin{pgfscope}%
\definecolor{textcolor}{rgb}{0.000000,0.000000,0.000000}%
\pgfsetstrokecolor{textcolor}%
\pgfsetfillcolor{textcolor}%
\pgftext[x=0.100000in,y=1.607764in,left,base]{\color{textcolor}\rmfamily\fontsize{10.000000}{12.000000}\selectfont \(\displaystyle 3000\)}%
\end{pgfscope}%
\begin{pgfscope}%
\pgfsetbuttcap%
\pgfsetroundjoin%
\definecolor{currentfill}{rgb}{0.000000,0.000000,0.000000}%
\pgfsetfillcolor{currentfill}%
\pgfsetlinewidth{0.803000pt}%
\definecolor{currentstroke}{rgb}{0.000000,0.000000,0.000000}%
\pgfsetstrokecolor{currentstroke}%
\pgfsetdash{}{0pt}%
\pgfsys@defobject{currentmarker}{\pgfqpoint{-0.048611in}{0.000000in}}{\pgfqpoint{0.000000in}{0.000000in}}{%
\pgfpathmoveto{\pgfqpoint{0.000000in}{0.000000in}}%
\pgfpathlineto{\pgfqpoint{-0.048611in}{0.000000in}}%
\pgfusepath{stroke,fill}%
}%
\begin{pgfscope}%
\pgfsys@transformshift{0.475001in}{2.187789in}%
\pgfsys@useobject{currentmarker}{}%
\end{pgfscope}%
\end{pgfscope}%
\begin{pgfscope}%
\definecolor{textcolor}{rgb}{0.000000,0.000000,0.000000}%
\pgfsetstrokecolor{textcolor}%
\pgfsetfillcolor{textcolor}%
\pgftext[x=0.100000in,y=2.139564in,left,base]{\color{textcolor}\rmfamily\fontsize{10.000000}{12.000000}\selectfont \(\displaystyle 4000\)}%
\end{pgfscope}%
\begin{pgfscope}%
\pgfpathrectangle{\pgfqpoint{0.475001in}{0.320679in}}{\pgfqpoint{5.037500in}{2.310000in}}%
\pgfusepath{clip}%
\pgfsetrectcap%
\pgfsetroundjoin%
\pgfsetlinewidth{1.505625pt}%
\definecolor{currentstroke}{rgb}{0.000000,0.000000,1.000000}%
\pgfsetstrokecolor{currentstroke}%
\pgfsetdash{}{0pt}%
\pgfpathmoveto{\pgfqpoint{2.259295in}{1.505583in}}%
\pgfpathlineto{\pgfqpoint{3.987426in}{2.525679in}}%
\pgfusepath{stroke}%
\end{pgfscope}%
\begin{pgfscope}%
\pgfpathrectangle{\pgfqpoint{0.475001in}{0.320679in}}{\pgfqpoint{5.037500in}{2.310000in}}%
\pgfusepath{clip}%
\pgfsetrectcap%
\pgfsetroundjoin%
\pgfsetlinewidth{1.505625pt}%
\definecolor{currentstroke}{rgb}{1.000000,0.549020,0.000000}%
\pgfsetstrokecolor{currentstroke}%
\pgfsetdash{}{0pt}%
\pgfpathmoveto{\pgfqpoint{2.691328in}{0.788401in}}%
\pgfpathlineto{\pgfqpoint{5.283524in}{1.444479in}}%
\pgfusepath{stroke}%
\end{pgfscope}%
\begin{pgfscope}%
\pgfpathrectangle{\pgfqpoint{0.475001in}{0.320679in}}{\pgfqpoint{5.037500in}{2.310000in}}%
\pgfusepath{clip}%
\pgfsetrectcap%
\pgfsetroundjoin%
\pgfsetlinewidth{1.505625pt}%
\definecolor{currentstroke}{rgb}{0.133333,0.545098,0.133333}%
\pgfsetstrokecolor{currentstroke}%
\pgfsetdash{}{0pt}%
\pgfpathmoveto{\pgfqpoint{0.703978in}{0.425679in}}%
\pgfpathlineto{\pgfqpoint{2.259295in}{1.132980in}}%
\pgfusepath{stroke}%
\end{pgfscope}%
\begin{pgfscope}%
\pgfsetrectcap%
\pgfsetmiterjoin%
\pgfsetlinewidth{0.803000pt}%
\definecolor{currentstroke}{rgb}{0.000000,0.000000,0.000000}%
\pgfsetstrokecolor{currentstroke}%
\pgfsetdash{}{0pt}%
\pgfpathmoveto{\pgfqpoint{0.475001in}{0.320679in}}%
\pgfpathlineto{\pgfqpoint{0.475001in}{2.630679in}}%
\pgfusepath{stroke}%
\end{pgfscope}%
\begin{pgfscope}%
\pgfsetrectcap%
\pgfsetmiterjoin%
\pgfsetlinewidth{0.803000pt}%
\definecolor{currentstroke}{rgb}{0.000000,0.000000,0.000000}%
\pgfsetstrokecolor{currentstroke}%
\pgfsetdash{}{0pt}%
\pgfpathmoveto{\pgfqpoint{5.512501in}{0.320679in}}%
\pgfpathlineto{\pgfqpoint{5.512501in}{2.630679in}}%
\pgfusepath{stroke}%
\end{pgfscope}%
\begin{pgfscope}%
\pgfsetrectcap%
\pgfsetmiterjoin%
\pgfsetlinewidth{0.803000pt}%
\definecolor{currentstroke}{rgb}{0.000000,0.000000,0.000000}%
\pgfsetstrokecolor{currentstroke}%
\pgfsetdash{}{0pt}%
\pgfpathmoveto{\pgfqpoint{0.475001in}{0.320679in}}%
\pgfpathlineto{\pgfqpoint{5.512501in}{0.320679in}}%
\pgfusepath{stroke}%
\end{pgfscope}%
\begin{pgfscope}%
\pgfsetrectcap%
\pgfsetmiterjoin%
\pgfsetlinewidth{0.803000pt}%
\definecolor{currentstroke}{rgb}{0.000000,0.000000,0.000000}%
\pgfsetstrokecolor{currentstroke}%
\pgfsetdash{}{0pt}%
\pgfpathmoveto{\pgfqpoint{0.475001in}{2.630679in}}%
\pgfpathlineto{\pgfqpoint{5.512501in}{2.630679in}}%
\pgfusepath{stroke}%
\end{pgfscope}%
\end{pgfpicture}%
\makeatother%
\endgroup%

		\caption{Plot of the relationship between beginning amount of \koh and the evolved heat from the reaction.
		Blue is the Part II reaction, and produces a slope of 95909 joules per mole of starting \koh.
		Green is the Part III reaction, and produces a slope of 73890 joules per mole of starting \koh
		Orange is the Part I reaction, and produces a slope of 41120 joules per mole of starting \koh.
	}\label{fig:lin}
	\end{center}
\end{figure}

