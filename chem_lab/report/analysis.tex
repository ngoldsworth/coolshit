\section{Analysis and Interpretation}

The three graphs are mostly similar in shape.
\Cref{fig:partI}, \Cref{fig:partII},  and \Cref{fig:partIIIb} all show a period of low temperature,
followed by a short period of increasing temperature, then followed by a plateau at a higher temperature.
This commonality between the three, and that temperature is increasing in each, 
shows that energy is being released into the solution by similar mechanism in each associated reaction.
\Cref{fig:partIIIb}, while it looks more jagged and random (due to $y$-axis scaling),
shows that the temperature of the \koh stayed roughly the same
during the two minutes during which nothing was added to the solution --the expected result. 

The data in \Cref{tab:main} was gathered from all six groups completing the experiment. 
In each part, the amount of \koh was determined via a molar mass conversion ratio:
\begin{subequations}
\begin{eqnarray}
\textrm{mass \koh} \times \frac{1}{\textrm{molar mass of \koh}} &= \textrm{amount \koh}\label{eq:molmass}\\
\SI{2.991}{g} \times \frac{\SI{1}{mol}}{\SI{56.1053}{g}} &= \SI{0.05331}{mol}\label{eq:molarmassnums}\\
\si{g} \times \frac{\si{mol}}{\si{g}} = \si{mol}\label{eq:molarmassunits}
\end{eqnarray}
\end{subequations}
where \eqref{eq:molmass} describes the general computation,
\eqref{eq:molarmassnums} is a sample calculation,
and \eqref{eq:molarmassunits} is a unit analysis of the computation.
The heat of a reaction is given by
\begin{subequations}
\begin{eqnarray}
Q = m\cdot c\cdot\Delta T = \left(\rho \cdot V\right)\cdot c \Delta T \label{eq:mcat}\\
1943.\space\si{J} = (\SI{1}{g/mL} \cdot \SI{100.0}{mL})
		\cdot \frac{4.184 \si{J}}{\si{g\cdot\degreeCelsius}}\label{eq:mcatsamp}
		\cdot \SI{4.644}{\degreeCelsius}\\
\si{Joule}=\left(\frac{\si{g}}{\si{mL}} \cdot \si{mL} \right)\cdot \frac{\si{J}}{\si{g\cdot\degreeCelsius}} \cdot \si{\degreeCelsius}\label{eq:mcatunits}
\end{eqnarray}
\end{subequations}
Where, in \eqref{eq:mcat}, $Q$ is the heat in joules, $m$ is the mass (obtained
from the solution volume $V$ and the density of water:$\rho = \SI{1}{g/mL}$),
$c$ is the specific heat capacity of water, and $\Delta T$ is the change in
temperature over the course of the reaction. \eqref{eq:mcatsamp} is a sample
calculation, and \eqref{eq:mcatunits} is the unit analysis.

The heat per mole of \koh is determined by diving the $Q$
in \eqref{eq:mcat} and dividing by the number of moles of \koh present in that reaction.
This turns out as:
\begin{subequations}
\begin{eqnarray}
\textrm{Heat per mol KOH} = \frac{Q}{\textrm{moles KOH}}\label{eq:permol}\\
36450.\quad \space\si{\joule\per\mol} = \frac{1943.\quad\si{\joule}}{\SI{0.05531}{\mol}}\label{eq:permolsamp}\\
\si{\joule\per\mol} = \frac{\si{\joule}}{\si{\mol}}\label{eq:permolunits}
\end{eqnarray}
\end{subequations}
Where \eqref{eq:permol} is the general computation,
\eqref{eq:permolsamp} is a sample calculation,
and \eqref{eq:permolunits} is a unit analysis.

In \Cref{tab:main}, Group 6 is the author's groups, data.
Compared to the other groups' data, group 6's data fits well with the linear relationship 
shown in \Cref{fig:lin}. Overall, the linear fit produces an equation of
\begin{equation}\label{eq:lin}
	Q(n) = 95910n - 160.122\\
\end{equation}
where $Q$ is the heat evolved (in joules) by a starting amount of $n$ moles of \koh.

It appears that the average heights of the orange and green lines add together to produce the average height of the blue line,
meaning that the heats from the Part I reactions plus the heats from the Part III reactions are about equivalent to the heats produced by the Part II reaction. 
This makes sense, as the chemical reactions are similar in each of the three parts, but just involve a different state of the same substances.
Parts I and III begin and finish, respectively, in the same state that Part II begins and ends. 





	

